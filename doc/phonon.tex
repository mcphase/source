\section{Including Phonons and Crystal-Field Phonon interactions}\label{phonons}

In this section we discuss, how lattice dynamics may be considered in the framework of
the Hamiltonian (\ref{fullhamiltonian}). We will see that this corresponds 
to a system of coupled Einstein oscillators. One such oscillator can be modelled 
by setting up a {\prg sipf} file with the module {\prg phonon}. Coupling has
to be done in {\prg mcdisp.j}. Rephrasing lattice dynamics in this way allows 
to couple phonons to the crystal field.

A three dimensional Einstein oscillator (for atom $n$) in a solid can be described by 
the following Hamiltonian

\begin{equation}\label{einstein}
\hat H_E(n)=\frac{a_0^2{\hat \mbf p_n}^2}{2m_n} - \frac{1}{2} {\hat \mbf u}^T_n \m{K}(nn) {\hat \mbf u}_n
\end{equation}

Here $\hat \mbf u$ is the dimensionless displacement vector 
($\hat \mbf u_n={\hat \mbf P}_n/a_0=\Delta {\hat \mbf r}_n/a_0$, 
with the Bohr radius $a_0=0.5219$~\AA), $m_n$ the
mass of the atom $n$, $\hat \mbf p_n=d\hat \mbf u_n/dt$ the conjugate momentum to $\hat \mbf u_n$ and
$\m{K}(nn)$ the Matrix describing the restoring force.

Coupling such oscillators leads to the Hamiltonian

\begin{equation}
\hat H_{phon}=\sum_n \hat H_E(n) -\frac{1}{2} \sum_{n\neq n'} {\hat \mbf u}_n^T \m{K}(nn')  {\hat \mbf u}_{n'}
\end{equation}

Note that our coupling constants $K_{\alpha\beta}(nn')=-A_{\alpha\beta}(nn')$, where
 $A_{\alpha\beta}$ are the second-order derivatives of the potenatial energy
as defined e.g. in \cite[page 99]{lovesey84-1}.

In a mean field type of theory 
the phonon single ion module has thus to solve the Hamiltonian

\begin{equation}\label{phonsiham}
\hat H_E=\frac{a_0^2{\hat \mbf p}^2}{2m} - \frac{1}{2} {\hat \mbf u}^T \m{K} {\hat \mbf u} - {\mbf F}^T {\hat \mbf u}
\end{equation}

Here the force $\mbf F$ corresponds to the exchange field $\mbf H_{xc}$ and $\hat \mbf u$ to
 the general operator $\hat \mbf I$ and $\m{K}(nn')$ to $\mathcal J(nn')$ of equation (\ref{fullhamiltonian}),
respectively. The single ion Hamiltonian (\ref{phonsiham}) can be solved by transforming
it to normal coordinates (main axis of the Einstein oscillator) using the transformation
matrix $\m{S}$, which diagonalises $\m{K}=\m{S}^T\m{\Omega}\m{S}$:

\begin{equation}
 \hat \mbf u'=\m{S} \hat \mbf u +\m{\Omega}^{-1} \m{S} \mbf F
\end{equation}

\begin{equation}\label{phonsihamdiag}
\hat H_E=\frac{a_0^2{\hat \mbf p}^{'2}}{2m}-\frac{1}{2} {\hat \mbf u}^{'T} \m{\Omega} {\hat \mbf u'} 
-\frac{1}{2}\mbf F^T  \mbf u_0
\end{equation}

Due to the action of the force $\mbf F$ the equilibrium position of the oscillator
is $\mbf u_0=-\m{S}^T\m{\Omega}^{-1}\m{S}\mbf F$ (it is the task of the function
{\prg Icalc} to return this equilibrium position), the energies correspond to the three elements
of the diagonal matrix $\m{\Omega}$, i.e. $\Omega_{11}=-m a_0^2 (\Delta_1 /\hbar)^2$,
$\Omega_{22}=-m a_0^2 (\Delta_2 /\hbar)^2$,
$\Omega_{33}=-m a_0^2 (\Delta_3 /\hbar)^2$. In order to run {\prg mcdisp} we
have to calculate the transition matrix elements:

The single ion susceptibility for such a transition, e.g. $\Delta_1$ - corresponds to

\begin{eqnarray}
\m{\chi}^0&=&\sum_{\nu\mu}\frac{\langle \nu|\hat \mbf u|\mu\rangle\langle \mu |\hat \mbf u^T|\nu \rangle}{\Delta_1 -\hbar \omega}
(p_{\nu} -p_{\mu}) \\
&=&\sum_{\nu\mu}\frac{\langle \nu|\m{S}^T \hat \mbf u'|\mu\rangle\langle \mu |\hat \mbf u'^T \m{S}|\nu \rangle}{\Delta_1 -\hbar \omega}
(p_{\nu} -p_{\mu}) 
\end{eqnarray}
Because the different components of $\mbf u'$ commute and the Hamiltonian (\ref{phonsihamdiag})
is separable, for the transition $\Delta_1$ only the terms with $u_1'$ in the nominator
contribute:

\begin{eqnarray}
\chi^0_{\alpha\beta}&=&S^T_{\alpha1}\sum_{\nu\mu}\frac{\langle \nu|\hat u_1'|\mu\rangle\langle \mu |\hat  u_1'|\nu \rangle}{\Delta_1 -\hbar \omega}
(p_{\nu} -p_{\mu}) S_{1\beta}\\
&=& S^T_{\alpha1}S_{1\beta}\frac{\hbar^2}{2ma_0^2\Delta_1}\left(\frac{1}{\Delta_1-\hbar\omega}+\frac{1}{\Delta_1+\hbar\omega}\right )
\end{eqnarray}

In order to derive the last result we had to express $\hat u_1'$ in terms of ladder  operators
$\hat u_1'=a_0^{-1} \hbar/\sqrt{2m\Delta_1}(\hat a+\hat a^{\dagger})$ and  apply $\hat a^{\dagger}|\nu\rangle=\sqrt{\nu+1}|\nu+1\rangle$,
$\hat a|\nu\rangle=\sqrt{\nu}|\nu-1\rangle$ and $\sum_{\nu=0}^{\infty}(p_{\nu}-p_{\nu+1})(\nu+1)=1$,
$p_{\nu}=exp(-\nu\Delta_1/kT)(1-exp(-\Delta_1/kT))$. This shows that the single ion susceptibility
of our atom can be written as a sum of three effective transitions (with temperature independent
susceptibility)

\begin{eqnarray}
\chi^0_{\alpha\beta}
&=& \sum_{i=1,2,3} S^T_{\alpha i}S_{i\beta}\frac{\hbar^2}{2ma_0^2\Delta_i}
\left(\frac{1}{\Delta_i-\hbar\omega}+\frac{1}{\Delta_i+\hbar\omega}\right )
\end{eqnarray}

Thus the module {\prg phonon} has to provide in it's function {\prg du1calc} these three
transitions (=number of transitions).

\subsection{Using Single Ion Module {\prg phonon}}

The module {\prg phonon} allows to consider the phononic degrees of freedom in McPhas.
The single ion input file for an oscillating atom (with amplitude 
up to a maximum displacement $u_{max}$, which is relevant for the function {\prg Icalc} of the
module) has to have the following format:

\begin{verbatim}
#!MODULE=phonon
#<!--mcphase.sipf-->
#
# phonon
# MODPAR1=mass of atom in units of m0 (atomic mass unit=1.660539e-27 kg)
#
#-----------
MODPAR1=57  # mass in(m0)
MODPAR2=-1.2   # Kxx
MODPAR3=-1.2   # Kyy
MODPAR4=-1.2   # Kzz
MODPAR5=0   # Kxy  in (meV)
MODPAR6=0   # Kxz
MODPAR7=0   # Kyz
MODPAR8=1   # umax          maximum (cutoff) for displacement [a0=0.5219 A]
MODPAR9=4   # 0       umax restriction in all directions
            #                1,2,3   umax restriction in x y z direction only
            #                4       umax restriction in x and y direction
            #                5       umax restriction in x and z direction
            #                6       umax restriction in y and z direction

#-------------------------------------------------------
# Neutron Scattering Length (10^-12 cm) (can be complex)
#-------------------------------------------------------
SCATTERINGLENGTHREAL=0.769
SCATTERINGLENGTHIMAG=0
#  ... note: - if an occupancy other than 1.0 is needed, just reduce 
#              the scattering length linear accordingly

SCATTERINGLENGTHREAL=0.945
SCATTERINGLENGTHIMAG=0
\end{verbatim}

Note, that {\prg MODPAR8} and {\prg MODPAR9} impose a restriction to the {\prg Icalc} function
of the module {\prg phonon}. This means, if used with program {\prg mcphas} the
module will not return displacements $\langle \mbf u \rangle$ larger than umax.
Technically this is done by applying the operation $u \leftarrow u_{max} * tanh (u/u_{max})$
at the end of the function {\prg Icalc}. 

{\bf Coordinate System:} the Euclidean coordinate system of phonon displacements $\mbf u$ with
components $u_x,u_y,u_z$ is oriented with respect to the crystal axes $\mbf a,\mbf b, \mbf c$
such that $\mbf y || \mbf b$, $\mbf z || \mbf a \times \mbf b $ and $\mbf x$ perpendicular
to $\mbf y$ and $\mbf z$.

The single ion property file contains the matrix $\m{K}(nn)$, the matrix $\m{K}(nn')$ decribing the
forces between
different ions $n$ and $n'$ have to be given in the file {\prg mcphas.j}, which could
look like:

\begin{verbatim}
# 
#<!--mcphase.mcphas.j-->
*************************************************************
# Lattice Constants (A)
#! a=4.047 b=4.047 c=9.612 alpha=  90 beta=  90 gamma=  90
#! r1a=   1 r2a=   0 r3a= 0.5
#! r1b=   0 r2b=   1 r3b= 0.5   primitive lattice vectors [a][b][c]
#! r1c=   0 r2c=   0 r3c= 0.5
# Nonzero Elastic constants in meV per primitive crystal unit cell 
# in Voigt notation only first index<=second index has to be given
# because the constants are symmetric Celij=Celji
# Elastic constants refer to the Euclidean coordinate system ijk defined
# with respect to abc as j||b, k||(a x b) and i normal to k and j
#!  Cel11=+129702.248 Cel12=+63972.0533 Cel13=+34346.9127
#!  Cel22=+129702.248 Cel23=+34346.9127
#!  Cel33=+60842.4597
#!  Cel44=+137509.379
#!  Cel55=+137509.379
#!  Cel66=+256056.495
#! nofatoms=1  nofcomponents=3  number of atoms in primitive unit cell/number of components of each spin
#*********************************************************************
#! da=   0 [a] db=   0 [b] dc=   0 [c] nofneighbours=2 diagonalexchange=1 sipffilename=phonon.sipf
# it follows the Born von Karman model according to  springs read from file 
# the mixing terms Gmix in meV/a0  with a0=0.5292e-10 m
#! Gindices= 1,1 1,2 1,3 2,1 2,2 2,3 3,1 3,2 3,3 4,1 4,2 4,3 5,1 5,2 5,3 6,1 6,2 6,3
#! G= +1 13 0  0 0 0  0 0 0   0 0 0   0 0 0  0 0 0
#da[a]   db[b]     dc[c]       Jaa[meV]  Jbb[meV]  Jcc[meV]  Jab[meV]  Jba[meV]  Jac[meV]  Jca[meV]  Jbc[meV]  Jcb[meV]
0 0 1 1.1 1.1 1.1
0 0 -1 1.1 1.1 1.1 
#*********************************************************************
\end{verbatim}

Note that this file may optionally contain elastic constants and mixing
term parameters, which are needed for the calculation
of magnetoelastic properties (see section \ref{JT}).
It is planned that these files should be created automatically from the 
output of DFT programs. [to be done]. Currently it is possible to create 
the input parameters from Born v Karman longitudinal and transversal springs
by the program {\prg makenn} with option {\prg -bvk}.
For an example of phonons in tungsten see {\prg examples/tungsten\_phonons }

If crystal field - phonon coupling is to be input, magnetic ions may be added to {\prg mcphas.j}.
They should not be placed at exactly the same position as the phonon atoms - i.e. da db dc should be chosen
(slightly) different to enable the mcphas.j loader to identify which  is which.

The coupling between phononic and crystal field degrees of freedom can be derived from
$\hat H_{cf-phon}=\sum_{nn'}(\mbf \nabla_{\hat \mbf u(n')} B_l^m(n))   \mbf u(n')  \hat O_{lm}(\hat \mbf J_n)$.
 This can be done
using the program  {\prg makenn} with option {\prg -cfph}  applying small differential changes to
the atomic positions in the unit cell in order to evaluate the gradient of the crystal field
parameters. For examples see  several models on crystal field phonon interaction 
in {\prg examples}, for starting a linear chain of Ce$^{3+}$ ions including a
theory manuscript can be found in {\prg examples/Ce3p\_chain\_cfphonon}.



\section{Crystal Field Phonon Interaction}\label{JT}


Here some formulas are listed. Starting point is the phonon part,
 written in more general terms. 


 The index $i$ now denotes all atomic
positions in a crystal, the momentum vector is $a_0 \vec p_i$ and  the
displacement vector is $\vec U_i$. Fig.~\ref{figbvk} shows the Born-van-Karman
model definition of longitudinal springs $c_L$ and transversal
springs $c_T$ used in the following derivation.

\begin{eqnarray}\label{Hphonon}
H_{\rm ph}& =&\sum_{i} \frac{a_0^2 \vec p_i^2}{2 m_i} + 
\frac{1}{2}\sum_{ij} \frac{c_L(ij)-c_T(ij)}{2|\vec R_{ij}|^2} 
(\vec U_j . \vec R_{ij}- \vec U_i . \vec R_{ij})^2 \nonumber \\
      && + \frac{c_T(ij)}{2} (\vec U_j - \vec U_i )^2 \\
 \end{eqnarray}

Note that the sum over indices $i,j$ counts each spring twice, thus a factor of $1/2$
has been added to the  Hamiltonian before the sum.

Note that the difference in undisplaced lattice positions is denoted by $\vec R_{ij}=\vec R_{j}-\vec R_{i}$.
The displacement vector $\vec U_i$  is split into 
a strain component $\bar \epsilon \vec R_i$ a rotational component
$\bar \omega \vec R_i$ and a dynamic component $\vec P_i$
obeying periodic boundary conditions 
$\vec U_i = \bar \epsilon \vec R_i + \bar \omega \vec R_i+ \vec P_i= \bar a \vec R_i+ \vec P_i$.




\begin{eqnarray}\label{Ur2}
(\vec U_j . \vec R_{ij}- \vec U_i . \vec R_{ij})^2 &=&
 (\vec R_{ij}^T\bar a \vec R_{ij}+ \vec R_{ij}^T \vec P_{ij})^2 \nonumber \\
&=& (\vec R_{ij}^T\bar a \vec R_{ij})^2 + 2 \vec R_{ij}^T\bar a \vec R_{ij}\vec R_{ij}^T \vec P_{ij} \nonumber \\
&&+ (\vec R_{ij}^T \vec P_{ij})^2
\end{eqnarray}

\begin{eqnarray}\label{U2}
(\vec U_j - \vec U_i )^2 &=& \vec  R_{ij}^T\bar a^T\bar a \vec R_{ij} \nonumber \\
&&2 \vec  R_{ij}^T\bar a^T\vec P_{ij} + \vec P_{ij}^T \vec P_{ij}
\end{eqnarray}

Equations (\ref{Ur2}) and (\ref{U2}) contain terms quadratic in $\bar a$, which will be
considered as elastic energy contributions. The terms quadratic in $\vec P$ will contribute
to the lattice dynamics. In addition there are also 
terms linear in $\bar a$ and $\vec P$. 
Thus the phonon Hamiltonian (\ref{Hphonon}) can be separated into the elastic Energy $E_{el}$, the
Einstein single ion oscillation term $H_{\rm E}$, the bilinear interaction term $H_{\rm int}$
and the mixing term $H_{mix}$ 

\begin{eqnarray}
H_{\rm ph}& =&E_{el} + H_{mix}+ H_{\rm E}+H_{\rm int}
\end{eqnarray}

\subsection{Elastic Energy $E_{el}$}

The elastic energy $E_{el}$ is bilinear in  $a$

\begin{eqnarray}\label{Eel}
E_{el} &=& \frac{1}{2}\sum_{ij} \frac{c_L(ij)-c_T(ij)}{2|\vec R_{ij}|^2} 
(\vec R_{ij}^T\bar a \vec R_{ij})^2  \\
      && + \frac{c_T(ij)}{2} \vec  R_{ij}^T\bar a^T\bar a \vec R_{ij} \nonumber \\
&=& \frac{1}{2} \sum_{ij,\alpha\beta\gamma\delta} \frac{c_L(ij)-c_T(ij)}{2|\vec R_{ij}|^2} 
R_{ij}^{\alpha}R_{ij}^{\beta}R_{ij}^{\gamma}R_{ij}^{\delta}
a_{\alpha\beta}a_{\gamma\delta} \nonumber \\
&& + \frac{c_T(ij)}{2} R_{ij}^{\alpha} R_{ij}^{\delta}
a_{\alpha\beta} \delta_{\beta\gamma} a_{\gamma\delta}  \nonumber
\end{eqnarray}

We calculate it's derivative with respect to $a_{\alpha\beta}$:

\begin{eqnarray}
\frac{\partial E_{el}}{\partial a_{\alpha\beta}}&=& \frac{1}{2}\sum_{ij,\gamma\delta} \frac{c_L(ij)-c_T(ij)}{|\vec R_{ij}|^2} 
R_{ij}^{\alpha}R_{ij}^{\beta}R_{ij}^{\gamma}R_{ij}^{\delta}
a_{\gamma\delta} \nonumber \\
&& +    \frac{c_T(ij)}{2} R_{ij}^{\alpha}R_{ij}^{\delta}
 \delta_{\beta\gamma} a_{\gamma\delta} + \nonumber \\
&& \frac{c_T(ij)}{2} R_{ij}^{\gamma} R_{ij}^{\beta}
a_{\gamma\delta} \delta_{\delta\alpha} 
\end{eqnarray}


%derivatives with respect to the strain component $\epsilon_{\alpha\beta}$ 
%with  $\alpha,\beta=1,2,3,\alpha \le \beta$ can be written as

%\begin{eqnarray}
%\frac{\partial}{\partial \epsilon_{\alpha\beta}}&=&
%\sum_{\gamma\delta=1,2,3}\frac{\partial a_{\gamma\delta}}{\partial \epsilon_{\alpha\beta}}
%\frac{\partial}{\partial a_{\gamma\delta}} \\
%&=&
%\frac{\partial}{\partial a_{\alpha\beta}} + (1-\delta_{\alpha\beta})
%\frac{\partial}{\partial a_{\beta\alpha}}  
%\end{eqnarray}

%We calculate the derivative of the elastic energy with respect to the strain:


%\begin{eqnarray}\label{Eelderiv}
%\frac{\partial E_{el}}{\partial \epsilon_{\alpha\beta}}&=&
%\frac{\partial E_{el}}{\partial a_{\alpha\beta}} + \frac{\partial E_{el}}{\partial a_{\beta\alpha}}-\delta_{\alpha\beta}
%\frac{\partial E_{el}}{\partial a_{\alpha\alpha}}  \\ 
%&=&\frac{1}{2}\sum_{ij,\gamma\delta} (2-\delta_{\alpha\beta})\frac{c_L(ij)-c_T(ij)}{|\vec R_{ij}|^2} 
%R_{ij}^{\alpha}R_{ij}^{\beta}R_{ij}^{\gamma}R_{ij}^{\delta} 
%a_{\gamma\delta} \nonumber \\
%& & +    \frac{c_T(ij)}{2} a_{\gamma\delta} ( R_{ij}^{\alpha}R_{ij}^{\delta}
% \delta_{\beta\gamma}  +   R_{ij}^{\gamma} R_{ij}^{\beta}
%\delta_{\delta\alpha}  \nonumber \\
%&& +    R_{ij}^{\beta} R_{ij}^{\delta}
% \delta_{\alpha\gamma}  +   R_{ij}^{\gamma}R_{ij}^{\alpha}
% \delta_{\delta\beta} \nonumber \\
%& & -    R_{ij}^{\alpha}R_{ij}^{\delta}
% \delta_{\alpha\beta}\delta_{\beta\gamma}    -  R_{ij}^{\gamma} R_{ij}^{\beta}
% \delta_{\delta\alpha} \delta_{\alpha\beta} ) \nonumber
%\end{eqnarray}



An we make also use of the definition of elastic constants:

\begin{eqnarray}\label{elconst}
c^{\alpha\beta\gamma\delta} &= &
\frac{\partial^2E_{el}}{\partial a_{\alpha\beta} \partial a_{\gamma\delta}} \\
&=& \frac{1}{2}\sum_{ij}\frac{c_L(ij)-c_T(ij)}{|\vec R_{ij}|^2} 
R_{ij}^{\alpha}R_{ij}^{\beta}R_{ij}^{\gamma}R_{ij}^{\delta}
 \nonumber \\
&& +    \frac{c_T(ij)}{2} R_{ij}^{\alpha}R_{ij}^{\delta}
 \delta_{\beta\gamma}  + \nonumber \\
&& \frac{c_T(ij)}{2} R_{ij}^{\gamma} R_{ij}^{\beta}
 \delta_{\delta\alpha} 
\end{eqnarray}
%& & +    c_T(ij)  ( R_{ij}^{\alpha}R_{ij}^{\delta}
% \delta_{\beta\gamma}  +   R_{ij}^{\gamma}R_{ij}^{\beta}
%\delta_{\delta\alpha}  \nonumber \\
%&& +    R_{ij}^{\beta}R_{ij}^{\delta}
% \delta_{\alpha\gamma}  +   R_{ij}^{\gamma}R_{ij}^{\alpha}
% \delta_{\delta\beta} \nonumber \\
%& & -    R_{ij}^{\alpha}R_{ij}^{\delta}
% \delta_{\alpha\beta}\delta_{\beta\gamma}    -  R_{ij}^{\gamma}R_{ij}^{\alpha}
% \delta_{\delta\alpha} \delta_{\alpha\beta} + \nonumber \\
%&& -    R_{ij}^{\beta}R_{ij}^{\alpha}
% \delta_{\alpha\gamma} \delta_{\gamma\delta}  -   R_{ij}^{\gamma}R_{ij}^{\alpha}
% \delta_{\delta\beta}\delta_{\gamma\delta}  \nonumber \\
%& & +    R_{ij}^{\alpha}R_{ij}^{\alpha}
% \delta_{\alpha\beta}\delta_{\beta\gamma} \delta_{\gamma\delta}   ) \nonumber



we make use of the fact that the strain $\bar \epsilon$ is a symmetric 
tensor ($\epsilon_{\alpha\beta}=\epsilon_{\beta\alpha}$) and a rotation
is antisymmetric ($\omega_{\alpha\beta}=-\omega_{\beta\alpha}$) and 
the linear transformation $\bar a$ can be written as $\bar a=\bar \epsilon + \bar \omega$. Thus
it is possible to rewrite the elastic energy in the well known fashion

\begin{eqnarray}\label{Eelconst}
E_{el} &=& \frac{1}{2}\sum_{\alpha\beta\gamma\delta} c^{\alpha\beta\gamma\delta} \epsilon_{\alpha\beta}\epsilon_{\gamma\delta}
\end{eqnarray}



Note that we have neglected the fact, that nonzero transversal springs will 
result in a dependence of the elastic energy on the rotation tensor $\bar \omega$ as
can been seen by inserting a rotation into the second part of (\ref{Eel}).
\footnote{$R^{\alpha}R^{\delta}\omega_{\sigma}
\epsilon_{\sigma\alpha\beta}\delta_{\beta\gamma}\omega_{\eta}\epsilon_{\eta\gamma\delta}=
R^{\alpha}R^{\delta}\omega_{\sigma} \omega_{\eta}
\epsilon_{\sigma\alpha\beta}\epsilon_{\eta\beta\delta}=
R^{\alpha}R^{\delta}\omega_{\sigma} \omega_{\eta}
(\delta_{\sigma\delta}\delta_{\alpha\eta}-\delta_{\sigma\eta}\delta_{\alpha\delta})=
(\vec R . \vec \omega)^2-R^2\omega^2 \neq 0$,
if $\vec R$ is not parallel to the rotation axis. Thus for some rotation the elastic energy
will depend on rotation angle if transversal springs are  introduced.
}
Therefore transversal springs have to be used with caution
 in the description of a phonon spectrum.

Note that the transversal spring part in the elastic constants as defined in (\ref{elconst}) 
is not symmetric upon exchange of indices. It is convenient to use a symmetrized version
of the elastic constants, which will yield the same elastic energy, because the
strain tensor is symmetric:

\begin{eqnarray}\label{elconstsymm}
c^{\alpha\beta\gamma\delta} &= &
 \frac{1}{2}\sum_{ij}\frac{c_L(ij)-c_T(ij)}{|\vec R_{ij}|^2} 
R_{ij}^{\alpha}R_{ij}^{\beta}R_{ij}^{\gamma}R_{ij}^{\delta}
 \nonumber \\
&& +    \frac{c_T(ij)}{4} ( R_{ij}^{\alpha}R_{ij}^{\delta}
 \delta_{\beta\gamma}  + \nonumber \\
&&  R_{ij}^{\gamma} R_{ij}^{\beta}
 \delta_{\delta\alpha}  \nonumber \\
&& +    R_{ij}^{\beta}R_{ij}^{\delta}
 \delta_{\alpha\gamma}  +   R_{ij}^{\gamma}R_{ij}^{\alpha}
 \delta_{\delta\beta} ) \nonumber \\
\end{eqnarray}



With the help of elastic constants we can rewrite the derivative of the elastic energy
(\ref{Eelconst}), again ignoring the rotation dependence:

\begin{eqnarray}\label{Eelderivelconst}
\frac{\partial E_{el}}{\partial \epsilon_{\alpha\beta}}&=&
\sum_{\gamma\delta=1,2,3} c^{\alpha\beta\gamma\delta} \epsilon_{\gamma\delta}
\end{eqnarray}




To easy the indexing we apply the notation of Voigt
\begin{eqnarray}\label{voigt}
\epsilon_1&=&\epsilon_{11}  \\
\epsilon_2&=&\epsilon_{22} \nonumber \\
\epsilon_3&=&\epsilon_{33}\nonumber \\
\epsilon_4&=&2\epsilon_{23}=2\epsilon_{32}\nonumber \\
\epsilon_5&=&2\epsilon_{31}=2\epsilon_{13}\nonumber \\
\epsilon_6&=&2\epsilon_{12}=2\epsilon_{21} \nonumber
\end{eqnarray}

Note the elastic constants do not contain any prefactor in Voigt notation, i.e.
\begin{eqnarray}\label{voigtel}
c^{11}&=&c^{1111} \\
c^{44}&=&c^{2323} \nonumber \\
\end{eqnarray}

There are 21 independent elastic constants, $c^{\alpha\beta}$ with $\beta=1,...,6$ and
$\alpha \le \beta$ (or less, see  http://koski.ucdavis.edu/BRILLOUIN/CRYSTALS/Elasticities.html). The other 60 elastic constants 
can be obtained from the symmetry relations $c^{\alpha\beta}=c^{\beta\alpha}$, 
$c^{\alpha\beta\gamma\delta}=c^{\beta\alpha\gamma\delta}=c^{\alpha\beta\delta\gamma}$.

The elastic energy in Voigt notation is given by

\begin{eqnarray}\label{Eelconstvoigt}
E_{el} &=& \frac{1}{2}\sum_{\alpha\gamma=1,..,6} c^{\alpha\gamma} \epsilon_{\alpha}\epsilon_{\gamma}
\end{eqnarray}

\subsection{Mixing term $H_{mix}$}

The mixing term $H_{mix}$ is 

\begin{eqnarray}\label{Hmix}
H_{mix}&=&\frac{1}{2} \sum_{ij} \frac{c_L(ij)-c_T(ij)}{|\vec R_{ij}|^2} 
\vec R_{ij}^T\bar a \vec R_{ij}\vec R_{ij}^T \vec P_{ij}  \nonumber \\
      && + c_T(ij) \vec  R_{ij}^T\bar a^T\vec P_{ij} \nonumber \\
&=& \frac{1}{2}\sum_{ij} 2\frac{c_L(ij)-c_T(ij)}{|\vec R_{ij}|^2} 
\vec R_{ij}^T\bar a \vec R_{ij}\vec R_{ij}^T \vec P_{i}  \nonumber \\
      && + 2 c_T(ij) \vec  R_{ij}^T\bar a^T\vec P_{i} \nonumber \\
&=& \sum_{ij,\alpha\beta} \frac{c_L(ij)-c_T(ij)}{|\vec R_{ij}|^2} 
R_{ij}^{\alpha} a_{\alpha\beta} R_{ij}^{\beta} R_{ij}^{\gamma}  P_{i}^{\gamma}  \nonumber \\
      && +  c_T(ij)   R_{ij}^{\beta} a_{\alpha\beta} \delta_{\alpha\gamma} P_{i}^{\gamma}  \\
&=&-\sum_{\stackrel{i,\alpha=1,..,6}{ \gamma=1,2,3}} G_{mix}^{\alpha\gamma}(i) \epsilon_{\alpha} P_{i}^{\gamma}/a_0 +
 \sum_{ij,\alpha\beta}  c_T(ij)   R_{ij}^{\beta} \omega_{\alpha\beta} P_{i}^{\alpha} \nonumber
\end{eqnarray}

Note that without crystal field phonon interaction the mixing term $H_{mix}$ is zero, because
the expectation values $\langle P_i \rangle$ are zero in this case. The last term reflects
the fact, that the energy is not invariant under rotations in case of transversal springs.

In (\ref{Hmix}) we made use of the phonon-strain interaction constants $G_{mix}^{\alpha\gamma}(i)$
defined writing explicitely the Voigt notation $\alpha=(\alpha\beta)$ by

\begin{eqnarray}\label{Gmix}
G_{mix}^{(\alpha\beta)\gamma}(i)&=&
-a_0\sum_{j} \frac{c_L(ij)-c_T(ij)}{|\vec R_{ij}|^2} 
R_{ij}^{\alpha} R_{ij}^{\beta} R_{ij}^{\gamma} -   \nonumber \\
      && -a_0\frac{1}{2}\sum_{j}  c_T(ij)   (R_{ij}^{\beta}  \delta_{\alpha\gamma}+
R_{ij}^{\alpha}  \delta_{\beta\gamma}) 
\end{eqnarray}

Note a symmetry property: 
 in case of inversion symmetry at the position of an ion, the 
phonon-strain interaction constants $G_{mix}^{\alpha\gamma}(i)$ will be zero.

\subsection{The lattice dynamic terms $H_{E}+H_{int}$}

Coming back to equations (\ref{Ur2}) and (\ref{U2}) we rewrite 
the terms quadratic in the phonon displacement operators 


\begin{eqnarray}
(\vec R_{ij}^T \vec P_{ij})^2&=&\sum_{\alpha,\beta=1,2,3} R_{ij}^{\alpha}R_{ij}^{\beta}(P_{j}^{\alpha}-P_{i}^{\alpha})(P_{j}^{\beta}-P_{i}^{\beta})  
\end{eqnarray}

\begin{eqnarray}
\vec P_{ij}^T \vec P_{ij}&=&\sum_{\alpha=1,2,3} (P_{j}^{\alpha}-P_{i}^{\alpha})^2 \nonumber \\
&=& \sum_{\alpha=1,2,3} (P_{j}^{\alpha})^2 +(P_{i}^{\alpha})^2- 2 P_{i}^{\alpha} P_{j}^{\alpha}
\end{eqnarray}



The single ion Hamiltonian $H_{\rm E}$ is similar to the linear chain, in general

\begin{eqnarray}
H_{\rm E}&=& \sum_{i} H_{\rm E}(i) \nonumber \\
 H_{\rm E}(i)&\equiv&\frac{a_0^2 \vec p_i^2}{2 m_i} -\frac{1}{2} 
\sum_{\alpha} K_{\alpha\beta}(ii) P_{i}^{\alpha}P_{i}^{\beta}a_0^{-2}   \\
K_{\alpha\beta}(ii)&\equiv& -a_0^2\sum_{j} \frac{R_{ij}^{\alpha}R_{ij}^{\beta}}{|\vec R_{ij}|^2} 
[c_L(ij)-c_T(ij)] + \nonumber \\
& & +  \delta_{\alpha\beta} c_T(ij)
\end{eqnarray}

and the interaction Hamiltonian  $H_{\rm int}$

\begin{eqnarray}
H_{\rm int}&=&-\frac{1}{2}\sum_{i\ne j,\alpha\beta} K_{\alpha\beta}(ij) P_{i}^{\alpha}P_{j}^{\beta}a_0^{-2}  \\
K_{\alpha\beta}(ij)&\equiv&a_0^2\frac{R_{ij}^{\alpha}R_{ij}^{\beta}}{|\vec R_{ij}|^2}[c_L(ij)-c_T(ij)]
+a_0^2\delta_{\alpha\beta}c_T(ij) \nonumber
\end{eqnarray}


\subsection{The total Crystal field Phonon Interaction Hamiltonian}

Also the crystal field may be written and expanded in terms of the strain $\epsilon$ and the 
$\vec P_i$ leading to the crystal field phonon interaction. Minimizing the Energy with respect to
the strain tensor $\epsilon$ leads to expressions for the strain $\epsilon$ in terms of expectation values
of Stevens Operators and displacement operators. 

The Hamiltonian can be written as a sum  (writing $\gamma$ instead of $lm$ for the crystal field parameter
indices, the $i$ index counts nuclei carrying with them
the charge producing the crystal field $i=1,...,N_{\rm nuclei}$, 
the $j$ index runs over magnetic ions in a crystal $j=N_{\rm nuclei}+1,... N_{\rm nuclei}+N_{\rm magnetic ions}$)

\begin{eqnarray}\label{Hamiltonian}
H_{\rm tot} & = & H_{\rm ph} +\sum_{j,\gamma} B_{\gamma}(j,\vec U_1,...,\vec U_N) O_{\gamma}(\vec J_i) \\
&=& H_{\rm ph} +\sum_{j,\gamma} B_{\gamma}(j,0,\dots,0) O_{\gamma}(\vec J_j) + H_{\rm cfph}\nonumber 
 \end{eqnarray}

\begin{eqnarray}\label{Hcfph}
H_{\rm cfph}&=&\sum_{i<j,\gamma}\nabla_{\vec U_i} B_{\gamma}(j) \vec U_i O_{\gamma}(\vec J_j) \nonumber \\
&=& \sum_{i<j,\gamma}\nabla_{\vec U_i} B_{\gamma}(j) (\bar a \vec R_i + \vec P_i)O_{\gamma}(\vec J_j) \nonumber \\
&=& \sum_{i<j,\alpha\beta=1,2,3,\gamma=1,...} R_i^{\beta}  \frac{\partial B_{\gamma}(j)}{\partial U_i^{\alpha}}  
  \epsilon_{\alpha\beta} O_{\gamma}(\vec J_j) \nonumber \\
 &&+\sum_{i<j,\gamma}\nabla_{\vec U_j} B_{\gamma}(j) \vec P_i O_{\gamma}(\vec J_j) \nonumber \\
&=& -\sum_{j,\alpha=1,..,6,\gamma=1,...} G_{\rm cfph}^{\alpha\gamma}(j) \epsilon_{\alpha} O_{\gamma}(\vec J_j) \nonumber \\
 &&-\sum_{i<j,\alpha=1,2,3,\gamma=1,...} \Gamma^{\alpha\gamma}(ij) P_i^{\alpha}a_0^{-1} O_{\gamma}(\vec J_j)
 \end{eqnarray}

The definition of the static magnetoelastic constants writing explicitely the Voigt notation of the first index $\alpha=(\alpha\beta)$
 is

\begin{eqnarray}\label{Gcfph}
 G_{\rm cfph}^{(\alpha\beta)\gamma}(j) &=&
-\frac{1}{2}\sum_{i}( R_i^{\beta}  \frac{\partial B_{\gamma}(j)}{\partial U_i^{\alpha}} 
+ R_i^{\alpha} \frac{\partial B_{\gamma}(j)}{\partial U_i^{\beta}} ) 
 \end{eqnarray}

The dynamic magnetoelastic constants (the crystal field phonon coupling constants) are

\begin{eqnarray}\label{Gammacfph}
\Gamma^{\alpha\gamma}(ij)&=& -a_0\frac{\partial B_{\gamma}(j)}{\partial U_i^{\alpha}}
 \end{eqnarray}

Note, that equation (\ref{Hcfph}) makes use of the displacement derivatives of the 
crystal field parameters and not the strain derivatives found in literature
 \cite{palmer78-2465,jensen75-320,mcewen91-3298}.
In the last line of this equation we have made use of the invariance of the
total crystal field energy under rotations
and therefore substituted $\bar a$ with the strain $\bar \epsilon$.

Summarizing, and remembering the dimensionless phonon displacement operators
$\mbf u = \mbf P / a_0$, we can write the total Hamiltonian as

\begin{eqnarray}\label{Htot}
H_{\rm tot} & = & \sum_{i,\gamma} B_{\gamma}(i,0,\dots,0) O_{\gamma}(\vec J_i) + \sum_{i} H_{\rm E}(i) + \\
&+&\frac{1}{2}\sum_{\alpha\gamma=1-6} c^{\alpha\gamma} \epsilon_{\alpha}\epsilon_{\gamma} - \nonumber \\
&-&\frac{1}{2}\sum_{i\ne j,\alpha\beta} K_{\alpha\beta}(ij) u_{i}^{\alpha}u_{j}^{\beta} 
-\sum_{i<j,\alpha=1-6,\gamma}\Gamma^{\alpha\gamma}(ij) u^{\alpha}_i O_{\gamma}(\vec J_j)
\nonumber \\
&-&\sum_{i,\alpha=1-6,\gamma=1,2,3} G_{mix}^{\alpha\gamma}(i) \epsilon_{\alpha} u_{i}^{\gamma} - \nonumber \\
&-&\sum_{i,\alpha=1-6,\gamma=1,...} G_{\rm cfph}^{\alpha\gamma}(i) \epsilon_{\alpha} O_{\gamma}(\vec J_i) \nonumber
 \end{eqnarray}



The first line in (\ref{Htot}) contains the single ion Hamiltonian (crystal field, phonon), 
and the second line the elastic energy (also a ''single ion'' term),
the third line the interaction terms (phonon, crystal field phonon), 
the forth line the mixing term and 
the last line  the magnetoelastic term. 
In a selfconsistent solution it is possible to 
determine (i) $\epsilon$, (ii) $\langle O_l^m(i) \rangle $ and (iii)
$\langle \vec u_i \rangle$. This will produce multipolar phase diagrams including the
magnetostrictive properties without the need of a detailed investigation of the symmetry adapted Hamiltonian.

Setting zero the derivative of the expectation value of the Hamiltonian (\ref{Hamiltonian}) with respect to
 $\bar a$  (i.e. minimizing the energy with respect to strain and  rotation) 
yields the following relations 

\begin{eqnarray}\label{zeroa}
0&=& \sum_{ij} \frac{c_L(ij)-c_T(ij)}{|\vec R_{ij}|^2} 
(\vec R_{ij}^T\bar \epsilon \vec R_{ij})R_{ij}^{\alpha}R_{ij}^{\beta}   \\
      && + \frac{c_T(ij)}{2} (R_{ij}^{\alpha} (\bar \epsilon \vec R_{ij})^{\beta}+R_{ij}^{\beta} (\bar \epsilon R_{ij})^{\alpha}) \nonumber \\
&& +2\sum_{ij} \frac{c_L(ij)-c_T(ij)}{|\vec R_{ij}|^2} 
R_{ij}^{\alpha}R_{ij}^{\beta} \vec R_{ij}^T \langle \vec P_{i} \rangle  \nonumber \\
      && +2 c_T(ij) \vec  R_{ij}^{\beta} \langle P_{i}^{\alpha} \rangle
+ \sum_{ij,\gamma}\frac{\partial B_{\gamma}(i)}{\partial U_j^{\alpha}}   R_j^{\beta}  \langle O_{\gamma}(\vec J_i) \rangle \nonumber
 \end{eqnarray}

The index $i$ needs only to go over the atoms in the unit cell, because the crystal structure
is periodic. There are 9 equations for $\alpha,\beta=1,2,3$ for nine components
$a_{\alpha\beta}$. Thus the coefficients of the strain  in  equation (\ref{zeroa})
can be evaluated numerically. For each mean field iteration the strain $\epsilon_{\alpha\beta}$ components
can be calculated from equation (\ref{zeroa}) and inserted into (\ref{Hamiltonian}) until
selfconsistency is achieved.

Considering only the strain $\bar \epsilon$ we can make use of the elastic constants
in forming the derivative of the Hamiltonian, 
we get 6 equations for $\alpha=1,...,6$, from which the 6 strain components can be determined.


\begin{eqnarray}\label{zeroeps}
\sum_{\beta=1,..6}  c^{\alpha\beta} \epsilon_{\beta} &=&
 \sum_{i,\delta=1,2,3}  G_{mix}^{\alpha\delta}(i)\langle u_{i}^{\delta} \rangle  \\
 &+& \sum_{i,\gamma=1,...} G_{\rm cfph}^{\alpha\gamma}(i) \langle O_{\gamma}(\vec J_i) \rangle \nonumber
\end{eqnarray}




\subsection{Exchange Striction}

If the two ion interactions are strain dependent, an additional contribution to the
strain tensor has to be considered in the equations (\ref{Htot}) to (\ref{zeroeps}).
The resulting contribution is commonly called {\em exchange striction}, because
predominantly the magnetic exchange interactions will contribute.

We therefore extend the model by the exchange striction writing the 
Hamiltonian

\begin{equation}\label{Hwithexstric}
H=H_{\rm tot} -\frac{1}{2} \sum_{n,n',\alpha,\beta}
 {\mathcal J}_{\alpha\beta}((1+\bar a)(\vec R_{j} - \vec R_i)) \hat \mathcal I_{\alpha}^i \hat \mathcal I_{\beta}^{j}
\end{equation}

and expand the two ion interaction ${\mathcal J}_{\alpha\beta}(\mbf R_{ij}) $
in a Taylor expansion for small strain:

\begin{eqnarray}\label{jjtaylor}
{\mathcal J}_{\alpha\beta}(\vec R=(1+\bar a)\vec R_{ij})&=&{\mathcal J}_{\alpha\beta}(\vec R_{ij})
+ \frac{\partial {\mathcal J}_{\alpha\beta}}{\partial R^{\alpha}} \frac{\partial (\bar a\vec R_{ij})^{\alpha}}{\partial \epsilon_{\beta}} \epsilon_{\beta} + \dots \nonumber \\
&=& {\mathcal J}_{\alpha\beta}(\vec R_{ij})+ 
\sum_{\alpha\gamma=1,2,3,\beta=1,..,6}
\frac{\partial {\mathcal J}_{\alpha\beta}}{\partial R^{\alpha}}
\frac{\partial \epsilon_{\alpha\gamma}\vec R_{ij}^{\gamma}}{\partial \epsilon_{\beta}} \epsilon_{\beta}
\end{eqnarray}

and equation (\ref{zeroeps}) is extended 

\begin{eqnarray}\label{zeroepsex}
\sum_{\beta=1,..,6}  c^{\alpha\beta} \epsilon_{\beta} &=&
 \sum_{i,\delta=1,2,3}  G_{mix}^{\alpha\delta}(i)\langle u_{i}^{\delta} \rangle  \\
 &+& \sum_{i,\gamma=1,...} G_{\rm cfph}^{\alpha\gamma}(i) \langle O_{\gamma}(\vec J_i) \rangle \nonumber \\
 &+& \frac{1}{2}\sum_{ii',\delta,\delta'\alpha',\gamma=1,..,3}
\frac{\partial {\mathcal J}_{\delta\delta'}(\vec R_{ii'})}{\partial R^{\alpha'}}
\frac{\partial \epsilon_{\alpha'\gamma} R_{ii'}^{\gamma}}{\partial \epsilon_{\alpha}}
\langle \hat \mathcal I_{\delta}^i \hat \mathcal I_{\delta'}^{i'}\rangle \nonumber
\end{eqnarray}


\subsection{Progam Workflow}

Technically the Hamiltonian (\ref{Htot}) is of the general form for 
McPhase~\cite{rotter12-213201}, with single
ion and interaction terms. The corresponding  mean field procedure can be done by an internal single ion module 
''epsilon'', which represents not an ion, but the
strain. Given $\langle u_{j}^{\alpha} \rangle$ and $\langle O_{\gamma}(\vec J_i) \rangle$ and
appropriate interaction constants to the ''epsilon'' from (\ref{Hmix}) and (\ref{Hcfph}), respectively,
the right side in (\ref{zeroeps}) can be evaluated in each mean field loop 
in module ''epsilon'' function {\em Icalc}. Then elastic constants in (\ref{zeroeps}) can be used to calculate
a new strain $\bar \epsilon$, a six component vector. Via the aforementioned appropriate
interaction constants (from (\ref{Gmix}) and (\ref{Gcfph})) the strain will produce mean
fields on lattice displacements and magnetic charge density (crystal field) and so on ...
The free energy returned by the module ''epsilon''  correspond to the elastic energy
per unit cell given by equation (\ref{Eelconstvoigt}). 

For the dynamics the ''epsilon'' module does not
yield any single ion excitations. The excitations can be calculated without taking into account the
term $H_{mix}$, because this is linear in the displacement operators and in the harmonic
approximation the spectrum of the harmonic Einstein oscillator will not change with such an
internal force. The strain has only to be taken into account as a linear modification of the crystal field
parameters in the first term of equation (\ref{Hcfph}). This is done automatically by creating the
file {\em mcdisp.mf} with the ''epsilon'' module.

Elastic constants and mixing term parameters $G_{mix}$ should be created with the
option {\em -bvk} of the program {\em makenn} and stored in the 
file {\em mcphas.j}. Using {\em makenn} with the option {\em -cfph} will create
the magnetoelastic parameters $G_{\rm cfph}$ and $\Gamma_{\rm cfph}$.

\subsubsection{Magnetoelastic Options to {\prg mcphas}}

\begin{itemize}
\item[{\prg -doeps}]
If the program {\em mcphas} is started with option {\prg -doeps}
and it finds elastic constants in the input file {\em mcphas.j} (note, that the elastic constants
in the input file are normalised to the primitive crystallographic unit cell, units are 
meV / primitive crystallographic unit cell), it
 will use these and determine selfconsistently the strain $\epsilon$ by solving equations
(\ref{zeroeps}) and (\ref{Htot}). Elastic energy and strain tensor are stored in {\em results/mcphas.fum}.
 In this way it is be possible to model Jahn Teller transitions,
phase diagrams, magnetostriction, thermal expansion (magnetic part) and dynamics consistently based
only on point charges and Born von Karman springs. 
If {\prg mcphas} is used with option {\prg -doeps} and it finds files {\prg mcphas.djdx}, {\prg mcphas.djdy}, {\prg mcphas.djdz} with
derivatives of the two ion interaction parameters with respect to
spatial coordinates $x,y,z$ respectively, it performs a calculation of
the corresponding correlation functions and computes the strain  using
equation (\ref{zeroepsex}) at each iteration. Subsequently, making use of the Taylor expansion
of the interaction constants (\ref{jjtaylor}) the computed strain is used 
in the next iteration for the the mean field Hamiltonian 
(mean field theory is used by {\em McPhase}
to solve the Hamiltonian (\ref{Hwithexstric}) and (\ref{Htot})). 
At the end of the iteration loop mean fields, moments $\langle I \rangle$ and 
a selfconsistent strain tensor $\epsilon$ is obtained.
 
\item[{\prg -linepscf}]
The option {\prg -linepscf} together with {\prg -doeps} will trigger 
a calculation where equation (\ref{zeroeps}) (and in case
of exchange striction terms (\ref{zeroepsex}))
is used to calculate the strain $\epsilon$, however the last two terms in
(\ref{Htot}) are always evaluated
for zero strain, i.e. the crystal field striction is not calculated
selfconsistently, but only in the linear approximation assuming that the 
strain leads only to a negligible perturbation of the crystal field
Hamiltonian. 

\item[{\prg -linepsjj}]
The option {\prg -linepsjj} together with {\prg -doeps} will trigger 
a calculation where equation (\ref{zeroepsex})
is used to calculate the strain $\epsilon$, yet the modification of
the two ion interaction by the strain is assumed to be small
and two ion interaction parameters (\ref{jjtaylor}) are used
in the mean field loop always for zero strain, i.e. the exchange striction is not calculated
selfconsistently, but only in the linear approximation assuming that the 
strain leads only to a negligible perturbation of the two ion interaction
Hamiltonian.


\end{itemize}


The length change $\Delta L/L$ of a sample in a dilatometer experiment
can be calculated from the strain tensor components
 using~\footnote{S. Bluegel, Juelich, private communication}

\begin{equation}
\frac{\Delta L}{L}=\sum_{\alpha\beta} \epsilon_{\alpha\beta} \hat L_{\alpha} \hat L_{\beta}
\end{equation}

where $\hat \mbf  L$ denotes the unit vector in the direction of
measurement.


Utopia: in a further step stress tensor components could be envisaged 
to act similar as an external magnetic field in {\em McPhase}, these will only act on the
''espilon'' module and on no other module. The equations given above have to be adapted accordingly 
and then it should be possible to calculate in addition to magnetic phase diagrams also stress dependence
of Jahn Teller transitions and excitations. 

