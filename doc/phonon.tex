\section{Including Phonons and Crystal-Field Phonon interactions}\label{phonons}

In this section we discuss, how lattice dynamics may be considered in the framework of
the Hamiltonian (\ref{fullhamiltonian}). We will see that this corresponds 
to a system of coupled Einstein oscillators. One such oscillator can be modelled 
by setting up a {\prg sipf} file with the module {\prg phonon}. Coupling has
to be done in {\prg mcdisp.j}. Rephrasing lattice dynamics in this way allows 
to couple phonons to the crystal field.

A three dimensional Einstein oscillator (for atom $n$) in a solid can be described by 
the following Hamiltonian

\begin{equation}\label{einstein}
H_E(n)=\frac{a_0^2{\mbf p_n}^2}{2m_n} + \frac{1}{2} {\mbf u}^T_n \m{K}(nn) {\mbf u}_n
\end{equation}

Here $\mbf u$ is the dimensionless displacement vector ($\mbf u_n={\mbf P}_n/a_0=\Delta {\mbf r}_n/a_0$, 
with the Bohr radius $a_0=0.5219$~\AA), $m_n$ the
mass of the atom $n$, $\mbf p_n=d\mbf u_n/dt$ the conjugate momentum to $\mbf u_n$ and
$\m{K}(nn)$ the Matrix describing the restoring force.

Coupling such oscillators leads to the Hamiltonian

\begin{equation}
H_{phon}=\sum_n H_E(n) -\frac{1}{2} \sum_{n\neq n'} {\mbf u}_n^T \m{K}(nn')  {\mbf u}_{n'}
\end{equation}

In a mean field type of theory 
the phonon single ion module has thus to solve the Hamiltonian

\begin{equation}\label{phonsiham}
H_E=\frac{a_0^2{\mbf p}^2}{2m} + \frac{1}{2} {\mbf u}^T \m{K} {\mbf u} - {\mbf F}^T {\mbf u}
\end{equation}

Here the force $\mbf F$ corresponds to the exchange field $\mbf H_{xc}$ and $\mbf u$ to
 the general operator $\mbf I$ and $\m{K}(nn')$ to $\mathcal J(nn')$ of equation (\ref{fullhamiltonian}),
respectively. The single ion Hamiltonian (\ref{phonsiham}) can be solved by transforming
it to normal coordinates (main axis of the Einstein oscillator) using the transformation
matrix $\m{S}$, which diagonalises $\m{K}=\m{S}^T\m{\Omega}\m{S}$:

\begin{equation}
 \mbf u'=\m{S} \mbf u -\m{\Omega}^{-1} \m{S} \mbf F
\end{equation}

\begin{equation}\label{phonsihamdiag}
H_E=\frac{a_0^2{\mbf p'}^2}{2m}+\frac{1}{2} {\mbf u'}^T \m{\Omega} {\mbf u'} 
-\frac{1}{2}\mbf F^T \m{S}^T \m{\Omega}^{-1} \m{S} \mbf F
\end{equation}

Due to the action of the force $\mbf F$ the equilibrium position of the oscillator
is $\mbf u_0=\m{S}^T\m{\Omega}^{-1}\m{S}\mbf F$ (it is the task of the function
{\prg Icalc} to return this equilibrium position), the energies correspond to the three elements
of the diagonal matrix $\m{\Omega}$, i.e. $\Omega_{11}=m a_0^2 (\Delta_1 /\hbar)^2$,
$\Omega_{22}=m a_0^2 (\Delta_2 /\hbar)^2$,
$\Omega_{33}=m a_0^2 (\Delta_3 /\hbar)^2$. In order to run {\prg mcdisp} we
have to calculate the transition matrix elements:

The single ion susceptibility for such a transition, e.g. $\Delta_1$ - corresponds to

\begin{eqnarray}
\m{\chi}^0&=&\sum_{\nu\mu}\frac{\langle \nu|\mbf u|\mu\rangle\langle \mu |\mbf u^T|\nu \rangle}{\Delta_1 -\hbar \omega}
(p_{\nu} -p_{\mu}) \\
&=&\sum_{\nu\mu}\frac{\langle \nu|\m{S}^T \mbf u'|\mu\rangle\langle \mu |\mbf u'^T \m{S}|\nu \rangle}{\Delta_1 -\hbar \omega}
(p_{\nu} -p_{\mu}) 
\end{eqnarray}
Because the different components of $\mbf u'$ commute and the Hamiltonian (\ref{phonsihamdiag})
is separable, for the transition $\Delta_1$ only the terms with $u_1'$ in the nominator
contribute:

\begin{eqnarray}
\chi^0_{\alpha\beta}&=&S^T_{\alpha1}\sum_{\nu\mu}\frac{\langle \nu|u_1'|\mu\rangle\langle \mu | u_1'|\nu \rangle}{\Delta_1 -\hbar \omega}
(p_{\nu} -p_{\mu}) S_{1\beta}\\
&=& S^T_{\alpha1}S_{1\beta}\frac{\hbar^2}{2ma_0^2\Delta_1}\left(\frac{1}{\Delta_1-\hbar\omega}+\frac{1}{\Delta_1+\hbar\omega}\right )
\end{eqnarray}

In order to derive the last result we had to express $u_1'$ in terms of ladder  operators
$u_1'=a_0^{-1} \hbar/\sqrt{2m\Delta_1}(a+a^{\dagger})$ and  apply $a^{\dagger}|\nu\rangle=\sqrt{\nu+1}|\nu+1\rangle$,
$a|\nu\rangle=\sqrt{\nu}|\nu-1\rangle$ and $\sum_{\nu=0}^{\infty}(p_{\nu}-p_{\nu+1})(\nu+1)=1$,
$p_{\nu}=exp(-\nu\Delta_1/kT)(1-exp(-\Delta_1/kT))$. This shows that the single ion susceptibility
of our atom can be written as a sum of three effective transitions (with temperature independent
susceptibility)

\begin{eqnarray}
\chi^0_{\alpha\beta}
&=& \sum_{i=1,2,3} S^T_{\alpha i}S_{i\beta}\frac{\hbar^2}{2ma_0^2\Delta_i}
\left(\frac{1}{\Delta_i-\hbar\omega}+\frac{1}{\Delta_i+\hbar\omega}\right )
\end{eqnarray}

Thus the module {\prg phonon} has to provide in it's function {\prg du1calc} these three
transitions (=number of transitions).

\subsection{Using Single Ion Module {\prg phonon}}

The module {\prg phonon} allows to consider the phononic degrees of freedom in McPhas.
The single ion input file for an oscillating atom has to have the following format:

\begin{verbatim}
#!MODULE=phonon
#<!--mcphase.sipf-->
#
# phonon
# MODPAR1=mass of atom in units of m0 (atomic mass unit=1.660539e-27 kg)
#
#-----------
MODPAR1=57  # mass in(m0)
MODPAR2=7   # Kxx
MODPAR3=7   # Kyy
MODPAR4=8   # Kzz
MODPAR5=0   # Kxy  in (meV)
MODPAR6=0   # Kxz
MODPAR7=0   # Kyz

#-------------------------------------------------------
# Neutron Scattering Length (10^-12 cm) (can be complex)
#-------------------------------------------------------
SCATTERINGLENGTHREAL=0.769
SCATTERINGLENGTHIMAG=0
#  ... note: - if an occupancy other than 1.0 is needed, just reduce 
#              the scattering length linear accordingly

SCATTERINGLENGTHREAL=0.945
SCATTERINGLENGTHIMAG=0
\end{verbatim}

Thus the single ion property file contains the matrix $\m{K}(nn)$, the matrix $\m{K}(nn')$ decribing the
forces between
different ions $n$ and $n'$ have to be given in the file {\prg mcphas.j}, which could
look like:

\begin{verbatim}
# 
#<!--mcphase.mcphas.j-->
*************************************************************
# Lattice Constants (A)
#! a=4.047 b=4.047 c=9.612 alpha=  90 beta=  90 gamma=  90
#! r1a=   1 r2a=   0 r3a= 0.5
#! r1b=   0 r2b=   1 r3b= 0.5   primitive lattice vectors [a][b][c]
#! r1c=   0 r2c=   0 r3c= 0.5
#! nofatoms=1  nofcomponents=3  number of atoms in primitive unit cell/number of components of each spin
#*********************************************************************
#! da=   0 [a] db=   0 [b] dc=   0 [c] nofneighbours=2 diagonalexchange=1 sipffilename=phonon.sipf
#da[a]   db[b]     dc[c]       Jaa[meV]  Jbb[meV]  Jcc[meV]  Jab[meV]  Jba[meV]  Jac[meV]  Jca[meV]  Jbc[meV]  Jcb[meV]
0 0 1 1.1 1.1 1.1
0 0 -1 1.1 1.1 1.1 
#*********************************************************************
\end{verbatim}

It is planned that these files should be created automatically from the 
output of DFT programs. [to be done]

If crystal field - phonon coupling is to be input, magnetic ions may be added to {\prg mcphas.j}.
They should not be placed at exactly the same position as the phonon atoms - i.e. da db dc should be chosen
(slightly) different to enable the mcphas.j loader to identify which  is which.

The coupling between phononic and crystal field degrees of freedom should be derived from
$H_{cf-phon}=(\mbf \nabla_{\mbf u(n')} B_l^m(n))   \mbf u(n')  O_l^m(\mbf J_n)$. This could be done
using the program {\prg pointc} and {\prg makenn} applying small differential changes to
the atomic positions in the unit cell in order to evaluate the gradient of the crystal field
parameters. [efficient program to do this task is to be written]
