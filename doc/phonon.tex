\section{Including Phonons and Crystal-Field Phonon interactions}\label{phonons}

In this section we discuss, how lattice dynamics may be considered in the framework of
the Hamiltonian (\ref{fullhamiltonian}). We will see that this corresponds 
to a system of coupled Einstein oscillators. One such oscillator can be modelled 
by setting up a {\prg sipf} file with the module {\prg phonon}. Coupling has
to be done in {\prg mcdisp.j}. Rephrasing lattice dynamics in this way allows 
to couple phonons to the crystal field.

A three dimensional Einstein oscillator (for atom $i$) in a solid can be described by 
the following Hamiltonian

{\color{blue}
\begin{equation}\label{einstein}
\hat H_E(i)=\frac{a_0^2{\hat \mbf w_i}^2}{2m_i} - \frac{1}{2} {\hat \mbf u}^T_i \m{K}(ii) {\hat \mbf u}_i
\end{equation}
}

Here $\hat \mbf u$ is the dimensionless displacement vector 
($\hat \mbf u_i={\hat \mbf P}_i/a_0=\Delta {\hat \mbf r}_i/a_0$, 
with the Bohr radius $a_0=0.5219$~\AA), $m_i$ the
mass of the atom $i$, $\hat \mbf w_i=d\hat \mbf u_i/dt= \mbf p_i/a_0$ the conjugate momentum to $\hat \mbf u_i$ and
$\m{K}(ii)$ the Matrix describing the restoring force.

Coupling such oscillators leads to the Hamiltonian

\begin{equation}
\hat H_{phon}=\sum_i \hat H_E(i) -\frac{1}{2} \sum_{i\neq i'} {\hat \mbf u}_i^T \m{K}(ii')  {\hat \mbf u}_{i'}
\end{equation}

Note that our coupling constants $K_{\alpha\beta}(ii')=-A_{\alpha\beta}(ii')$, where
 $A_{\alpha\beta}$ are the second-order derivatives of the potential energy
as defined e.g. in \cite[page 99]{lovesey84-1}.

In a mean field type of theory 
the phonon single ion module has thus to solve the Hamiltonian

\begin{equation}\label{phonsiham}
\hat H_E=\frac{a_0^2{\hat \mbf w}^2}{2m} - \frac{1}{2} {\hat \mbf u}^T \m{K} {\hat \mbf u} - {\mbf F}^T {\hat \mbf u}
\end{equation}

Here the force $\mbf F$ corresponds to the exchange field $\mbf H_{xc}$ and $\hat \mbf u$ to
 the general operator $\hat \mbf I$ and $\m{K}(nn')$ to $\mathcal J(nn')$ of equation (\ref{fullhamiltonian}),
respectively. The single ion Hamiltonian (\ref{phonsiham}) can be solved by transforming
it to normal coordinates (main axis of the Einstein oscillator) using the transformation
matrix $\m{S}$, which diagonalises $\m{K}=\m{S}^T\m{\Omega}\m{S}$:

\begin{equation}
 \hat \mbf u'=\m{S} \hat \mbf u +\m{\Omega}^{-1} \m{S} \mbf F
\end{equation}

\begin{equation}\label{phonsihamdiag}
\hat H_E=\frac{a_0^2{\hat \mbf w}^{'2}}{2m}-\frac{1}{2} {\hat \mbf u}^{'T} \m{\Omega} {\hat \mbf u'} 
-\frac{1}{2}\mbf F^T  \mbf u_0
\end{equation}

Due to the action of the force $\mbf F$ the equilibrium position of the oscillator
is $\mbf u_0=-\m{S}^T\m{\Omega}^{-1}\m{S}\mbf F$ (it is the task of the function
{\prg Icalc} to return this equilibrium position), the energies correspond to the three elements
of the diagonal matrix $\m{\Omega}$, i.e. $\Omega_{11}=-m a_0^2 (\Delta_1 /\hbar)^2$,
$\Omega_{22}=-m a_0^2 (\Delta_2 /\hbar)^2$,
$\Omega_{33}=-m a_0^2 (\Delta_3 /\hbar)^2$. In order to run {\prg mcdisp} we
have to calculate the transition matrix elements:

The single ion susceptibility for such a transition, e.g. $\Delta_1$ - corresponds to

\begin{eqnarray}
\m{\chi}^0&=&\sum_{\nu\mu}\frac{\langle \nu|\hat \mbf u|\mu\rangle\langle \mu |\hat \mbf u^T|\nu \rangle}{\Delta_1 -\hbar \omega}
(p_{\nu} -p_{\mu}) \\
&=&\sum_{\nu\mu}\frac{\langle \nu|\m{S}^T \hat \mbf u'|\mu\rangle\langle \mu |\hat \mbf u'^T \m{S}|\nu \rangle}{\Delta_1 -\hbar \omega}
(p_{\nu} -p_{\mu}) 
\end{eqnarray}
Because the different components of $\mbf u'$ commute and the Hamiltonian (\ref{phonsihamdiag})
is separable, for the transition $\Delta_1$ only the terms with $u_1'$ in the nominator
contribute:

\begin{eqnarray}
\chi^0_{\alpha\beta}&=&S^T_{\alpha1}\sum_{\nu\mu}\frac{\langle \nu|\hat u_1'|\mu\rangle\langle \mu |\hat  u_1'|\nu \rangle}{\Delta_1 -\hbar \omega}
(p_{\nu} -p_{\mu}) S_{1\beta}\\
&=& S^T_{\alpha1}S_{1\beta}\frac{\hbar^2}{2ma_0^2\Delta_1}\left(\frac{1}{\Delta_1-\hbar\omega}+\frac{1}{\Delta_1+\hbar\omega}\right )
\end{eqnarray}

In order to derive the last result we had to express $\hat u_1'$ in terms of ladder  operators
$\hat u_1'=a_0^{-1} \hbar/\sqrt{2m\Delta_1}(\hat a+\hat a^{\dagger})$ and  apply $\hat a^{\dagger}|\nu\rangle=\sqrt{\nu+1}|\nu+1\rangle$,
$\hat a|\nu\rangle=\sqrt{\nu}|\nu-1\rangle$ and $\sum_{\nu=0}^{\infty}(p_{\nu}-p_{\nu+1})(\nu+1)=1$,
$p_{\nu}=exp(-\nu\Delta_1/kT)(1-exp(-\Delta_1/kT))$. This shows that the single ion susceptibility
of our atom can be written as a sum of three effective transitions (with temperature independent
susceptibility)

\begin{eqnarray}
\chi^0_{\alpha\beta}
&=& \sum_{n=1,2,3} S^T_{\alpha n}S_{n\beta}\frac{\hbar^2}{2ma_0^2\Delta_n}
\left(\frac{1}{\Delta_n-\hbar\omega}+\frac{1}{\Delta_n+\hbar\omega}\right )
\end{eqnarray}

Thus the module {\prg phonon} has to provide in it's function {\prg du1calc} these three
transitions (=number of transitions).

\subsection{Using Single Ion Module {\prg phonon}}

The module {\prg phonon} allows to consider the phononic degrees of freedom in {\prg McPhase}.
The single ion input file for an oscillating atom (with amplitude 
up to a maximum displacement $u_{max}$, which is relevant for the function {\prg Icalc} of the
module) has to have the following format:

\begin{verbatim}
#!MODULE=phonon
#<!--mcphase.sipf-->
#
# phonon
# MODPAR1=mass of atom in units of m0 (atomic mass unit=1.660539e-27 kg)
#
#-----------
MODPAR1=57  # mass in(m0)
MODPAR2=-1.2   # Kxx
MODPAR3=-1.2   # Kyy
MODPAR4=-1.2   # Kzz
MODPAR5=0   # Kxy  in (meV)
MODPAR6=0   # Kxz
MODPAR7=0   # Kyz
MODPAR8=1   # umax          maximum (cutoff) for displacement [a0=0.5219 A]
MODPAR9=4   # 0       umax restriction in all directions
            #                1,2,3   umax restriction in x y z direction only
            #                4       umax restriction in x and y direction
            #                5       umax restriction in x and z direction
            #                6       umax restriction in y and z direction

#-------------------------------------------------------
# Neutron Scattering Length (10^-12 cm) (can be complex)
#-------------------------------------------------------
SCATTERINGLENGTHREAL=0.769
SCATTERINGLENGTHIMAG=0
#  ... note: - if an occupancy other than 1.0 is needed, just reduce 
#              the scattering length linear accordingly

SCATTERINGLENGTHREAL=0.945
SCATTERINGLENGTHIMAG=0
\end{verbatim}

Note, that {\prg MODPAR8} and {\prg MODPAR9} impose a restriction to the {\prg Icalc} function
of the module {\prg phonon}. This means, if used with program {\prg mcphas} the
module will not return displacements $\langle \mbf u \rangle$ larger than umax.
Technically this is done by applying the operation $u \leftarrow u_{max} * tanh (u/u_{max})$
at the end of the function {\prg Icalc}. 

{\bf Coordinate System:} the Euclidean coordinate system of phonon displacements $\mbf u$ with
components $u_x,u_y,u_z$ is oriented with respect to the crystal axes $\mbf a,\mbf b, \mbf c$
such that $\mbf y || \mbf b$, $\mbf z || \mbf a \times \mbf b $ and $\mbf x$ perpendicular
to $\mbf y$ and $\mbf z$.

The single ion property file contains the matrix $\m{K}(ii)$, the matrix $\m{K}(ii')$ decribing the
forces between
different ions $i$ and $i'$ have to be given in the file {\prg mcphas.j}, which could
look like:

\begin{verbatim}
# 
#<!--mcphase.mcphas.j-->
*************************************************************
# Lattice Constants (A)
#! a=4.047 b=4.047 c=9.612 alpha=  90 beta=  90 gamma=  90
#! r1a=   1 r2a=   0 r3a= 0.5
#! r1b=   0 r2b=   1 r3b= 0.5   primitive lattice vectors [a][b][c]
#! r1c=   0 r2c=   0 r3c= 0.5
# Nonzero Elastic constants in meV per primitive crystal unit cell 
# in Voigt notation only first index<=second index has to be given
# because the constants are symmetric Celij=Celji
# Elastic constants refer to the Euclidean coordinate system ijk defined
# with respect to abc as j||b, k||(a x b) and i normal to k and j
#!  Cel11=+129702.248 Cel12=+63972.0533 Cel13=+34346.9127
#!  Cel22=+129702.248 Cel23=+34346.9127
#!  Cel33=+60842.4597
#!  Cel44=+137509.379
#!  Cel55=+137509.379
#!  Cel66=+256056.495
#! nofatoms=1  nofcomponents=3  number of atoms in primitive unit cell/number of components of each spin
#*********************************************************************
#! da=   0 [a] db=   0 [b] dc=   0 [c] nofneighbours=2 diagonalexchange=1 sipffilename=phonon.sipf
# it follows the Born von Karman model according to  springs read from file 
# the mixing terms Gmix in meV/a0  with a0=0.5292e-10 m
#! Gindices= 1,1 1,2 1,3 2,1 2,2 2,3 3,1 3,2 3,3 4,1 4,2 4,3 5,1 5,2 5,3 6,1 6,2 6,3
#! G= +1 13 0  0 0 0  0 0 0   0 0 0   0 0 0  0 0 0
#da[a]   db[b]     dc[c]       Jaa[meV]  Jbb[meV]  Jcc[meV]  Jab[meV]  Jba[meV]  Jac[meV]  Jca[meV]  Jbc[meV]  Jcb[meV]
0 0 1 1.1 1.1 1.1
0 0 -1 1.1 1.1 1.1 
#*********************************************************************
\end{verbatim}

Note that this file may optionally contain elastic constants and mixing
term parameters, which are needed for the calculation
of magnetoelastic properties (see section \ref{JT}).
It is planned that these files should be created automatically from the 
output of DFT programs. [to be done]. Currently it is possible to create 
the input parameters from Born v Karman longitudinal and transversal springs
by the program {\prg makenn} with option {\prg -bvk}.
For an example of phonons in tungsten see {\prg examples/tungsten\_phonons }




\section{Crystal Field Phonon Interaction}\label{JT}


Here some formulas are listed. Starting point is the phonon part,
 written in more general terms. 

\begin{figure}[th]
\setlength{\unitlength}{0.14in} % selecting unit length
\centering % used for centering Figure
\begin{picture}(32,15) % picture environment with the size (dimensions)
% 32 length units wide, and 15 units high.
% Declares a diamond shape
\newsavebox{\diamondshap}
\savebox{\diamondshap}(0,0)
{
   \put(-1,0){\line(3,1){1}}
   \put(-1,0){\line(3,-1){1}}
   \put(1,0){\line(-3,1){1}}
   \put(1,0){\line(-3,-1){1}}
}
% Declares a parallelogram
\newsavebox{\parallelogramshap}
\savebox{\parallelogramshap}
{
   \put(0,0){\line(2,0){10}}
   \put(1.5,2.5){\line(2,0){10}}
   \put(0,0){\line(3,5){1.5}}
   \put(10,0){\line(3,5){1.5}}
}
\put(12,7){\vector(1,0){14}}
\put(12,7){\vector(-1,-2){2}}
\put(26,7){\vector(1,1){3}}
   \put(12,7){\line(-1,0){2}}
   \put(10,7){\line(0,-1){4}}
   \put(26,7){\line(1,0){3}}
   \put(29,7){\line(0,1){3}}

\put(27,6) {$L_2$}
\put(30,8) {$T_2$}
\put(8.5,5) {$T_1$}
\put(10.5,8) {$L_1$}

\put(19,7.5) {$\mbf R_{12}$}
\put(11.5,4.5) {$\mbf P_1$}
\put(26.3,8.5) {$\mbf P_2$}

\put(15,2){$E=\frac{c_L}{2}(L_1+L_2)^2+\frac{c_T}{2}(T_1+T_2)^2$}
%\put(10,28){\usebox{\parallelogramshape}}
\put(12,7){\usebox{\diamondshap}}
\put(26,7){\usebox{\diamondshap}}
%\put(29,6.2){\line(0,1){18}}
%\put(30,15) {$\langle\hat \mathcal I^{s'}_{\beta}\rangle$}
%\put(29,24){\vector(-1,0){3}}
%\put(6,-0.6){\framebox(20,3){\begin{tabular}{c}compute free energy and \\ compare to other solutions \end{tabular} }}
\end{picture}
\caption{Illustration of the transversal and longitudinal BvK springs} % title of the Figure
\label{figbvk} % label to refer figure in text
\end{figure}

 The index $i$ now denotes all atomic
positions in a crystal, the momentum vector is $\mbf p_i=a_0 \mbf w_i$ and  the
displacement vector is $\hat \mbf U_i$. Fig.~\ref{figbvk} shows the Born-van-Karman
model definition of longitudinal springs $c_L$ and transversal
springs $c_T$ used in the following derivation.

\begin{eqnarray}\label{Hphonon}
\hat H_{\rm phon}& =&\sum_{i} \frac{a_0^2\hat  \mbf w_i^2}{2 m_i} + 
\frac{1}{2}\sum_{ij} \frac{c_L(ij)-c_T(ij)}{2|\mbf R_{ij}|^2} 
(\hat \mbf U_j . \mbf R_{ij}-\hat  \mbf U_i . \mbf R_{ij})^2 \nonumber \\
      && + \frac{c_T(ij)}{2} (\hat \mbf U_j - \hat \mbf U_i )^2 \\
 \end{eqnarray}

Note that the sum over indices $i,j$ counts each spring twice, thus a factor of $1/2$
has been added to the  Hamiltonian before the sum.

Note that the difference in undisplaced lattice positions is denoted by $\mbf R_{ij}=\mbf R_{j}-\mbf R_{i}$.
The displacement vector $\hat \mbf U_i$  is split into 
a strain component $\bar \epsilon \mbf R_i$ a rotational component
$\bar \omega \mbf R_i$ and a dynamic component $\hat \mbf P_i$ (displacement operator)
obeying periodic boundary conditions 
$\hat \mbf U_i = \bar \epsilon \mbf R_i + \bar \omega \mbf R_i+\hat  \mbf P_i= \bar a \mbf R_i+\hat  \mbf P_i$.




\begin{eqnarray}\label{Ur2}
(\hat \mbf U_j . \mbf R_{ij}-\hat  \mbf U_i . \mbf R_{ij})^2 &=&
 (\mbf R_{ij}^T\bar a \mbf R_{ij}+ \mbf R_{ij}^T \mbf P_{ij})^2 \nonumber \\
&=& (\mbf R_{ij}^T\bar a \mbf R_{ij})^2 + 2 \mbf R_{ij}^T\bar a \mbf R_{ij}\mbf R_{ij}^T \hat \mbf P_{ij} \nonumber \\
&&+ (\mbf R_{ij}^T \hat \mbf P_{ij})^2
\end{eqnarray}

\begin{eqnarray}\label{U2}
(\hat \mbf U_j - \hat \mbf U_i )^2 &=& \mbf  R_{ij}^T\bar a^T\bar a \mbf R_{ij} \nonumber \\
&&2 \mbf  R_{ij}^T\bar a^T\hat \mbf P_{ij} +\hat  \mbf P_{ij}^T \hat \mbf P_{ij}
\end{eqnarray}

Equations (\ref{Ur2}) and (\ref{U2}) contain terms quadratic in $\bar a$, which will be
considered as elastic energy contributions. The terms quadratic in $\hat \mbf P$ will contribute
to the lattice dynamics. In addition there are also 
terms linear in $\bar a$ and $\hat \mbf P$. 
Thus the phonon Hamiltonian (\ref{Hphonon}) can be separated into the elastic Energy $E_{el}$, the
Einstein single ion oscillation term $H_{\rm E}$, the bilinear interaction term $H_{\rm int}$
and the mixing term $H_{mix}$ 

\begin{eqnarray}
\hat H_{\rm phon}& =&E_{el} +\hat  H_{mix}+\hat  H_{\rm E}+\hat H_{\rm int}
\end{eqnarray}

\subsection{Elastic Energy $E_{el}$}

The elastic energy $E_{el}$ is bilinear in  $a$

\begin{eqnarray}\label{Eel}
E_{el} &=& \frac{1}{2}\sum_{ij} \frac{c_L(ij)-c_T(ij)}{2|\mbf R_{ij}|^2} 
(\mbf R_{ij}^T\bar a \mbf R_{ij})^2  \\
      && + \frac{c_T(ij)}{2} \mbf  R_{ij}^T\bar a^T\bar a \mbf R_{ij} \nonumber \\
&=& \frac{1}{2} \sum_{ij,\alpha\beta\gamma\delta} \frac{c_L(ij)-c_T(ij)}{2|\mbf R_{ij}|^2} 
R_{ij}^{\alpha}R_{ij}^{\beta}R_{ij}^{\gamma}R_{ij}^{\delta}
a_{\alpha\beta}a_{\gamma\delta} \nonumber \\
&& + \frac{c_T(ij)}{2} R_{ij}^{\alpha} R_{ij}^{\delta}
a_{\alpha\beta} \delta_{\beta\gamma} a_{\gamma\delta}  \nonumber
\end{eqnarray}

We calculate it's derivative with respect to $a_{\alpha\beta}$:

\begin{eqnarray}
\frac{\partial E_{el}}{\partial a_{\alpha\beta}}&=& \frac{1}{2}\sum_{ij,\gamma\delta} \frac{c_L(ij)-c_T(ij)}{|\mbf R_{ij}|^2} 
R_{ij}^{\alpha}R_{ij}^{\beta}R_{ij}^{\gamma}R_{ij}^{\delta}
a_{\gamma\delta} \nonumber \\
&& +    \frac{c_T(ij)}{2} R_{ij}^{\alpha}R_{ij}^{\delta}
 \delta_{\beta\gamma} a_{\gamma\delta} + \nonumber \\
&& \frac{c_T(ij)}{2} R_{ij}^{\gamma} R_{ij}^{\beta}
a_{\gamma\delta} \delta_{\delta\alpha} 
\end{eqnarray}


%derivatives with respect to the strain component $\epsilon_{\alpha\beta}$ 
%with  $\alpha,\beta=1,2,3,\alpha \le \beta$ can be written as

%\begin{eqnarray}
%\frac{\partial}{\partial \epsilon_{\alpha\beta}}&=&
%\sum_{\gamma\delta=1,2,3}\frac{\partial a_{\gamma\delta}}{\partial \epsilon_{\alpha\beta}}
%\frac{\partial}{\partial a_{\gamma\delta}} \\
%&=&
%\frac{\partial}{\partial a_{\alpha\beta}} + (1-\delta_{\alpha\beta})
%\frac{\partial}{\partial a_{\beta\alpha}}  
%\end{eqnarray}

%We calculate the derivative of the elastic energy with respect to the strain:


%\begin{eqnarray}\label{Eelderiv}
%\frac{\partial E_{el}}{\partial \epsilon_{\alpha\beta}}&=&
%\frac{\partial E_{el}}{\partial a_{\alpha\beta}} + \frac{\partial E_{el}}{\partial a_{\beta\alpha}}-\delta_{\alpha\beta}
%\frac{\partial E_{el}}{\partial a_{\alpha\alpha}}  \\ 
%&=&\frac{1}{2}\sum_{ij,\gamma\delta} (2-\delta_{\alpha\beta})\frac{c_L(ij)-c_T(ij)}{|\mbf R_{ij}|^2} 
%R_{ij}^{\alpha}R_{ij}^{\beta}R_{ij}^{\gamma}R_{ij}^{\delta} 
%a_{\gamma\delta} \nonumber \\
%& & +    \frac{c_T(ij)}{2} a_{\gamma\delta} ( R_{ij}^{\alpha}R_{ij}^{\delta}
% \delta_{\beta\gamma}  +   R_{ij}^{\gamma} R_{ij}^{\beta}
%\delta_{\delta\alpha}  \nonumber \\
%&& +    R_{ij}^{\beta} R_{ij}^{\delta}
% \delta_{\alpha\gamma}  +   R_{ij}^{\gamma}R_{ij}^{\alpha}
% \delta_{\delta\beta} \nonumber \\
%& & -    R_{ij}^{\alpha}R_{ij}^{\delta}
% \delta_{\alpha\beta}\delta_{\beta\gamma}    -  R_{ij}^{\gamma} R_{ij}^{\beta}
% \delta_{\delta\alpha} \delta_{\alpha\beta} ) \nonumber
%\end{eqnarray}



An we make also use of the definition of elastic constants:

\begin{eqnarray}\label{elconst}
c^{\alpha\beta\gamma\delta} &= &
\frac{\partial^2E_{el}}{\partial a_{\alpha\beta} \partial a_{\gamma\delta}} \\
&=& \frac{1}{2}\sum_{ij}\frac{c_L(ij)-c_T(ij)}{|\mbf R_{ij}|^2} 
R_{ij}^{\alpha}R_{ij}^{\beta}R_{ij}^{\gamma}R_{ij}^{\delta}
 \nonumber \\
&& +    \frac{c_T(ij)}{2} R_{ij}^{\alpha}R_{ij}^{\delta}
 \delta_{\beta\gamma}  + \nonumber \\
&& \frac{c_T(ij)}{2} R_{ij}^{\gamma} R_{ij}^{\beta}
 \delta_{\delta\alpha} 
\end{eqnarray}
%& & +    c_T(ij)  ( R_{ij}^{\alpha}R_{ij}^{\delta}
% \delta_{\beta\gamma}  +   R_{ij}^{\gamma}R_{ij}^{\beta}
%\delta_{\delta\alpha}  \nonumber \\
%&& +    R_{ij}^{\beta}R_{ij}^{\delta}
% \delta_{\alpha\gamma}  +   R_{ij}^{\gamma}R_{ij}^{\alpha}
% \delta_{\delta\beta} \nonumber \\
%& & -    R_{ij}^{\alpha}R_{ij}^{\delta}
% \delta_{\alpha\beta}\delta_{\beta\gamma}    -  R_{ij}^{\gamma}R_{ij}^{\alpha}
% \delta_{\delta\alpha} \delta_{\alpha\beta} + \nonumber \\
%&& -    R_{ij}^{\beta}R_{ij}^{\alpha}
% \delta_{\alpha\gamma} \delta_{\gamma\delta}  -   R_{ij}^{\gamma}R_{ij}^{\alpha}
% \delta_{\delta\beta}\delta_{\gamma\delta}  \nonumber \\
%& & +    R_{ij}^{\alpha}R_{ij}^{\alpha}
% \delta_{\alpha\beta}\delta_{\beta\gamma} \delta_{\gamma\delta}   ) \nonumber



we make use of the fact that the strain $\bar \epsilon$ is a symmetric 
tensor ($\epsilon_{\alpha\beta}=\epsilon_{\beta\alpha}$) and a rotation
is antisymmetric ($\omega_{\alpha\beta}=-\omega_{\beta\alpha}$) and 
the linear transformation $\bar a$ can be written as $\bar a=\bar \epsilon + \bar \omega$. Thus
it is possible to rewrite the elastic energy in the well known fashion

\begin{eqnarray}\label{Eelconst}
E_{el} &=& \frac{1}{2}\sum_{\alpha\beta\gamma\delta} c^{\alpha\beta\gamma\delta} \epsilon_{\alpha\beta}\epsilon_{\gamma\delta}
\end{eqnarray}



Note that we have neglected the fact, that nonzero transversal springs will 
result in a dependence of the elastic energy on the rotation tensor $\bar \omega$ as
can been seen by inserting a rotation into the second part of (\ref{Eel}).
\footnote{$R^{\alpha}R^{\delta}\omega_{\sigma}
\epsilon_{\sigma\alpha\beta}\delta_{\beta\gamma}\omega_{\eta}\epsilon_{\eta\gamma\delta}=
R^{\alpha}R^{\delta}\omega_{\sigma} \omega_{\eta}
\epsilon_{\sigma\alpha\beta}\epsilon_{\eta\beta\delta}=
R^{\alpha}R^{\delta}\omega_{\sigma} \omega_{\eta}
(\delta_{\sigma\delta}\delta_{\alpha\eta}-\delta_{\sigma\eta}\delta_{\alpha\delta})=
(\mbf R . \mbf \omega)^2-R^2\omega^2 \neq 0$,
if $\mbf R$ is not parallel to the rotation axis. Thus for some rotation the elastic energy
will depend on rotation angle if transversal springs are  introduced.
}
Therefore transversal springs have to be used with caution
 in the description of a phonon spectrum.

Note that the transversal spring part in the elastic constants as defined in (\ref{elconst}) 
is not symmetric upon exchange of indices. It is convenient to use a symmetrized version
of the elastic constants, which will yield the same elastic energy, because the
strain tensor is symmetric:

{\color{blue}
\begin{eqnarray}\label{elconstsymm}
c^{\alpha\beta\gamma\delta} &= &
 \frac{1}{2}\sum_{ij}\frac{c_L(ij)-c_T(ij)}{|\mbf R_{ij}|^2} 
R_{ij}^{\alpha}R_{ij}^{\beta}R_{ij}^{\gamma}R_{ij}^{\delta}
 \nonumber \\
&& +    \frac{c_T(ij)}{4} ( R_{ij}^{\alpha}R_{ij}^{\delta}
 \delta_{\beta\gamma}  + \nonumber \\
&&  R_{ij}^{\gamma} R_{ij}^{\beta}
 \delta_{\delta\alpha}  \nonumber \\
&& +    R_{ij}^{\beta}R_{ij}^{\delta}
 \delta_{\alpha\gamma}  +   R_{ij}^{\gamma}R_{ij}^{\alpha}
 \delta_{\delta\beta} ) \nonumber \\
\end{eqnarray}
}


With the help of elastic constants we can rewrite the derivative of the elastic energy
(\ref{Eelconst}), again ignoring the rotation dependence:

\begin{eqnarray}\label{Eelderivelconst}
\frac{\partial E_{el}}{\partial \epsilon_{\alpha\beta}}&=&
\sum_{\gamma\delta=1,2,3} c^{\alpha\beta\gamma\delta} \epsilon_{\gamma\delta}
\end{eqnarray}




To easy the indexing we apply the notation of Voigt
\begin{eqnarray}\label{voigt}
\epsilon_1&=&\epsilon_{11}  \\
\epsilon_2&=&\epsilon_{22} \nonumber \\
\epsilon_3&=&\epsilon_{33}\nonumber \\
\epsilon_4&=&2\epsilon_{23}=2\epsilon_{32}\nonumber \\
\epsilon_5&=&2\epsilon_{31}=2\epsilon_{13}\nonumber \\
\epsilon_6&=&2\epsilon_{12}=2\epsilon_{21} \nonumber
\end{eqnarray}

Note the elastic constants do not contain any prefactor in Voigt notation, i.e.
\begin{eqnarray}\label{voigtel}
c^{11}&=&c^{1111} \\
c^{44}&=&c^{2323} \nonumber \\
\end{eqnarray}

There are 21 independent elastic constants, $c^{\alpha\beta}$ with $\beta=1,...,6$ and
$\alpha \le \beta$ (or less, see  http://koski.ucdavis.edu/BRILLOUIN/CRYSTALS/Elasticities.html). The other 60 elastic constants 
can be obtained from the symmetry relations $c^{\alpha\beta}=c^{\beta\alpha}$, 
$c^{\alpha\beta\gamma\delta}=c^{\beta\alpha\gamma\delta}=c^{\alpha\beta\delta\gamma}$.

The elastic energy in Voigt notation is given by

\begin{eqnarray}\label{Eelconstvoigt}
E_{el} &=& \frac{1}{2}\sum_{\alpha\gamma=1,..,6} c^{\alpha\gamma} \epsilon_{\alpha}\epsilon_{\gamma}
\end{eqnarray}

\subsection{Mixing term $H_{mix}$}

The mixing term $H_{mix}$ is 

\begin{eqnarray}\label{Hmix}
\hat H_{mix}&=&\frac{1}{2} \sum_{ij} \frac{c_L(ij)-c_T(ij)}{|\mbf R_{ij}|^2} 
\mbf R_{ij}^T\bar a \mbf R_{ij}\mbf R_{ij}^T \hat \mbf P_{ij}  \nonumber \\
      && + c_T(ij) \mbf  R_{ij}^T\bar a^T\mbf P_{ij} \nonumber \\
&=& \frac{1}{2}\sum_{ij} 2\frac{c_L(ij)-c_T(ij)}{|\mbf R_{ij}|^2} 
\mbf R_{ij}^T\bar a \mbf R_{ij}\mbf R_{ij}^T \hat \mbf P_{i}  \nonumber \\
      && + 2 c_T(ij) \mbf  R_{ij}^T\bar a^T\hat \mbf P_{i} \nonumber \\
&=& \sum_{ij,\alpha\beta} \frac{c_L(ij)-c_T(ij)}{|\mbf R_{ij}|^2} 
R_{ij}^{\alpha} a_{\alpha\beta} R_{ij}^{\beta} R_{ij}^{\gamma}\hat   P_{i}^{\gamma}  \nonumber \\
      && +  c_T(ij)   R_{ij}^{\beta} a_{\alpha\beta} \delta_{\alpha\gamma}\hat  P_{i}^{\gamma}  \\
&=&-\sum_{\stackrel{i,\alpha=1,..,6}{ \gamma=1,2,3}} G_{mix}^{\alpha\gamma}(i) \epsilon_{\alpha}\hat  P_{i}^{\gamma}/a_0 +
 \sum_{ij,\alpha\beta}  c_T(ij)   R_{ij}^{\beta} \omega_{\alpha\beta}\hat  P_{i}^{\alpha} \nonumber
\end{eqnarray}

Note that without crystal field phonon interaction the mixing term $\hat H_{mix}$ is zero, because
the expectation values $\langle \hat \mbf P_i \rangle$ are zero in this case. The last term reflects
the fact, that the energy is not invariant under rotations in case of transversal springs.

In (\ref{Hmix}) we made use of the phonon-strain interaction constants $G_{mix}^{\alpha\gamma}(i)$
defined writing explicitly the Voigt notation $\alpha=(\alpha\beta)$ by

\begin{eqnarray}\label{Gmix}
G_{mix}^{(\alpha\beta)\gamma}(i)&=&
-a_0\sum_{j} \frac{c_L(ij)-c_T(ij)}{|\mbf R_{ij}|^2} 
R_{ij}^{\alpha} R_{ij}^{\beta} R_{ij}^{\gamma} -   \nonumber \\
      && -a_0\frac{1}{2}\sum_{j}  c_T(ij)   (R_{ij}^{\beta}  \delta_{\alpha\gamma}+
R_{ij}^{\alpha}  \delta_{\beta\gamma}) 
\end{eqnarray}

Note a symmetry property: 
 in case of inversion symmetry at the position of an ion, the 
phonon-strain interaction constants $G_{mix}^{\alpha\gamma}(i)$ will be zero.

\subsection{The lattice dynamic terms $H_{E}+H_{int}$}

Coming back to equations (\ref{Ur2}) and (\ref{U2}) we rewrite 
the terms quadratic in the phonon displacement operators 


\begin{eqnarray}
(\mbf R_{ij}^T \hat \mbf P_{ij})^2&=&\sum_{\alpha,\beta=1,2,3} R_{ij}^{\alpha}R_{ij}^{\beta}(\hat P_{j}^{\alpha}-\hat P_{i}^{\alpha})(\hat P_{j}^{\beta}-\hat P_{i}^{\beta})  
\end{eqnarray}

\begin{eqnarray}
\hat \mbf P_{ij}^T \hat \mbf P_{ij}&=&\sum_{\alpha=1,2,3} (\hat P_{j}^{\alpha}-\hat P_{i}^{\alpha})^2 \nonumber \\
&=& \sum_{\alpha=1,2,3} (\hat P_{j}^{\alpha})^2 +(\hat P_{i}^{\alpha})^2- 2 \hat P_{i}^{\alpha} \hat P_{j}^{\alpha}
\end{eqnarray}



The single ion Hamiltonian $H_{\rm E}$ is similar to the linear chain, in general

\begin{eqnarray}
\hat H_{\rm E}&=& \sum_{i} \hat H_{\rm E}(i) \nonumber \\
 \hat H_{\rm E}(i)&\equiv&\frac{a_0^2 \hat \mbf w_i^2}{2 m_i} -\frac{1}{2} 
\sum_{\alpha} K_{\alpha\beta}(ii) \hat P_{i}^{\alpha}\hat P_{i}^{\beta}a_0^{-2}   
\end{eqnarray}

{\color{blue}
\begin{eqnarray}
K_{\alpha\beta}(ii)&\equiv& -a_0^2\sum_{j} \frac{R_{ij}^{\alpha}R_{ij}^{\beta}}{|\mbf R_{ij}|^2} 
[c_L(ij)-c_T(ij)] + \nonumber \\
& & +  \delta_{\alpha\beta} c_T(ij)
\end{eqnarray}
}
and the interaction Hamiltonian  $H_{\rm int}$

\begin{eqnarray}
\hat H_{\rm int}&=&-\frac{1}{2}\sum_{i\ne j,\alpha\beta} K_{\alpha\beta}(ij) \hat P_{i}^{\alpha}\hat P_{j}^{\beta}a_0^{-2}  \\
\end{eqnarray}
{\color{blue}
\begin{eqnarray}
K_{\alpha\beta}(ij)&\equiv&a_0^2\frac{R_{ij}^{\alpha}R_{ij}^{\beta}}{|\mbf R_{ij}|^2}[c_L(ij)-c_T(ij)]
+a_0^2\delta_{\alpha\beta}c_T(ij) \nonumber
\end{eqnarray}
}

\subsection{The total Crystal field Phonon Interaction Hamiltonian}


To calculate  crystal field parameters $B_{\gamma}(i)=B_l^m$ ($\gamma$ is a short hand
notation for $lm$) for the 
ion at site $i$ in the crystal with Stevens convention 
we use the well known expressions of the point charge model:

\begin{equation}\label{pcmodelclc}
B_{\gamma}(i)=B_l^m=-|e|\theta_l^i\langle r^l \rangle \gamma_{lm}(i) p_{lm}
\end{equation}

Here, the $\gamma_{lm}(i)$ are computed from the relative position $R_{ji},\Omega_{ji}$
 (in spherical coordinates)
of the point charges $q_j$ using


\begin{equation}\label{pcmodelgamma}
\gamma_{lm}(i) = \sum_j  \frac{q_j}{2l+1}\frac{Z_{lm}(\Omega_{ji})}{\epsilon_0R_{ji}^{l+1}}
\end{equation}

Equation (\ref{pcmodelgamma}) is based on the adiabatic approximation and 
the dependence of the crystal field parameters on the displacements $\hat \mbf U_i$ is not considered.
Going beyond the adiabatic approximation this dependence may be considered and 
the crystal field parameters may be expanded in terms of the strain $\epsilon$ and the 
$\hat \mbf P_i$ leading to the crystal field phonon interaction. Minimizing the Energy with respect to
the strain tensor $\epsilon$ leads to expressions for the strain $\epsilon$ in terms of expectation values
of Stevens Operators and displacement operators. 

The Hamiltonian can be written as a sum of phonon, crystal field and Zeeman contributions
 (writing $\gamma$ instead of $lm$ for the crystal field parameter
indices, the $i$ index counts nuclei carrying with them
the charge producing the crystal field $i=1,...,N_{\rm nuclei}$, 
the $j$ index runs over magnetic ions,i.e. the only partially filled
f or d electron shells  in a crystal $j=N_{\rm nuclei}+1,... N_{\rm nuclei}+N_{\rm magnetic ions}$,
this is why we writ $i<j$ in the following sums)

\begin{eqnarray}\label{Hamiltonian}
\hat H_{\rm tot} & = & \hat H_{\rm phon} +\sum_{i,\gamma} B_{\gamma}(i,\hat \mbf U_1,...,\hat \mbf U_N) O_{\gamma}(\hat \mbf J_i) - \sum_i g_{J_i} \mu_B \hat \mbf J_i \mbf H \\
&=& \hat H_{\rm phon} +\sum_{i,\gamma} B_{\gamma}(i,0,\dots,0) O_{\gamma}(\hat \mbf J_i) + \hat H_{\rm cfph}\nonumber 
 \end{eqnarray}

\begin{eqnarray}\label{Hcfph}
\hat H_{\rm cfph}&=&\sum_{i<j,\gamma}\nabla_{\hat \mbf U_i} B_{\gamma}(j) \hat \mbf U_i O_{\gamma}(\hat \mbf J_j) \nonumber \\
&=& \sum_{i<j,\gamma}\nabla_{\hat \mbf U_i} B_{\gamma}(j) (\bar a \mbf R_i + \hat \mbf P_i)O_{\gamma}(\hat \mbf J_j) \nonumber \\
&=& \sum_{i<j,\alpha\beta=1,2,3,\gamma=1,...} R_i^{\beta}  \frac{\partial B_{\gamma}(j)}{\partial \hat U_i^{\alpha}}  
  \epsilon_{\alpha\beta} O_{\gamma}(\hat \mbf J_j) \nonumber \\
 &&+\sum_{i<j,\gamma}\nabla_{\hat \mbf U_j} B_{\gamma}(j) \hat \mbf P_i O_{\gamma}(\hat \mbf J_j) \nonumber \\
 \end{eqnarray}

In the last line of this equation we have made use of the invariance of the
total crystal field energy under rotations
and therefore substituted $\bar a$ with the strain $\bar \epsilon$.
Abbreviating notation we arrive at the final result for the crystal field phonon interaction:

\begin{eqnarray}\label{Hcfphsub}
\hat H_{\rm cfph}&=& -\sum_{j,\alpha=1,..,6,\gamma=1,...} G_{\rm cfph}^{\alpha\gamma}(j) \epsilon_{\alpha} O_{\gamma}(\hat \mbf J_j) \nonumber \\
 &&-\sum_{i<j,\alpha=1,2,3,\gamma=1,...} \Gamma^{\alpha\gamma}(ij) \hat P_i^{\alpha}a_0^{-1} O_{\gamma}(\hat \mbf J_j)
 \end{eqnarray}

Note, that the first part of this equation denotes the coupling of the strain to the crystal field
and is commonly known as magnetoelastic interaction~\cite{morin90-1}.
The definition of the static magnetoelastic constants writing explicitly the Voigt notation of the first index $\alpha=(\alpha\beta)$
 is

{\color{blue}
\begin{equation}\label{Gcfph}
 G_{\rm cfph}^{(\alpha\beta)\gamma}(j) =
-\frac{1}{2}\sum_{i}( R_i^{\beta}  \frac{\partial B_{\gamma}(j)}{\partial \hat  U_i^{\alpha}} 
+ R_i^{\alpha} \frac{\partial B_{\gamma}(j)}{\partial \hat  U_i^{\beta}} ) 
\end{equation}
}

The dynamic magnetoelastic constants (the crystal field phonon coupling constants) are

{\color{blue}
\begin{equation}\label{Gammacfph}
\Gamma^{\alpha\gamma}(ij)= -a_0\frac{\partial B_{\gamma}(j)}{\partial \hat  U_i^{\alpha}}
 \end{equation}
}

Note, that equation (\ref{Gcfph}) makes use of the displacement derivatives of the 
crystal field parameters and not the strain derivatives found in literature
 \cite{palmer78-2465,jensen75-320,mcewen91-3298}. In section~\ref{determinePcmodel} a procedure
to calculate these derivatives is described.

Summarizing, and remembering the dimensionless phonon displacement operators
$\hat \mbf u = \hat \mbf P / a_0$, we can write the total Hamiltonian as

{\color{blue}
\begin{eqnarray}\label{Htot}
\hat H_{\rm tot} & = & \sum_{i,\gamma} B_{\gamma}(i,0,\dots,0) O_{\gamma}(\hat \mbf J_i) - \sum_i g_{J_i} \mu_B \hat \mbf J_i \mbf H+ \\
&+&\sum_{i} \hat H_{\rm E}(i) + \frac{1}{2}\sum_{\alpha\gamma=1-6} c^{\alpha\gamma} \epsilon_{\alpha}\epsilon_{\gamma} - \nonumber \\
&-&\frac{1}{2}\sum_{i\ne j,\alpha\beta} K_{\alpha\beta}(ij) \hat u_{i}^{\alpha}\hat u_{j}^{\beta} 
-\sum_{i<j,\alpha=1-6,\gamma}\Gamma^{\alpha\gamma}(ij) \hat u^{\alpha}_i O_{\gamma}(\hat \mbf J_j)
\nonumber \\
&-&\sum_{i,\alpha=1-6,\gamma=1,2,3} G_{mix}^{\alpha\gamma}(i) \epsilon_{\alpha} \hat u_{i}^{\gamma} - \nonumber \\
&-&\sum_{i,\alpha=1-6,\gamma=1,...} G_{\rm cfph}^{\alpha\gamma}(i) \epsilon_{\alpha} O_{\gamma}(\hat \mbf J_i) \nonumber
 \end{eqnarray}
}


The first line in (\ref{Htot}) contains the single ion Hamiltonian (crystal field, Zeeman)
 
and the second line the phonon Einstein oscillator and elastic energy (also a ''single ion'' terms),
the third line the interaction terms (phonon, crystal field phonon), 
the forth line the mixing term and 
the last line  the magnetoelastic term. 
In a selfconsistent solution it is possible to 
determine (i) $\epsilon$, (ii) $\langle \hat O_l^m(i) \rangle $ and (iii)
$\langle \hat \mbf u_i \rangle$. This will produce multipolar phase diagrams including the
magnetostrictive properties without the need of a detailed investigation of the symmetry adapted Hamiltonian.

Setting zero the derivative of the expectation value of the Hamiltonian (\ref{Hamiltonian}) with respect to
 $\bar a$  (i.e. minimizing the energy with respect to strain and  rotation) 
yields the following relations 


\begin{eqnarray}\label{zeroa}
0&=& \sum_{ij} \frac{c_L(ij)-c_T(ij)}{|\mbf R_{ij}|^2} 
(\mbf R_{ij}^T\bar \epsilon \mbf R_{ij})R_{ij}^{\alpha}R_{ij}^{\beta}   \\
      && + \frac{c_T(ij)}{2} (R_{ij}^{\alpha} (\bar \epsilon \mbf R_{ij})^{\beta}+R_{ij}^{\beta} (\bar \epsilon R_{ij})^{\alpha}) \nonumber \\
&& +2\sum_{ij} \frac{c_L(ij)-c_T(ij)}{|\mbf R_{ij}|^2} 
R_{ij}^{\alpha}R_{ij}^{\beta} \mbf R_{ij}^T \langle \mbf P_{i} \rangle  \nonumber \\
      && +2 c_T(ij) \mbf  R_{ij}^{\beta} \langle \hat P_{i}^{\alpha} \rangle
+ \sum_{ij,\gamma}\frac{\partial B_{\gamma}(i)}{\partial \hat U_j^{\alpha}}   R_j^{\beta}  \langle O_{\gamma}(\hat \mbf J_i) \rangle \nonumber
 \end{eqnarray}

The index $i$ needs only to go over the atoms in the unit cell, because the crystal structure
is periodic. There are 9 equations for $\alpha,\beta=1,2,3$ for nine components
$a_{\alpha\beta}$. Thus the coefficients of the strain  in  equation (\ref{zeroa})
can be evaluated numerically. For each mean field iteration the strain $\epsilon_{\alpha\beta}$ components
can be calculated from equation (\ref{zeroa}) and inserted into (\ref{Hamiltonian}) until
selfconsistency is achieved.

Considering only the strain $\bar \epsilon$ we can make use of the elastic constants
in forming the derivative of the Hamiltonian, 
we get 6 equations for $\alpha=1,...,6$, from which the 6 strain components can be determined.



\begin{eqnarray}\label{zeroeps}
\sum_{\beta=1,..6}  c^{\alpha\beta} \epsilon_{\beta} &=&
 \sum_{i,\delta=1,2,3}  G_{mix}^{\alpha\delta}(i)\langle \hat u_{i}^{\delta} \rangle  \\
 &+& \sum_{i,\gamma=1,...} G_{\rm cfph}^{\alpha\gamma}(i) \langle O_{\gamma}(\hat \mbf J_i) \rangle \nonumber
\end{eqnarray}


\subsection{Exchange Striction}

If the two ion interactions are strain dependent, an additional contribution to the
strain tensor has to be considered in the equations (\ref{Htot}) to (\ref{zeroeps}).
The resulting contribution is commonly called {\em exchange striction}, because
predominantly the magnetic exchange interactions will contribute.

We therefore extend the model by the exchange striction writing the 
Hamiltonian


\begin{equation}\label{Hwithexstric}
\hat H=\hat H_{\rm tot} -\frac{1}{2} \sum_{ij,\alpha\beta}
 {\mathcal J}_{\alpha\beta}((1+\bar a)(\mbf R_{j} - \mbf R_i)) \hat \mathcal I_{\alpha}^i \hat \mathcal I_{\beta}^{j}
\end{equation}

and expand the two ion interaction ${\mathcal J}_{\alpha\beta}(\mbf R_{ij}) $
in a Taylor expansion for small strain:


\begin{eqnarray}\label{jjtaylor}
{\mathcal J}_{\alpha\beta}(\mbf R=(1+\bar a)\mbf R_{ij})&=&{\mathcal J}_{\alpha\beta}(\mbf R_{ij})
+ \frac{\partial {\mathcal J}_{\alpha\beta}}{\partial R^{\alpha}} \frac{\partial (\bar a\mbf R_{ij})^{\alpha}}{\partial \epsilon_{\beta}} \epsilon_{\beta} + \dots \nonumber \\
&=& {\mathcal J}_{\alpha\beta}(\mbf R_{ij})+ 
\sum_{\alpha'\gamma=1,2,3,\beta'=1,..,6}
\frac{\partial {\mathcal J}_{\alpha\beta}}{\partial R^{\alpha'}}
\frac{\partial \epsilon_{\alpha'\gamma}\mbf R_{ij}^{\gamma}}{\partial \epsilon_{\beta'}} \epsilon_{\beta'}
\end{eqnarray}

Note that the second part of equation~(\ref{jjtaylor}) describes the exchange striction Hamiltonian, if
inserted in equation~(\ref{Hwithexstric}). 

{\color{blue}
\begin{equation}\label{Hwexstric}
\hat H=\hat H_{\rm tot} -\frac{1}{2} \sum_{ij,\alpha\beta}
 {\mathcal J}_{\alpha\beta}(\mbf R_{ij}) \hat \mathcal I_{\alpha}^i \hat \mathcal I_{\beta}^{j}
 -\frac{1}{2} \sum_{ \stackrel{ij,\alpha\beta\alpha'\gamma=1,2,3}{\beta'=1,..,6}}
\frac{\partial {\mathcal J}_{\alpha\beta}}{\partial R^{\alpha'}}
\frac{\partial \epsilon_{\alpha'\gamma}\mbf R_{ij}^{\gamma}}{\partial \epsilon_{\beta'}} \epsilon_{\beta'}
 \hat \mathcal I_{\alpha}^i \hat \mathcal I_{\beta}^{j}
\end{equation}
}

In section~\ref{determineExstricpar} we show, how  
the position derivatives of the exchange parameters can be calculated.

Equation (\ref{zeroeps}) is extended for the case of exchange striction:

{\color{blue}
\begin{eqnarray}\label{zeroepsex}
\sum_{\beta=1,..,6}  c^{\alpha\beta} \epsilon_{\beta} &=&
 \sum_{i,\delta=1,2,3}  G_{mix}^{\alpha\delta}(i)\langle u_{i}^{\delta} \rangle  \\
 &+& \sum_{i,\gamma=1,...} G_{\rm cfph}^{\alpha\gamma}(i) \langle O_{\gamma}(\hat \mbf J_i) \rangle \nonumber \\
 &+& \frac{1}{2}\sum_{ii',\delta,\delta'\alpha',\gamma=1,..,3}
\frac{\partial {\mathcal J}_{\delta\delta'}(\mbf R_{ii'})}{\partial R^{\alpha'}}
\frac{\partial \epsilon_{\alpha'\gamma} R_{ii'}^{\gamma}}{\partial \epsilon_{\alpha}}
\langle \hat \mathcal I_{\delta}^i \hat \mathcal I_{\delta'}^{i'}\rangle \nonumber
\end{eqnarray}
}



\subsection{Progam Workflow for calculating Magnetoelastic Effects and the Static Jahn Teller Effect}

Technically the Hamiltonian (\ref{Htot}) is of the general form for 
McPhase~\cite{rotter12-213201}, with single
ion and interaction terms. The corresponding  mean field procedure can be done by an internal single ion module 
''epsilon'', which represents not an ion, but the
strain. Given $\langle \hat  u_{j}^{\alpha} \rangle$ and $\langle O_{\gamma}(\hat \mbf J_i) \rangle$ and
appropriate interaction constants to the ''epsilon'' from (\ref{Hmix}) and (\ref{Hcfph}), respectively,
the right side in (\ref{zeroeps}) can be evaluated in each mean field loop 
in module ''epsilon'' function {\em Icalc}. Then elastic constants in (\ref{zeroeps}) can be used to calculate
a new strain $\bar \epsilon$, a six component vector. Via the aforementioned appropriate
interaction constants (from (\ref{Gmix}) and (\ref{Gcfph})) the strain will produce mean
fields on lattice displacements and magnetic ions charge density (crystal field) and so on ...
The free energy returned by the module ''epsilon''  correspond to the elastic energy
per unit cell given by equation (\ref{Eelconstvoigt}). 

For the dynamics the ''epsilon'' module does not
yield any single ion excitations. The excitations can be calculated without taking into account the
term $H_{mix}$, because this is linear in the displacement operators and in the harmonic
approximation the spectrum of the harmonic Einstein oscillator will not change with such an
internal force. The strain has only to be taken into account as a linear modification of the crystal field
parameters in the first term of equation (\ref{Hcfph}). This is done automatically by creating the
file {\em mcphas.mf} with the ''epsilon'' module, i.e. the option ''-doeps'' of {\prg mcphas}, see
section~\ref{magelastmcphasoptions}.

\subsection{Calculation of the Elastic Constants using {\prg makenn}}\label{determineElconst}

Elastic constants and mixing term parameters $G_{mix}$ should be created with the
option {\em -bvk} of the program {\em makenn} and stored in the 
file {\em mcphas.j}. Using {\em makenn} with the option {\em -cfph} will create
the magnetoelastic parameters $G_{\rm cfph}$ and $\Gamma_{\rm cfph}$.

\subsection{Calculation of the Crystal Field Phonon Interaction using a Point Charge Model}\label{determineBlmfromPC}

If crystal field - phonon coupling is to be applied in running {\prg mcphas} and {\prg mcdisp},
 magnetic ions may be added to a phonon set-up of {\prg mcphas.j}.
These magnetic ions should not be placed at exactly the same position as the phonon atoms - i.e. da db dc should be chosen
(slightly) different to enable the mcphas.j loader to identify which is which.

The coupling between phononic and crystal field degrees of freedom can be calculated from equations
(\ref{Gcfph}) and (\ref{Gammacfph}).
 This can be done
using the program  {\prg makenn} with option {\prg -cfph}  applying small differential changes to
the atomic positions in the unit cell in order to evaluate the gradient of the crystal field
parameters. {\prg makenn} will do this calculation for every magnetic ion in the unit cell,
i.e. for all ions which have the {\prg so1ion} or {\prg ic1ion} module in the {\prg sipf} file.
It will create for these ions a new position in makenn.j with the required
shift and also store a designated sipf file. After running {\prg makenn} with option {\prg -cfph} 
the user should only take care to change the module {\prg so1ion}  or {\prg ic1ion} to  or {\prg phonon}
in the original sipf files of the magnetic ions so that there is always one phononic and one magnetic
sipf file for each magnetic atom in the unit cell.

For examples see  several models on crystal field phonon interaction 
in {\prg examples}, for starting a linear chain of Ce$^{3+}$ ions including a
theory manuscript can be found in {\prg examples/Ce3p\_chain\_cfphonon}.

Note {\prg makenn}  makes use of the program {\prg pointc}\index{pointc} to calculate 
crystal field parameters $B_{\gamma}(i)=B_l^m$ ($\gamma$ is a short hand
notation for $lm$) at site $i$  with Stevens convention (or alternatively also $L_l^m$ with
Wybourne normalisation, see appendix~\ref{cfparconventions} for details and definition of symbols) 
using the well known expressions of the point charge model (\ref{pcmodelclc}).



\subsection{Setting up a Point Charge Model which yields given Crystal Field Parameters}\label{determinePcmodel}

Sometimes some or all crystal field parameters are available, either from literature or determined from
the high temperature expansion of inverse susceptibility etc.  In these cases the point charge model
should reproduce these crystal field parameters and we describe here a procedure which has been tested
with some success on a number of cases.

As an example we consider the ternary compound TmNiC$_2$, denoting point charges on
Tm, Ni and C by $C_{Tm}$, $C_{Ni}$ and $C_{C}$, respectively.
The full calculation with all input files and logbook {\prg calc.html} is available in {\prg examples/tmcni2 }.

From the high temperature expansion of the inverse susceptibility the second order crystal field
parameters of this orthorhombic phase have been determined by~\cite{roman23-125137}
to be $B_2^0=0.22$~meV and $B_2^2=-0.33$~meV. 
We make now use of the fact, that every crystal field parameter is linear dependent on
the point charge, compare equations~(\ref{pcmodelclc}) and (\ref{pcmodelgamma}).
Thus we get the following set of linear equations:

\begin{eqnarray}\label{pclinredproblem}
B_2^0 &=& {\rm Ni}_{20} C_{Ni}+{\rm C}_{20} C_{C}+{\rm Tm}_{20} C_{Tm} \nonumber \\
B_2^2 &=& {\rm Ni}_{22} C_{Ni}+{\rm C}_{22} C_{C}+{\rm Tm}_{22} C_{Tm} \nonumber \\
0 &=& C_{Ni}+2 {\rm C}_{C}+ C_{Tm} 
\end{eqnarray}

The last equation ensures the charge neutrality and should also be taken into account.
The coefficients ${\rm Ni}_{lm}$, ${\rm C}_{lm}$ and ${\rm Tm}_{lm}$ can be obtained by
a point charge calculation with setting all charges zero except one (which is set to $1|e|$) and
evaluating the computed crystal field parameter $B_l^m$. 

The following code snippet shows how to calculate the coefficients
C ${\rm C}_{20}$ and ${\rm C}_{22}$ using {\prg cif2mcphas} in linux:

\begin{verbatim}
cif2mcphas -pc 20 -sp -nm Ni -ch Tm=0,Ni=0,C=1 TmNiC2.cif
#  extract coefficient 
. getvariable B20 Tm1_1.sipf
# stream editor sed is used to ensure if it is zero ("not found") a 0.0 is stored in C20 instead of "not"
export C20=$(echo $MCPHASE_GETVARIABLE_VALUE | sed s/not/0.0/ )
. getvariable B22 Tm1_1.sipf
export C22=$(echo $MCPHASE_GETVARIABLE_VALUE | sed s/not/0.0/ )
\end{verbatim}

\dots and in windows:

\begin{verbatim}
call cif2mcphas -pc 20 -sp -nm Ni -ch Tm=0,Ni=0,C=1  TmNiC2.cif 
echo B20=0.0 > dd
echo B22=0.0 >> dd
REM then push the sipf file onto dd - if it contains B20 or B22 these will be read
type Tm1_1.sipf >> dd
call getvariable B20 dd
set C20=%MCPHASE_GETVARIABLE_VALUE%
call getvariable B22 dd
set C22=%MCPHASE_GETVARIABLE_VALUE%
\end{verbatim}


\begin{figure}[htb]%h=here, t=top, b=bottom, p=separate figure page
\begin{center}\leavevmode
\includegraphics[bb=0 0 300 300,angle=0, width=0.4\textwidth]{../examples/tmnic2/B20convergence.jpg}
\end{center}
\caption{Convergence of the point charge calculation of the crystal field parameter $B_2^0$ in TmNiC$_2$.}
\label{B20convergence}
\end{figure}

Convergence is reached in the current
system when point charges up to a distance of $r_{max}=20$~\AA\  are taken into account, see fig.~\ref{B20convergence}. For the
radial integrals $\langle r^l \rangle$ of Tm$^{3+}$ we used the valus given by~\cite{edvardsson98-230} including
the shielding factors $\sigma$.
Thus we arrived at a system of 3 equations (\ref{pclinredproblem}) with 3 unknown charges. This system may be readily solved. However,
it turns out, that the computed charges are unreasonable large and have unreasonable signs.

\begin{figure}[htb]%h=here, t=top, b=bottom, p=separate figure page
\begin{center}\leavevmode
\includegraphics[bb=0 0 500 500, angle=0, width=0.4\textwidth]{../examples/tmnic2/model.jpg}
\end{center}
\caption{Point charge model for TmNiC$_2$, large spheres correspond to atomic positions, small spheres to
electronic bond charges $E^1, \dots, E^6$.}
\label{modelTmNiC2}
\end{figure}

Therefore, we introduced some additional point charges and placed these on the lines connecting
a pair of atoms. This choice was motivated by the intuition, that the covalent bonding between
the elements will lead to some anisotropic electronic charge density. We had to introduce 
six such point charges, these are indicated by $E^1, \dots, E^6$ and refer to electronic bond
charges for  C-C, C-Ni, C-Ni, Tm-C, Tm-Ni and Tm-Tm pairs, respectively (see fig~\ref{modelTmNiC2}) 
 Here we show the corresponding part of the input cif file:

\begin{verbatim}
loop_
  _atom_site_label
  _atom_site_type_symbol
  _atom_site_fract_x
  _atom_site_fract_y
  _atom_site_fract_z
  _atom_site_U_iso_or_equiv
  _atom_site_adp_type
  _atom_site_occupancy
  _atom_site_site_symmetry_order
  _atom_site_calc_flag
  _atom_site_refinement_flags_posn
  _atom_site_refinement_flags_adp
  _atom_site_refinement_flags_occupancy
  _atom_site_disorder_assembly
  _atom_site_disorder_group
 Tm1 Tm 0.5000 0.0000 0.3850(2) 0.0078(2) Uani 1 4 d S T P . .
 Ni1 Ni 0.0000 0.0000 -0.0034(3) 0.0089(3) Uani 1 4 d S T P . .
 C1 C 0.0000 0.349(2) 0.1851(18) 0.0105(16) Uani 1 2 d S T P . .
; E1 C-C bond charge as 0 0.5 z as C1
 E1 E 0.0000 0.500(2) C1        0.0105(16) Uani 1 2 d S T P . .
; atomic radii  Tm 227 pm C 170pm  Ni 163pm
; E2 E3 C-Ni bonds at 
E2 E Ni1+(C1-Ni1)*163/(163+170) Ni1+(C1-Ni1)*163/(163+170) Ni1+(C1-Ni1)*163/(163+170)   0.0105(16) Uani 1 2 d S T P . .
;  ... and
E3 E Ni1+(C1-Ni1)*163/(163+170) Ni1+0.5+(C1-Ni1-0.5)*163/(163+170) Ni1+0.5+(C1-Ni1-0.5)*163/(163+170)	  0.0105(16) Uani 1 2 d S T P . .
; E4 Tm - C bond charge at
 E4 E C1+(Tm1-C1)*170/(227+170)  C1+(Tm1-C1)*170/(227+170)   C1+(Tm1-C1)*170/(227+170)    0.0105(16) Uani 1 2 d S T P . .
; E5 Tm - Ni bond charge at
 E5 E N1+(Tm1-Ni1)*163/(227+163) N1+(Tm1-Ni1)*163/(227+163) N1+(Tm1-Ni1)*163/(227+163)   0.0105(16) Uani 1 2 d S T P . .
; Tm - Tm bond charge at
E6 E  Tm1 Tm1+0.25  Tm1+0.25 0.0105(16) Uani 1 2 d S T P . .
\end{verbatim}



The set of equations (\ref{pclinredproblem}) has to be extended for these bond electron point charges and
we get

\begin{eqnarray}\label{pclinproblem}
B_2^0 &=& {\rm Ni}_{20} C_{Ni}+{\rm C}_{20} C_{C}+{\rm Tm}_{20} C_{Tm}+\sum_{i=1}^6 E^i_{20} C_{E^i} \nonumber \\
B_2^2 &=& {\rm Ni}_{22} C_{Ni}+{\rm C}_{22} C_{C}+{\rm Tm}_{22} C_{Tm} +\sum_{i=1}^6 E^i_{22} C_{E^i} \nonumber \\
0 &=& C_{Ni}+2 {\rm C}_{C}+ C_{Tm} +C_{E^1}+2C_{E^2}+2C_{E^3}+4C_{E^4}+2C_{E^5}+2C_{E^6}
\end{eqnarray}


By variation of the charges $C_{E^i}$ a set of point charges and crystal field parameters
could be found, which reproduces closely the crystal field effects reported in~\cite{roman23-125137}, see
table~\ref{pcmodel}.

\begin{table}[thb] 
\begin{center}  
\caption {Point Charge Model for TmNiC$_2$}  
\label{pcmodel} 
\begin{tabular} 
{l|l|l|l|l|l}
atom	& bond	& charge($|e|$) & da  & db    & dc \\
\hline
Tm	&	& 2.159	    & 0.5   &	0   &	0.385 \\
Ni	&	& -0.339    & 0	    &	0   &	-0.0034 \\
C	&	& 3.230	    & 0	    & 0.349 &	0.1851  \\
$E^1$	& C-C	& -2.6945   & 0	    &	0.5 &	0.1851  \\
$E^2$	& C-Ni  & -1.5506   & 0	    &	0.17083&0.08887  \\
$E^3$	& C-Ni  & -0.3955   & 0	    &	0.42609&0.34412  \\
$E^4$	& Tm-C  & -0.2008   & 0.21411&	0.19955&0.2707  \\
$E^5$	& Tm-Ni & -0.2084   & 0.20897&	0   &	0.16233  \\
$E^6$	& Tm-Tm & -0.2369   & 0.5   &	0.25&	0.635  \\
\hline
$B_2^0$ & 0.22 meV &&&\\
$B_2^2$ &-0.328 meV &&&\\
$B_4^0$ &-0.00105 meV &&&\\
$B_4^2$ & 0.00296 meV &&&\\
$B_4^4$ &-0.00994 meV &&&\\
$B_6^0$ &-3.604e-06 meV &&&\\
$B_6^2$ & 7.969e-06 meV &&&\\
$B_6^4$ & 1.574e-05 meV &&&\\
$B_6^6$ & 2.11e-05 meV &&&\\
\end{tabular}
\end{center}   
\end{table}

This set of parameters was used to calculate the thermodynamic and spectroscopic physical properties
using {\prg mcphasit} and {\prg mcdispit}, respectively. Note that in such calculations the
bonding charges $E^1, \dots, E^6$ have to be removed, because electrons are correlated and 
cannot move freely in a solid (compare Pauli principle, failure of the Drude model).
Technically we have to remove the E charges from the list of atoms in 
{\prg mcphas.j} to prevent {\prg mcphas} from calculating too large energy u and {\prg mcdisp} 
from calculating optical modes due to vibrations of electronic bonding charges E.
This can be done with {\prg reduce\_unitcell} with option {\prg -delatom}.
In {\prg examples/tmnic2/calc.bat } the procedure is described in detail.


\subsection{Calculation of the Exchange Striction Parameters using {\prg makenn}}\label{determineExstricpar}

The positional derivatives of the two ion interactions in equation~(\ref{Hwexstric})
has to be evaluated in order to calculate the exchange striction. Within the McPhase 
package the program {\prg makenn}  may perform this task and it will use analytical formulas.
Here comes a typical code example:

\begin{verbatim}
REM create exchange interaction
call makenn 7 -kaneyoshi 6 4.5 4
call copy results/makenn.j mcphas.j
REM ... and its position derivatives using the -djd... option of makenn
call makenn 7 -kaneyoshi 6 4.5 4 -djdx
call makenn 7 -kaneyoshi 6 4.5 4 -djdy
call makenn 7 -kaneyoshi 6 4.5 4 -djdz
call copy results/makenn.djdx mcphas.djdx
call copy results/makenn.djdy mcphas.djdy
call copy results/makenn.djdz mcphas.djdz

REM start mcphase with option -doeps to calculate strain
call mcphasit -doeps 
\end{verbatim}

\subsection{Magnetoelastic Options to {\prg mcphas}}\label{magelastmcphasoptions}

\begin{itemize}
\item[{\prg -doeps}]
If the program {\em mcphas} is started with option {\prg -doeps}
and it finds elastic constants in the input file {\em mcphas.j} (note, that the elastic constants
in the input file are normalised to the primitive crystallographic unit cell, units are 
meV / primitive crystallographic unit cell), it
 will use these and determine selfconsistently the strain $\epsilon$ by solving equations
(\ref{zeroeps}) and (\ref{Htot}). Elastic energy and strain tensor are stored in {\em results/mcphas.fum}.
 In this way it is be possible to model Jahn Teller transitions,
phase diagrams, magnetostriction, thermal expansion (magnetic part) and dynamics consistently based
only on point charges and Born von Karman springs. 
If {\prg mcphas} is used with option {\prg -doeps} and it finds files {\prg mcphas.djdx}, {\prg mcphas.djdy}, {\prg mcphas.djdz} with
derivatives of the two ion interaction parameters with respect to
spatial coordinates $x,y,z$ respectively, it performs a calculation of
the corresponding correlation functions and computes the strain  using
equation (\ref{zeroepsex}) at each iteration. Subsequently, making use of the Taylor expansion
of the interaction constants (\ref{jjtaylor}) the computed strain is used 
in the next iteration for the the mean field Hamiltonian 
(mean field theory is used by {\em McPhase}
to solve the Hamiltonian (\ref{Hwithexstric}) and (\ref{Htot})). 
At the end of the iteration loop mean fields, moments $\langle I \rangle$ and 
a selfconsistent strain tensor $\epsilon$ is obtained.
 
\item[{\prg -linepscf}]
The option {\prg -linepscf} together with {\prg -doeps} will trigger 
a calculation where equation (\ref{zeroeps}) (and in case
of exchange striction terms (\ref{zeroepsex}))
is used to calculate the strain $\epsilon$, however the last two terms in
(\ref{Htot}) are always evaluated
for zero strain, i.e. the crystal field striction is not calculated
selfconsistently, but only in the linear approximation assuming that the 
strain leads only to a negligible perturbation of the crystal field
Hamiltonian. 

\item[{\prg -linepsjj}]
The option {\prg -linepsjj} together with {\prg -doeps} will trigger 
a calculation where equation (\ref{zeroepsex})
is used to calculate the strain $\epsilon$, yet the modification of
the two ion interaction by the strain is assumed to be small
and two ion interaction parameters (\ref{jjtaylor}) are used
in the mean field loop always for zero strain, i.e. the exchange striction is not calculated
selfconsistently, but only in the linear approximation assuming that the 
strain leads only to a negligible perturbation of the two ion interaction
Hamiltonian.


\end{itemize}


The length change $\Delta L/L$ of a sample in a dilatometer experiment
can be calculated from the strain tensor components
 using~\footnote{S. Bluegel, Juelich, private communication}

\begin{equation}
\frac{\Delta L}{L}=\sum_{\alpha\beta} \epsilon_{\alpha\beta} \hat L_{\alpha} \hat L_{\beta}
\end{equation}

where $\hat \mbf  L$ denotes the unit vector in the direction of
measurement.

\subsection{TmCu$_2$ - magnetoelastic effects and Neutron Spectra calculated with CF-phonon interaction}

As an example for static and dynamic calculations including the crystal-field phonon interaction
we present a ''simple'' calculation on TmCu$_2$, all input files ({\prg calc.bat}
and logbook {\prg calc.pdf} with all commands can be found in {\prg examples/tmcu2\_cf\_phonon}.
''Simple'' in this context means, that the input parameters are quite basic and not fine
tuned to match the experiment. As input we used for this orthorhombic system the
crystal structure (see {\prg TmCu2.cif}), point charges $C_{Tm}=0.8|e|$ $C_{Cu}=-0.4|e|$
with a cutoff radius of 15~\AA\ and
a  distance dependent Born von Karman longitudinal spring model as output by default
from {\prg makenn -bvk}.

\begin{figure}[htb]%h=here, t=top, b=bottom, p=separate figure page
\begin{center}\leavevmode
\includegraphics[bb=0 0 300 300,angle=0, width=0.4\textwidth]{../examples/tmcu2_cf_phonon/thecf.jpg}
\end{center}
\caption{Crystal field striction, i.e. the crystal field influence on the thermal expansion
in TmCu$_2$.  - calculation results}
\label{tmcu2thecfclc}
\end{figure}

\begin{figure}[htb]%h=here, t=top, b=bottom, p=separate figure page
\begin{center}\leavevmode
\includegraphics[bb=0 0 900 900,angle=0, width=0.4\textwidth]{../examples/tmcu2_cf_phonon/expdat/thecf_tmcu2_exp.jpg}
\end{center}
\caption{Crystal field striction, i.e. the crystal field influence on the thermal expansion
in TmCu$_2$. - experimental results~\cite{gratz93-7955}}
\label{tmcu2thecfexp}
\end{figure}

With {\prg mcphas -doeps} the static thermodynamic properties have been calculated, in particular
the strain tensor. The computed crystal field contribution to the thermal expansion (fig.~\ref{tmcu2thecfclc}) is 
in good agreement with the experimental data from powder X-ray diffraction (fig.~\ref{tmcu2thecfexp}).


\begin{figure}[htb]%h=here, t=top, b=bottom, p=separate figure page
\begin{center}\leavevmode
\includegraphics[bb=0 0 700 700,angle=0, width=0.6\textwidth]{../examples/tmcu2_cf_phonon/mag20K.jpg}
\end{center}
\caption{Dispersive modes in TmCu$_2$ along (11L) at $T=20$~K, symbol size corresponds to the 
magnetic scattering cross section, red: calculation without crystal field phonon interaction, blue:
calculation with crystal field phonon interaction.}
\label{tmcu2mag20k}
\end{figure}

By using {\prg mcdisp} at the temperature of 20~K the magnetic and nuclear inelastic neutron
scattering cross section can be calculated. We show in fig.~\ref{tmcu2mag20k} the magnetic scattering along
the (11L) direction. Including the crystal field phonon interaction clearly leads to a dynamic
Jahn-Teller effect: the flat modes  split, shift and get
dispersive and there is considerable magnetic intensity transfer to phonons.
Thus the double peak feature in unpublished experimental data on powder can possibly
be interpreted, see fig.~\ref{tmcu2mag20kexp}

\begin{figure}[htb]%h=here, t=top, b=bottom, p=separate figure page
\begin{center}\leavevmode
\includegraphics[bb=100 100 700 700,angle=0, width=0.6\textwidth]{../examples/tmcu2_cf_phonon/expdat/TmCu2 20K.jpg}
\end{center}
\caption{TmCu$_2$ neutron powder spectrum at $T=20$~K as measured on IN6 (Institute Laue Langevin) and
SV22 (FZ-Juelich), M. Loewenhaupt, private communication.}
\label{tmcu2mag20kexp}
\end{figure}

\clearpage

\subsection{Project - Extending McPhase to include Stress tensor}

 In a further step stress tensor components could be envisaged 
to act similar as an external magnetic field in {\em McPhase}, these will only act on the
''espilon'' module and on no other module. The equations given above have to be adapted accordingly 
and then it should be possible to calculate in addition to magnetic phase diagrams also stress dependence
of Jahn-Teller transitions and excitations. 


