\newpage
\section{Derivation of the Chargedensity Formula}\label{chargedensityoperator}
 
 The charge density operator for an electron system is given by a sum of delta
 functions on the location of the individual electrons in a system carrying
an elementary charge $e$:
 
 \begin{equation}
 \hat\rho(\mbf r) = \sum_i -|e| \delta(\mbf r_i - \mbf r)
 \end{equation} 
 
 Using spherical coordinates the delta function in this expression can be rewritten as
 
 \begin{equation}
 \delta(\mbf r_i - \mbf r) = \frac{1}{r^2}\delta(r-r_i) \delta(\Omega- \Omega_i)
 \end{equation} 
 
 The delta function of spherical harmonics can be expressed by spherical and tesseral harmonic
 functions (see appendix~\ref{tesseral}):
 
 \begin{equation}
  \delta(\Omega- \Omega_i)= \sum_{l,m} Y_l^{m\star}(\Omega_i) Y_l^{m}(\Omega)= \sum_{l,m} Z_l^{m}(\Omega_i) Z_l^{m}(\Omega)
 \end{equation} 
 
 Using the above equations and assuming the same radial part $R(r)$ of the wave function for all electrons in an unfilled %%@
shell the radial integrals in the expectation value of the charge density operator can be substituted by the radial %%@
wavefunction $R(r)$ and we obtain the expression
 
  \begin{equation}\label{chargedensity2}
	       \langle \hat\rho(\mbf r)\rangle=
	       -|e|  |R(r)|^2 \sum_{l,m}Z_l^{m}(\Omega) \langle\sum_i Z_l^{m}(\Omega_i)\rangle
   \end{equation} 

If higher multiplets are neglected, 
the expectation values of tesseral harmonics at the right side of equation (\ref{chargedensity2}) can
be rewritten using the operator equivalent method by Stevens\cite{hutchings64-227}:

  \begin{equation}\label{chargedensity_coefficients}
	      \langle\sum_i Z_l^{m}(\Omega_i)\rangle = |p_{lm}| \theta_{l} \langle O_l^m(\mbf J)\rangle_T 
	      \end{equation} 

For the notation see \cite{hutchings64-227}:
 $p_{nm}$ the pre factors of harmonic tesseral
functions $Z_{lm}(\Omega)$ as given in table IV in \cite{hutchings64-227} and in appendix~\ref{tesseral},
 and $\theta_l$
corresponds to the number of $4f$ electrons ($\nu$) in the
$4f^{\nu}$ configuration for $l=0$ and to
 the Stevens factors $\alpha$,$\beta$ and $\gamma$ for $l=$~2,4,6, respectively.

This results in the following expression for the chargedensity:
		  
  \begin{equation}\label{chargedensity_operator2}
	       \langle \hat\rho(\mbf r)\rangle=-|e||R_{4f}(r)|^2 \sum_{l=0,2,4,6}\sum_{m=-l,...,l}
	         |p_{lm}| \theta_{l} \langle O_l^m(\mbf J)\rangle_T Z_{lm}(\Omega)
	      \end{equation} 

  
