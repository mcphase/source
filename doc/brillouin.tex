\newpage
\section{Module {\prg brillouin} - an Isotropic Spin Single Ion Module}\label{brillouin}

A magnetic ion often can be described by a pure spin moment (transition
metal ions, $^{3+}$Gd, $^{2+}$Eu). In this description the magnetic moment
${\mbf M}=g_J \mu_B <{\hat \mbf J}>_T$ is calculated by the Brillouin function. 

\begin{equation}
<\hat \mbf J>=\frac{\mbf H}{|\mbf H|} B_{J}(x=g_J \mu_B |\mbf H|/kT)
\end{equation}

with

\begin{equation}
B_J(x)=\frac{\sum_{m=-J}^{J} m x^m}{\sum_{m=-J}^{J}  x^m}=
\frac{J(x^{J+2}-x^{-J})+(J+1)x(x^{-J}-x^J)}{(1-x)(x^{-J}-x^{J+1})}
\end{equation}

Note that the partition sum $Z$ is given by $Z=\sum_{m=-J}^{J}  x^m=\frac{x^{J+1}-x^{-J}}{x-1}$.
This is all what is needed to calculate the mean field ground state with 
module {\prg mcphas}.

For the calculation of the dynamical properties using module {\prg mcdisp\index{mcdisp}}
the single ion susceptibility of for a free magnetic ion is needed. In principle
there are (J-1) transitions with non vanishing matrix elements of
the momentum operator $\mbf J$ with all having the same
 excitation energy $\Delta=g_J \mu_B H$.
 These different transitions can
be combined into one effective inelastic excitation. The corresponding
transition matrix to be used in equation~(\ref{mmatrix}) of the MF-RPA formalism
is given by 

\begin{equation}
M_{\alpha\beta}=\frac{-b_{\alpha}b_{\beta}R_J}{Z}
\end{equation}

with the angular dependence given by the vector $\mbf b$

\begin{eqnarray}
b_x& = & \frac{-H_y+i H_x \frac{H_z}{|\mbf H|}}{2|\mbf H| sin\Theta} \\
b_y& = & \frac{H_x+i H_y \frac{H_z}{|\mbf H|}}{2|\mbf H| sin\Theta} \\
b_z& = & \frac{H_x^2+H_y^2}{2i|\mbf H|^2 sin\Theta}=\frac{-isin\Theta}{2} \\
&&sin\Theta=\frac{\sqrt{H_x^2+H_y^2}}{|\mbf H|}
\end{eqnarray}

and with the sum $R_J$ given by

\begin{equation}
R_J=(x-1)\sum_{m=-J}^{J-1}(J+m+1)(J-m)x^m=\frac{2Jx^{-J}+(2J+2)x(x^{J}-x^{-J})-2Jx^{J+2}}{(1-x)^2}
\end{equation}

The module given in {\prg /examples/gdni2b2c/1ion\_mod/brillouin.c} 
can be compiled
by typing make in the directory {\prg /examples/gdni2b2c/1ion\_mod/} thus using the
{\prg  /examples/gdni2b2c/1ion\_mod/makefile}. It 
evaluates the Brillouin function and thus calculates the thermal expectation 
value $<>_T$ of the vector $\mbf J$. 
The moment ${\mbf M}=g_J \mu_B <{\mbf J}>_T$
is returned to the mcphas\index{mcphas} program:

\newpage
\section{An Isotropic Spin - single ion module}\label{brillouin}

A magnetic ion often can be described by a pure spin moment (transition
metal ions, $^{3+}$Gd, $^{2+}$Eu). In this description the magnetic moment
${\mbf M}=g_J \mu_B <{\mbf J}>_T$ is calculated by the Brillouin function. 

\begin{equation}
<\mbf J>=\frac{\mbf H}{|\mbf H|} B_{J}(x=g_J \mu_B |\mbf H|/kT)
\end{equation}

with

\begin{equation}
B_J(x)=\frac{\sum_{m=-J}^{J} m x^m}{\sum_{m=-J}^{J}  x^m}=
\frac{J(x^{J+2}-x^{-J})+(J+1)x(x^{-J}-x^J)}{(1-x)(x^{-J}-x^{J+1})}
\end{equation}

Note that the partition sum $Z$ is given by $Z=\sum_{m=-J}^{J}  x^m=\frac{x^{J+1}-x^{-J}}{x-1}$.
This is all what is needed to calculate the mean field ground state with 
module {\prg mcphas}.

For the calculation of the dynamical properties using module {\prg mcdisp\index{mcdisp}}
the single ion susceptibility of for a free magnetic ion is needed. In principle
there are (J-1) transitions with non vanishing matrix elements of
the momentum operator $\mbf J$ with all having the same
 excitation energy $\Delta=g_J \mu_B H$.
 These different transitions can
be combined into one effective inelastic excitation. The corresponding
transition matrix to be used in equation~(\ref{mmatrix}) of the MF-RPA formalism
is given by 

\begin{equation}
M_{\alpha\beta}=\frac{-b_{\alpha}b_{\beta}R_J}{Z}
\end{equation}

with the angular dependence given by the vector $\mbf b$

\begin{eqnarray}
b_x& = & \frac{-H_y+i H_x \frac{H_z}{|\mbf H|}}{2|\mbf H| sin\Theta} \\
b_y& = & \frac{H_x+i H_y \frac{H_z}{|\mbf H|}}{2|\mbf H| sin\Theta} \\
b_z& = & \frac{H_x^2+H_y^2}{2i|\mbf H|^2 sin\Theta}=\frac{-isin\Theta}{2} \\
&&sin\Theta=\frac{\sqrt{H_x^2+H_y^2}}{|\mbf H|}
\end{eqnarray}

and with the sum $R_J$ given by

\begin{equation}
R_J=(x-1)\sum_{m=-J}^{J-1}(J+m+1)(J-m)x^m=\frac{2Jx^{-J}+(2J+2)x(x^{J}-x^{-J})-2Jx^{J+2}}{(1-x)^2}
\end{equation}

The module given in {\prg /examples/gdni2b2c/1ion\_mod/brillouin.c} 
can be compiled
by typing make in the directory {\prg /examples/gdni2b2c/1ion\_mod/} thus using the
{\prg  /examples/gdni2b2c/1ion\_mod/makefile}. It 
evaluates the Brillouin function and thus calculates the thermal expectation 
value $<>_T$ of the vector $\mbf J$. 
The moment ${\mbf M}=g_J \mu_B <{\mbf J}>_T$
is returned to the mcphas\index{mcphas} program:

\newpage
\section{An Isotropic Spin - single ion module}\label{brillouin}

A magnetic ion often can be described by a pure spin moment (transition
metal ions, $^{3+}$Gd, $^{2+}$Eu). In this description the magnetic moment
${\mbf M}=g_J \mu_B <{\mbf J}>_T$ is calculated by the Brillouin function. 

\begin{equation}
<\mbf J>=\frac{\mbf H}{|\mbf H|} B_{J}(x=g_J \mu_B |\mbf H|/kT)
\end{equation}

with

\begin{equation}
B_J(x)=\frac{\sum_{m=-J}^{J} m x^m}{\sum_{m=-J}^{J}  x^m}=
\frac{J(x^{J+2}-x^{-J})+(J+1)x(x^{-J}-x^J)}{(1-x)(x^{-J}-x^{J+1})}
\end{equation}

Note that the partition sum $Z$ is given by $Z=\sum_{m=-J}^{J}  x^m=\frac{x^{J+1}-x^{-J}}{x-1}$.
This is all what is needed to calculate the mean field ground state with 
module {\prg mcphas}.

For the calculation of the dynamical properties using module {\prg mcdisp\index{mcdisp}}
the single ion susceptibility of for a free magnetic ion is needed. In principle
there are (J-1) transitions with non vanishing matrix elements of
the momentum operator $\mbf J$ with all having the same
 excitation energy $\Delta=g_J \mu_B H$.
 These different transitions can
be combined into one effective inelastic excitation. The corresponding
transition matrix to be used in equation~(\ref{mmatrix}) of the MF-RPA formalism
is given by 

\begin{equation}
M_{\alpha\beta}=\frac{-b_{\alpha}b_{\beta}R_J}{Z}
\end{equation}

with the angular dependence given by the vector $\mbf b$

\begin{eqnarray}
b_x& = & \frac{-H_y+i H_x \frac{H_z}{|\mbf H|}}{2|\mbf H| sin\Theta} \\
b_y& = & \frac{H_x+i H_y \frac{H_z}{|\mbf H|}}{2|\mbf H| sin\Theta} \\
b_z& = & \frac{H_x^2+H_y^2}{2i|\mbf H|^2 sin\Theta}=\frac{-isin\Theta}{2} \\
&&sin\Theta=\frac{\sqrt{H_x^2+H_y^2}}{|\mbf H|}
\end{eqnarray}

and with the sum $R_J$ given by

\begin{equation}
R_J=(x-1)\sum_{m=-J}^{J-1}(J+m+1)(J-m)x^m=\frac{2Jx^{-J}+(2J+2)x(x^{J}-x^{-J})-2Jx^{J+2}}{(1-x)^2}
\end{equation}

The module given in {\prg /examples/gdni2b2c/1ion\_mod/brillouin.c} 
can be compiled
by typing make in the directory {\prg /examples/gdni2b2c/1ion\_mod/} thus using the
{\prg  /examples/gdni2b2c/1ion\_mod/makefile}. It 
evaluates the Brillouin function and thus calculates the thermal expectation 
value $<>_T$ of the vector $\mbf J$. 
The moment ${\mbf M}=g_J \mu_B <{\mbf J}>_T$
is returned to the mcphas\index{mcphas} program:

\newpage
\section{An Isotropic Spin - single ion module}\label{brillouin}

A magnetic ion often can be described by a pure spin moment (transition
metal ions, $^{3+}$Gd, $^{2+}$Eu). In this description the magnetic moment
${\mbf M}=g_J \mu_B <{\mbf J}>_T$ is calculated by the Brillouin function. 

\begin{equation}
<\mbf J>=\frac{\mbf H}{|\mbf H|} B_{J}(x=g_J \mu_B |\mbf H|/kT)
\end{equation}

with

\begin{equation}
B_J(x)=\frac{\sum_{m=-J}^{J} m x^m}{\sum_{m=-J}^{J}  x^m}=
\frac{J(x^{J+2}-x^{-J})+(J+1)x(x^{-J}-x^J)}{(1-x)(x^{-J}-x^{J+1})}
\end{equation}

Note that the partition sum $Z$ is given by $Z=\sum_{m=-J}^{J}  x^m=\frac{x^{J+1}-x^{-J}}{x-1}$.
This is all what is needed to calculate the mean field ground state with 
module {\prg mcphas}.

For the calculation of the dynamical properties using module {\prg mcdisp\index{mcdisp}}
the single ion susceptibility of for a free magnetic ion is needed. In principle
there are (J-1) transitions with non vanishing matrix elements of
the momentum operator $\mbf J$ with all having the same
 excitation energy $\Delta=g_J \mu_B H$.
 These different transitions can
be combined into one effective inelastic excitation. The corresponding
transition matrix to be used in equation~(\ref{mmatrix}) of the MF-RPA formalism
is given by 

\begin{equation}
M_{\alpha\beta}=\frac{-b_{\alpha}b_{\beta}R_J}{Z}
\end{equation}

with the angular dependence given by the vector $\mbf b$

\begin{eqnarray}
b_x& = & \frac{-H_y+i H_x \frac{H_z}{|\mbf H|}}{2|\mbf H| sin\Theta} \\
b_y& = & \frac{H_x+i H_y \frac{H_z}{|\mbf H|}}{2|\mbf H| sin\Theta} \\
b_z& = & \frac{H_x^2+H_y^2}{2i|\mbf H|^2 sin\Theta}=\frac{-isin\Theta}{2} \\
&&sin\Theta=\frac{\sqrt{H_x^2+H_y^2}}{|\mbf H|}
\end{eqnarray}

and with the sum $R_J$ given by

\begin{equation}
R_J=(x-1)\sum_{m=-J}^{J-1}(J+m+1)(J-m)x^m=\frac{2Jx^{-J}+(2J+2)x(x^{J}-x^{-J})-2Jx^{J+2}}{(1-x)^2}
\end{equation}

The module given in {\prg /examples/gdni2b2c/1ion\_mod/brillouin.c} 
can be compiled
by typing make in the directory {\prg /examples/gdni2b2c/1ion\_mod/} thus using the
{\prg  /examples/gdni2b2c/1ion\_mod/makefile}. It 
evaluates the Brillouin function and thus calculates the thermal expectation 
value $<>_T$ of the vector $\mbf J$. 
The moment ${\mbf M}=g_J \mu_B <{\mbf J}>_T$
is returned to the mcphas\index{mcphas} program:

\input{../examples/gdni2b2c/1ion_mod/brillouin.c}



