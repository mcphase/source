\section{{\prg Searchspace and Simannfit} - Fitting Experimental Data}\label{simannfit}

In order to fit experimental data a very general concept of fitting program 
has been developed. 
{\prg Perl} has to be installed on the system in order to run
this fitting program.

The program {\prg searchspace\index{searchspace}} is used to cover the parameter space and test
different regions of this space for local minima 
of a specific function {\prg sta(par1, par2 par3 ...)} of
these parameters.. It may be used to 
find good starting values for the program {\prg simannfit\index{simannfit}}.

The program {\prg simannfit\index{simannfit}} uses a simulated annealing algorithm described
in~\cite{kirkpatrick83-671}.
This algorithm does a random walk through the parameter space 
with the aim to fit the parameters to  a minimum of the function {\prg sta}.
  Starting at a set of parameter values (par1, par2, par3) the algorithm
changes these values by a randomly chosen step width (the maximum step width
is specified at the beginning) to (par1', par2', par3').
 The function {\prg sta(par1', par2', par3', ...)} is calculated at
for these new parameter values. The step is accepted if 
$exp([sta(par')-sta(par)]/T)<{\rm a random number out of} [0,1]$.
Otherwise it is rejected. The statistical temperature $T$ involved in this
condition is lowered from step to step as well as the maximum step width allowed.

In the following we describe how to set up the program packages of {\prg McPhase} to
solve a specific fitting problem.
As an example we refer to the fitting of the magnetic 
excitations of CeCu$_2$ in {\prg mcphas/examples/cecu2a/fit}. 

\begin{figure}[hb]\label{safit}
\includegraphics[angle=0,width=0.7\columnwidth]{figsrc/simannfit.eps}
\caption{Layout of the fitting modules.}
\end{figure}


\subsection{Setting up parameter files for fitting}

In order to fit, it is necessary to tell the fitting programs {\prg searchspace\index{searchspace}} and
 {\prg simannfit\index{simannfit}}, which 
parameters in some input files should be varied. Take for instance
the input file {\prg mcphas.jjj}: if you want to vary a parameter, create a file 
{\prg mcphas.jjj.forfit} which is an exact copy of {\prg mcphas.jjj} - in 
this file you substitute the value of the parameter which you want to vary  
 by {\em par}name[value,min,max,xx,stp].
 Here value denotes the starting value, min and max the parameter range, xx an arbitrary
 number (this field will be used by the algorithm to calculate the error of the
 parameter). Stp denotes the maximum step width for this parameter.
Here follows an example:

\begin{enumerate}
\item Original input file {\prg /mcphas/examples/ndba2cu3o7/mcdiff.in}:
{\footnotesize
\begin{verbatim}
...
#
#{atom-file} da[a]  db[b]    dc[c]     dr1[r1]  dr2[r2]  dr3[r3]   <Ma>     <Mb>  <Mc> [mb]
#{corresponding effective fields gjmbHeff [meV]- if passed to mcdiff only these are used 
# for caculation (not the magnetic moments)}
{Nd3p.sipf} 0.50000 0.50000 0.50000 0.50000 0.50000 0.50000 +0.00000 +0.00000 -1.36910  
           corresponding effective fields gjmbHeff [meV]--> +0.00000 +0.00000 -0.07530
{Nd3p.sipf} 0.50000 0.50000 1.50000 0.50000 0.50000 1.50000 +0.00000 +0.00000 +1.36910  
           corresponding effective fields gjmbHeff [meV]--> +0.00000 +0.00000 +0.07530
...
\end{verbatim}
}

	
\item Modified input file{\prg /mcphas/examples/dba2cu3o7/mcdiff.in.forfit}: 
{\footnotesize
\begin{verbatim}
...
#{atom-file} da[a]  db[b]    dc[c]     dr1[r1]  dr2[r2]  dr3[r3]   <Ma>     <Mb>  <Mc> [mb]
#{corresponding effective fields gjmbHeff [meV]- if passed to mcdiff only these are used 
# for caculation (not the magnetic moments)}
{Nd3p.sipf} 0.5000 0.5000 0.5000 0.5000 0.5000 0.5000  0.0000 0.0000 -1.3691  
  corresponding effective fields gjmbHeff [meV]-->     0.0000 parHb %%@
[-2.815596e-003,-2.000000e-002,1.000000e-002,2.762505e-005,4.851325e-003] parHc %%@
[-5.328829e-002,-9.000000e-002,-5.000000e-002,1.614710e-005,4.896285e-003]
{Nd3p.sipf} 0.5000 0.5000 1.5000 0.5000 0.5000 1.5000  0.0000 0.0000 1.3691  
   corresponding effective fields gjmbHeff [meV]-->    0.0000 function[-1*parHb] function[-1*parHc]
...
\end{verbatim}
}

\end{enumerate}

As can be seen constraints on the parameters can be implemented by
substituting the parameter value by {\em function[par}name{\em ]}.
This means for instance, that instead of {\em function[par2a]} the algorithm
uses the parameter {\em par2a} specified at another place of the input file.

\begin{figure}[hb]%h=here, t=top, b=bottom, p=separate figure page
\begin{center}\leavevmode
\includegraphics[angle=0, width=0.8\textwidth]{figsrc/searchspace.eps}
\end{center}\label{searchspace}
\caption{Coverage of a two-dimensional parameter space by program {\prg searchspace\index{searchspace}}}
\end{figure}

{\bf What does searchspace/simannfit do with these parameter files ?}

The fitting programs {\prg searchspace\index{searchspace}} and 
{\prg simannfit\index{simannfit}} take the modified file and generate from it an
input file in the original format (using several different
sets of parameters) and call the program
{\prg calcsta} for each parameter set.

\subsection{Telling the fitting program, how to calculate the standard deviation {\prg sta} which should be minimised}


The user must provide a program {\prg calcsta}  (under windows 
it is {\prg calcsta.bat}), which calculates the standard deviation
of some experimental data from model calculated data. This program can
be a bash script such as {\prg /mcphas/examples/cecu2a/fit/calcsta} or just a
very short program such as:


\begin{verbatim}
#!/usr/bin/perl
# do a mcphase simulation. The measured magnetisation data must
# be in directory ./fit/mcphas.fum column 11 12 13.
# the program mcphas\index{mcphas} will compare measured data with calculated
# data and generate an output such as sta=14.3112, which
# contains the standard deviation
system ("mcphas");


\end{verbatim}

... in general the output of this batch (to stdout) must contain a line
such as {\em sta = 4122.32}, which contains the calculated standard
deviation which will be minimized by the fitting procedure.

Important notes:
\begin{enumerate}
\item
Before starting a fit it is necessary to test the program {\prg calcsta} by typing
{\prg ./calcsta}. The calculation should run and there must appear an output
the line {\prg sta= ...} on the screen. If this works, a fit may be started.
\item the programs {\prg McPhas} and {\prg McDisp} always try to
calculate a standard deviation and write such {\em sta = ...} statement
to stdout.  So it is possible to use these programs to create the desired output.
\item
 Alternatively (or in addition)
any other program can be called in {\prg calcsta} and calculate the 
standard deviation {\prg sta}. Useful  programs are {\prg chi2}\index{chi2},
{\prg rpvalue}\index{rpvalue}.
\item Several ($N$) occurrencies of {\em sta = } $\delta_i^2$  contribute additively to
the variance $s^2=1/N\sum_i \delta_i^2$, which is minimized in the process.
\item
if {\em sta=} is followed by two numbers $\delta_i^2$ and ${\rm err}^2$, the first is interpreted as the 
 $\delta_i^2=$ deviation$^2$ from an experimental data point (which is to be fitted).
The second number ${\rm err}^2$  is interpreted as the squared errorbar of this 
datapoint. If ${\rm err}^2$ is not given, the program minimizes the variance
{\em sta}$=s^2=1/N\sum_i \delta_i^2$.
If ${\rm err}^2$ is given, the program minimizes $\chi^2=1/N\sum_i \delta_i^2/{\rm err}^2$.
The error is used at the end of the fitting to calculate
a covariance matrix, which can be used to estimate correlations
and errors of the fitting parameters: $F_{ij}=\partial \delta_i / {\rm err}_i\partial {\rm par}_j$,
the covariance matrix is then ${\rm cov}=\chi^2(F^TF)^{-1}$, it's diagonal elements correspond
to the squared error of the fitting parameters par$_j$ and the off diagonal elements to
the correlation among these fitting parameters.  Useful to generate
the required {\em sta= ...} statements is the program {\prg chi2}\index{chi2}.
\item
 note, that the program {\prg simannfit\index{simannfit}} calls the user written program
{\prg calcsta} with a number as argument, e.g. as {\prg calcsta 21.3}. This number
(i.e. 21.3 in our example) denotes a maximum value of the standard deviation. The
user written program may use this number and stop, when in the summation process
to obtain standard deviation {\prg sta} reaches a value larger than this maximum
number. Using this option is advisable in complex fitting problems in order to
optimise the time of the fitting procedure. Note, that the module {\prg mcphas}
can be used with the option {\prg -stamax 21.3}, which leads to a stop when the value of 
the standard deviation 21.3 is reached. 
\end{enumerate}

\subsection{Starting a parameter space search}

If the program {\prg calcsta} has been set up, the parameter space may be
searched by the program {\prg searchspace\index{searchspace}}. It is started
by the command (in the example you should go to the directory {\prg /mcphas/examples/cecu2a/fit}) 
:

\begin{quote}
 {\prg searchspace 0 mcphas.jjj [and possible other input parameter files]}.
\end{quote}


Note that the user written program {\prg calcsta} has to be in the
directory where {\prg searchspace\index{searchspace}} is started from.
The ''0'' in the command means that the search is started for the first time.
The program generates an output file {\prg searchspace.0} with a list
of parameter sets. This list may be shortened to stop searching 
some regions of the parameter space and subsequently the program may be started
with

\begin{quote}
 {\prg searchspace 1 mcphas.jjj [and possible other input parameter files]}.
\end{quote}


Now the values stored in {\prg searchspace.0} are read and 
the parameter space is searched in more detail and a longer list 
of parameters is stored as {\prg searchspace.1}. This may be again
edited to drop some regions of the parameter space and {\prg searchspace\index{searchspace}}
may be started with {\prg searchspace 2 ...} and so on ....

The program creates new parameter sets from an old set by incrementing/decrementing
each of the parameters subsequently by a small step. In the first level this step
is just equal to a quarter of the parameter range. Going from one level to
 the next this step is halved. Figure ~\ref{searchspace} shows how a two dimensional
 parameter space is covered by this procedure. If all neighbouring points of
 a given point show a larger value of {\prg sta}, then the set is recorded, because it might be
 near a minimum of {\prg sta}. The parameter values are stored in the file
 {\prg results/searchspace.searchspace-level.localminima}.

\subsection{analysing the results of {\prg searchspace}}

The results of a parameter search can be analysed by starting searchspace for example as

\begin{quote}
{\prg searchspace -13 results/searchspace.2.localminima mcphas.jjj [end possible other input files]}.
\end{quote}

In this example the parameter set (line) number 13 is taken from the
 file {\prg results/searchspace.2.localminima}. Obviously this parameter set is the local minimum number 13
from a previous searchspace run with searchspace level 2. Using this input
 searchspace updates {\prg mcphas.jjj} and
other parameter files with this parameter set.
The standard deviation can be recalculated now easily by typing {\prg calcsta}, thereby all
output files are updated and the quality of this parmaeter set can be inspected, e.g. by
making graphs etc.
Note that also the files {\prg *.forfit} are updated with
these parameters (this is useful for a later use with {\prg simannfit}).

\subsection{Starting a fit}

If the program {\prg calcsta} has been set up, the fitting is started
by the command (in the example you should go to the directory {\prg /mcphas/examples/cecu2a/fit}) 
:

\begin{quote}
 {\prg simannfit 10 [-t 1000][-s 100000] [-n 200] mcphas.jjj [and possible other input parameter files]}.
\end{quote}

\begin{itemize}
\item
The ''10'' in the command means that the initial statistical
''temperature'' of the algorithm
is set to 10.  
\item option -l 0.2 ... sets limit - program will end if $sta<0.2$
\item Option -t sets time limit until program end (in seconds).
\item Option -s gives maximal number of iteration steps to be done.
\item Option -n allows to store every 200 steps the parametersets in a table in  {\prg results/simannfit.0}.
\item Option -p 30 probes solution, i.e. runs simannfit , stepwidths are not decreased during
                   fitting, if parameter set $parn$ is found with $sta<=sta of initial parameters$
                   then all sets par in {\prg results/simannfit.1} are scanned and
                   distance $d=\sum_i^N (par_i-parn_i)^2/stepwidth_i^2$ is calculated. If 
                   $d>N $(=number of parameters) for
                   for all sets par, then $parn$ is appended to the list in {\prg results/simannfit.1}
                   up to 20 parameters are appended to this list, after that the program stops
                   - in this way a series of equally good solutions can be explored.
\item option -w 1.4     before starting simannfit, multiply all stepwidths by factor 1.4
\end{itemize}
Note that the user written program {\prg calcsta} has to be in the
directory where {\prg simannfit\index{simannfit}} is started from.

The program generates for each parameter a histogram file showing the number of occurrences
of a certain parameter value in solutions, where sta decreased.

\subsection{Stopping a fit/parameter space search}

The fitting procedure should be stopped by changing the program {\prg calcsta}, so that it
writes {\prg sta=0} to the standard output (stdout in Linux). Then the fitting
procedure is stopped. The last value of parameters fitted can be found in the
input parameter files.

\subsection{Fitting is an art: some general remarks}

Fitting data is in most cases not a straightforward procedure and requires
a lot of experience and intuition. Even the fitting of a Gaussian to some
experimental data is sometimes difficult and the available algorithms fail,
 if the initial
stepwidths or starting values are not chosen with some care.
This rule holds even more for complex non linear fitting in a large
parameter space. In many cases a simple theoretical model is used
and it is {\bf not} clear at the beginning if this model is able
to describe the data at all. 
So how can reasonable starting parameters be found ? What value
for the statistical temperature has to be chosen ?

One possible way to attack this problem is to monitor the status of a fit
by viewing online its quality when fitting. The value of {\em sta} which is put onto
the screen together with the standard deviation is only a very rudimentary information.
In most cases it is necessary to view the data which is to be fitted graphically and
see how the calculated data matches. It is also advisable to monitor the
standard deviation and how it changes with time. After starting the
fit it should the be possible to judge, if stepwidths should be modified or if
the parameter range was chosen too small or too large etc.

To view online the contents of a file which contains data in column format
the program {\prg display} can be used. To start the display\index{display} of
the file {\em data.dat} with x-axis data in column 1 and y-axis data
in column 2 just type (java has to be installed in order to use this):

display\index{display} 1 2 data.dat

In order to send this data display\index{display} to the background type under Linux:

display\index{display} 1 2 data.dat \&

and under windows:

start display\index{display} 1 2 data.dat

\subsection{Crystal Field  Example: Fitting Point Charges to Inverse Susceptibility Data}

A frequent wish is to fit crystal field parameters to experimental data. This task can
be simple in high symmetry compounds and  very complex due to large number of parameters
and complex experimental data. Here we demonstrate on an example, how a fit can take into account
the high temperature expansion of magnetic susceptibility in order to reduce the
number of free parameters in an orthorhombic system. It is well known, that the 
crystal field parameters $B_2^0$ and $B_2^2$ can be determined from the high
 temperature expansion of susceptiblity. Thus there is no need to fit these parameter, this
is simple. 

If it is desirable to determine point charges and not the crystal field parameters, the
number of point charges can be reduced if $B_2^0$ and $B_2^2$ are fixed. For example
in a ternary system TmNiC$_2$ it is sufficient to fit the point charge on Tm. We quote here
the correponding {\prg calcsta.forfit}  for linux users, which can be used by a command
such as {\prg simannfit 0.1 calcsta}:

\begin{verbatim}
# <h1> TmNiC2 </h1>
# fitting point charges using standard model with CF-coupling trying to model chi 

# we do not vary charge on Ni and C - these can be obtained from the
# B20 and B22 (PRB Roman) as determined by high temperature expansion of 
# the single crystal susceptibility
# we vary point charge on Tm
export CTm=parTm [3.000000e+00,-3.000000e+00,3.000000e+00,3.086877e-04,1.901175e-01] 

# a cutoff radius
export RR=parR [3.599614e+00,3.000000e+00,6.000000e+00,2.497474e-04,4.472897e-02] 

# screening factors for l=2,4,6 dependent linear on the distance 
# e.g. SF20 is screening factor for l=2 at NN distance r=+2.5978 A
# and SF2 is screening factor for l=2 at r=R, the cutoff radius
export SF20=parSF20 [5.727796e-01,4.000000e-01,1.000000e+00,3.727098e-04,3.432656e-02]
export SF2=parSF2 [1.000000e+00,0.000000e+00,1.000000e+00,3.105923e-04,7.189467e-02]
export SF40=parSF40 [5.787451e-01,4.000000e-01,1.000000e+00,2.256921e-04,5.359141e-02]
export SF4=parSF4 [8.647244e-01,0.000000e+00,1.000000e+00,2.185745e-04,8.422583e-02]
export SF60=parSF60 [9.401158e-01,4.000000e-01,1.000000e+00,5.031542e-04,3.946531e-02]
export SF6=parSF6 [9.348959e-01,0.000000e+00,1.000000e+00,7.718764e-04,8.030474e-02]

setvalue 1 2 $SF20 sf.r
setvalue 1 3 $SF40 sf.r
setvalue 1 4 $SF60 sf.r
setvalue 2 2 $SF2 sf.r
setvalue 2 3 $SF4 sf.r
setvalue 2 4 $SF6 sf.r
setvalue 2 1 $RR sf.r

# determine coefficient of pointcharge
#  to yield correct B20 und B22
# Hint: the CF parameters are linear in the point charges $CTm,$CC,$CNi
# (compare CF theory slide in tutorial)
# B20=NI20*$CNi+C20*$CC+TM20*$CTm=0.2199 meV Roman etal PRB 107, 125137 (2023) 
# B22=NI22*$CNi+C22*$CC+TM22*$CTm=-0.328 meV

cif2mcphas -pc $RR -sp -nm Ni -ch Tm=0,Ni=0,C=1 TmNiC2_400K.cif -sf sf.r
. getvariable B20 Tm1_1.sipf
export C20=$(echo $MCPHASE_GETVARIABLE_VALUE | sed s/not/0.0/ )
. getvariable B22 Tm1_1.sipf
export C22=$(echo $MCPHASE_GETVARIABLE_VALUE | sed s/not/0.0/ )


cif2mcphas -pc $RR -sp -nm Ni -ch Tm=0,Ni=1,C=0 TmNiC2_400K.cif -sf sf.r
. getvariable B20 Tm1_1.sipf
export NI20=$(echo $MCPHASE_GETVARIABLE_VALUE | sed s/not/0.0/ )
. getvariable B22 Tm1_1.sipf
export NI22=$(echo $MCPHASE_GETVARIABLE_VALUE | sed s/not/0.0/ )

cif2mcphas -pc $RR -sp -nm Ni -ch Tm=1,Ni=0,C=0 TmNiC2_400K.cif -sf sf.r
. getvariable B20 Tm1_1.sipf
export TM20=$(echo $MCPHASE_GETVARIABLE_VALUE | sed s/not/0.0/ )
. getvariable B22 Tm1_1.sipf
export TM22=$(echo $MCPHASE_GETVARIABLE_VALUE | sed s/not/0.0/ )

# now we have to solve the equations for the CF paramters B20 and B22
# B20=NI20*$CNi+C20*$CC+TM20*$CTm=0.2199 meV Roman etal PRB 107, 125137 (2023) 
# B22=NI22*$CNi+C22*$CC+TM22*$CTm=-0.328 meV

# resolve the system of 2 linear equations to get $CC and $CNi 
# the sed command is to get rid of exponential expressions such as 3e-4
# which the in line calculator of linux, bc does not understand 
export A=$(echo "(0.2199-($TM20)*($CTm))/($NI20)" | sed s/e[+]*/*10^/g | bc -l)
export B=$(echo "(-0.328-($TM22)*($CTm))/($NI22)" | sed s/e[+]*/*10^/g | bc -l)

# $Cni=A-C20*$CC/NI20
# $Cni+C22*$CC/NI22=B

# A-B=C20*$CC/NI20-C22*$CC/NI22=$CC*(C20/NI20-C22/NI22)=$CC*C
export C=$(echo "($C20)/($NI20)-($C22)/($NI22)" | sed s/e[+]*/*10^/g | bc -l)
export CC=$(echo "(($A)-($B))/($C)" | sed s/e[+]*/*10^/g | bc -l)
export CNi=$(echo "$A-($C20)*($CC)/($NI20)" | sed s/e[+]*/*10^/g | bc -l)

# use cif2mcphas to calculate point charge model
cif2mcphas -pc $RR -sp -nm Ni -ch Tm=$CTm,Ni=$CNi,C=$CC ../TmNiC2_400K.cif -sf sf.r

# now the point charge determined crystal field parameters are in
# Tm1_1.sipf  and this can be used to calculate chi and fit it to data ...

for T in $(seq 1 30 400); do singleion -M  -r Tm1_1.sipf $T 1 0 0 0 0 0; done > results/chianew.clc 2>/dev/null
for T in $(seq 1 30 400); do singleion -M  -r Tm1_1.sipf $T 0 1 0 0 0 0; done > results/chibnew.clc 2>/dev/null
for T in $(seq 1 30 400); do singleion -M  -r Tm1_1.sipf $T 0 0 1 0 0 0; done > results/chicnew.clc 2>/dev/null

delcomments -fromline 20 results/chianew.clc results/chibnew.clc results/chicnew.clc >/dev/null
fillcol 6 1/c9 results/chianew.clc
fillcol 7 1/c10 results/chibnew.clc
fillcol 8 1/c11 results/chicnew.clc
factcol 1 0 results/chianew.clc results/chibnew.clc results/chicnew.clc
# put susceptibility how it should be to column 1 in chi*new.clc
add 2 1 results/chianew.clc 1 2 exp/chia.dat
add 2 1 results/chibnew.clc 1 2 exp/chib.dat
add 2 1 results/chicnew.clc 1 2 exp/chic.dat

# chi2 is used to calculate sta 
chi2 6 1 3 results/chianew.clc
chi2 7 1 4 results/chibnew.clc
chi2 8 1 5 results/chicnew.clc

# this is just to print out what parameters have been used
echo cif2mcphas -pc $RR -sp -nm Ni -ch Tm=$CTm,Ni=$CNi,C=$CC TmNiC2_400K.cif -sf sf.r

\end{verbatim}

For windows users the following file {\prg calcsta.bat.forfit} will do a similar job using
a fitting command such as {\prg simannfit 0.1 calcsta.bat }:

\begin{verbatim}
REM <h1> TmNiC2 </h1>
REM fitting point charges using standard model with CF-coupling trying to model chi 

REM we do not vary charge on Ni and C - these can be obtained from the
REM B20 and B22 (PRB Roman) as determined by high temperature expansion of 
REM the single crystal susceptibility
REM we vary point charge on Tm
call set CTm=parTm [3.000000e+00,-3.000000e+00,3.000000e+00,3.086877e-04,1.901175e-01] 

REM a cutoff radius
call set RR=parR [3.599614e+00,3.000000e+00,6.000000e+00,2.497474e-04,4.472897e-02] 

REM screening factors for l=2,4,6 dependent linear on the distance 
REM e.g. SF20 is screening factor for l=2 at NN distance r=+2.5978 A
REM and SF2 is screening factor for l=2 at r=R, the cutoff radius
call set SF20=parSF20 [5.727796e-01,4.000000e-01,1.000000e+00,3.727098e-04,3.432656e-02]
call set SF2=parSF2 [1.000000e+00,0.000000e+00,1.000000e+00,3.105923e-04,7.189467e-02]
call set SF40=parSF40 [5.787451e-01,4.000000e-01,1.000000e+00,2.256921e-04,5.359141e-02]
call set SF4=parSF4 [8.647244e-01,0.000000e+00,1.000000e+00,2.185745e-04,8.422583e-02]
call set SF60=parSF60 [9.401158e-01,4.000000e-01,1.000000e+00,5.031542e-04,3.946531e-02]
call set SF6=parSF6 [9.348959e-01,0.000000e+00,1.000000e+00,7.718764e-04,8.030474e-02]

call setvalue 1 2 %SF20% sf.r
call setvalue 1 3 %SF40% sf.r
call setvalue 1 4 %SF60% sf.r
call setvalue 2 2 %SF2% sf.r
call setvalue 2 3 %SF4% sf.r
call setvalue 2 4 %SF6% sf.r
call setvalue 2 1 %RR% sf.r

REM determine coefficient of pointcharge
REM  to yield correct B20 und B22
REM Hint: the CF parameters are linear in the point charges $CTm,$CC,$CNi
REM (compare CF theory slide in tutorial)
REM B20=NI20*CNi+C20*CC+TM20*CTm=0.2199 meV Roman etal PRB 107, 125137 (2023) 
REM B22=NI22*CNi+C22*CC+TM22*CTm=-0.328 meV

call cif2mcphas -pc %RR% -sp -nm Ni -ch Tm=0,Ni=0,C=1 TmNiC2_400K.cif -sf sf.r
call getvariable B20 Tm1_1.sipf
call export C20=%MCPHASE_GETVARIABLE_VALUE%
if C20=="not" set C20=0
call getvariable B22 Tm1_1.sipf
call export C22=%MCPHASE_GETVARIABLE_VALUE%
if C22=="not" set C22=0


call cif2mcphas -pc %RR% -sp -nm Ni -ch Tm=0,Ni=1,C=0 TmNiC2_400K.cif -sf sf.r
call getvariable B20 Tm1_1.sipf
call export NI20=%MCPHASE_GETVARIABLE_VALUE%
if NI20=="not" set NI20=0
call getvariable B22 Tm1_1.sipf
call export NI22=%MCPHASE_GETVARIABLE_VALUE%
if NI22=="not" set NI22=0

call cif2mcphas -pc %RR% -sp -nm Ni -ch Tm=1,Ni=0,C=0 TmNiC2_400K.cif -sf sf.r
call getvariable B20 Tm1_1.sipf
call export TM20=%MCPHASE_GETVARIABLE_VALUE%
if TM20=="not" set TM20=0
call getvariable B22 Tm1_1.sipf
call export TM22=%MCPHASE_GETVARIABLE_VALUE%
if TM22=="not" set TM22=0

REM now we have to solve the equations for the CF paramters B20 and B22
REM B20=NI20*$CNi+C20*$CC+TM20*$CTm=0.2199 meV Roman etal PRB 107, 125137 (2023) 
REM B22=NI22*$CNi+C22*$CC+TM22*$CTm=-0.328 meV

REM resolve the system of 2 linear equations to get $CC and $CNi 
call export A=(0.2199-(%TM20%)*(%CTm%))/(%NI20%)
call export B=(-0.328-(%TM22%)*(%CTm%))/(%NI22%)

REM Cni=A-C20*CC/NI20
REM Cni+C22*CC/NI22=B

REM A-B=C20*CC/NI20-C22*CC/NI22=CC*(C20/NI20-C22/NI22)=CC*C
call export C=(%C20%)/(%NI20%)-(%C22%)/(%NI22%)
call export CC=((%A%)-(%B%))/(%C%)
call export CNi=%A%-(%C20%)*(%CC%)/(%NI20%)

REM use cif2mcphas to calculate point charge model
call cif2mcphas -pc %RR% -sp -nm Ni -ch Tm=%CTm%,Ni=%CNi%,C=%CC% ../TmNiC2_400K.cif -sf sf.r

REM now the point charge determined crystal field parameters are in
REM Tm1_1.sipf  and this can be used to calculate chi and fit it to data ...

call perl -e "for($T=1;$T<400;$T+=30){system('singleion '.$T.' 1 0 0 0 0 0');}" > results\chianew.clc
call perl -e "for($T=1;$T<400;$T+=30){system('singleion '.$T.' 0 1 0 0 0 0');}" > results\chibnew.clc
call perl -e "for($T=1;$T<400;$T+=30){system('singleion '.$T.' 0 0 1 0 0 0');}" > results\chicnew.clc

call delcomments -fromline 20 results/chianew.clc results/chibnew.clc results/chicnew.clc 
call fillcol 6 1/c9 results/chianew.clc
call fillcol 7 1/c10 results/chibnew.clc
call fillcol 8 1/c11 results/chicnew.clc
call factcol 1 0 results/chianew.clc results/chibnew.clc results/chicnew.clc
REM put susceptibility how it should be to column 1 in chi*new.clc
call add 2 1 results/chianew.clc 1 2 exp/chia.dat
call add 2 1 results/chibnew.clc 1 2 exp/chib.dat
call add 2 1 results/chicnew.clc 1 2 exp/chic.dat

REM chi2 is used to calculate sta 
call chi2 6 1 3 results/chianew.clc
call chi2 7 1 4 results/chibnew.clc
call chi2 8 1 5 results/chicnew.clc

REM this is just to print out what parameters have been used
echo cif2mcphas -pc %RR% -sp -nm Ni -ch Tm=%CTm%,Ni=%CNi%,C=%CC% TmNiC2_400K.cif -sf sf.r

\end{verbatim}



The screening file {\prg sf.r} is:

\begin{verbatim}
# screeningfile 
# r[A]   sf(l=2)   sf(l=4)  sf(l=6)
+2.5978 0.56 0.53 0.48 
3.589 0.6747 0.44 0.677
\end{verbatim}







\vspace{1cm}
{\em Exercises:}
\begin{itemize}
\item 
Start the fit example given for the fitting of the magnetic 
excitations of CeCu$_2$ in {\prg mcphas/examples/cecu2a/fit}.
First copy the file {\prg mcphas.jjj.sav} to {\prg mcphas.jjj}
Then start the fit by typing {\prg simannfit 10 mcphas.jjj}.
\item
Stop the fit by editing {\prg calcsta} as described above.
\item 
Copy {\prg mcphas.jjj.fit} to {\prg mcphas.jjj}  and type
{\prg calcsta}. What is the standard-deviation ? 
\item 
In order to fit  crystal field parameters to transition
energies of neutron intensities write a module
{\prg calcsta} for CeCu$_2$ which uses only the program
cfield to calculate the transition matrix elements
and compare them to experimental data.
The section~\ref{addprog} contains some further programs which
might help. 
\end{itemize}
