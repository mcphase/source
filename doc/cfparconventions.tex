\section{Crystal Field and Parameter Conventions}\label{cfparconventions}

For historical reasons, crystal field parameters (effectively the radial matrix elements of the crystal field
interactions) may be expressed in two different "normalisation", which we shall call \emph{Stevens} and 
\emph{Wybourne}. Stevens~\cite{stevens52-209,hutchings64-227} initially expressed the radial parts of the crystal field interaction
in terms of angular momentum operators $J_x$, $J_y$, $J_z$. He did this by taking the Cartesian expressions for 
the tesseral harmonic functions (see Appendix~\ref{tesseral}), and replacing all instances of the coordinates
$x$, $y$, and $z$ with $J_x$, $J_y$ and $J_z$ and allowing for the commutation relations of the angular
momentum operators, but without considering the normalisation condition of these functions and hence are missing
the prefactors before the square brackets in the expressions in Appendix~\ref{tesseral}. We denote these
prefactors $p_{lm}$. The Stevens crystal field Hamiltonian is thus

\[
\mathcal{H}_{\mathrm{cf}} = \sum_{l,m} A_{lm} \langle r^l \rangle \langle J || \theta_l || J \rangle \hat{O}_l^m (J)
\]

\noindent where the product $A_{lm} \langle r^l \rangle$ is commonly taken in the literature as the crystal
field parameter, because the factorisation into an intrinsic parameter $A_{lm}$ and the expectation value of
the radial wavefunction $\langle r^l \rangle$ is derived from the point charge model and is not generally
valid. Alternatively, the product $B_l^m = A_{lm} \langle r^l \rangle \langle J || \theta_l || J \rangle$ is
also commonly used, particularly in the neutron scattering literature.

Wybourne~\cite{wybourne65} and subsequent co-authors on the other hand chose to use the tensor operators 
$\hat{C}_{lm}$ which transform in the same way as the functions $C_{lm}(\theta,\phi) = \sqrt{4\pi / (2l+1)}
Y_{lm}(\theta,\phi)$, where $Y_{lm}(\theta,\phi)$ are the spherical harmonic functions, to describe the crystal
field. Thus the angular-dependent part of the crystal field matrix elements used by Wybourne differed from
that of Stevens by the factor $\lambda_{k0} = p_{k0} \sqrt{4\pi / (2l+1)}$ and
 $\lambda_{lm} = p_{lm} \sqrt{8\pi / (2l+1)}$ for $q\neq 0$. The crystal field Hamiltonian used by Wybourne is
thus (in our notation)

\[
\mathcal{H}_{\mathrm{cf}} = \sum_{l,m} D_l^{m} \hat{C}_{lm}
\]

The disadvantage of the Wybourne approach is that one requires imaginary crystal field parameters, because the
tensor operators $\hat{C}_{lm}$ are not Hermitian. In {\prg McPhase}, we have instead chosen to use slightly
different tensor operators $\hat{T}_{lm}$, which are the Hermitian combinations of the $\hat{C}_{lm}$,

\[
  \hat{T}_{l0} = \hat{C}_{l0}, \qquad \hat{T}_{l,\pm|m|} = \sqrt{\pm 1} \left[ \hat{C}_{l,-|m|} \pm (-1)^{|m|} \hat{C}_{l,|m|} \right]
\]

\noindent giving the Hamiltonian

\[
\mathcal{H}_{\mathrm{cf}} = \sum_{l,m} L_l^m \hat{T}_{lm}
\]

Our $L_l^m$ parameters therefore have the same normalisation as the Wybourne parameters but will be real.
%, with the imaginary parts expressed by the $q<0$ components of the $\hat{T}_{lm}$ operator.

In summary:
\begin{enumerate}
 \item  the various types of crystal field parameters are given in 
table~\ref{cfpartypes}.
\item crystal field parameters may have different units, in {\prg
McPhase} we use meV.
\item crystal field coordinate systems used in literature may be 
different from the convention used by single ion modules {\prg so1ion,ic1ion,icf1ion},
which is the crystal field axes $xyz$ are such that $x||a,y||b,z||c$ in case
of orthogonal lattices and in case of non-orthogonal axes 
the convention is $y||\vec b$, $z||(\vec a \times \vec b)$ and $x$ perpendicular to $y$ and $z$.
\end{enumerate}

\begin{table}
\begin{center} \begin{tabular}{@{\extracolsep{\fill}}l|r|l|l} \hline
Literature	&  Parameter  & Relation to $L_l^m$                                                   & Normalisation \\ 
		&  name in    &                                                                       &\\
		&{\prg McPhase}&                                                                       &\\ \hline
$A_l^m$ in Hutchings 1964~\cite{hutchings64-227}
		&  Alm        & =           $ \lambda_{lm} L_l^m /\langle r^l \rangle$               & Stevens \\
$A_l^m$ in Kassman 1970~\cite{kassman70-4118}  & & & \\
$A_l^m$ in Abragam 1970~\cite{abragam70-1}  & & & \\
\hline
$B_l^m$ in Hutchings 1964~\cite{hutchings64-227}
		&  Blm        & =           $ \left \{ \begin{array}{lr}  \lambda_{lm} L_l^m \langle J || \theta_l || J \rangle & {\rm  for rare earth} \\
                                                                          \lambda_{lm} L_l^m \langle L || \theta_l || L \rangle & {\rm  for trans metals}
		                                       \end{array} \right .$                                  & Stevens \\                               

$B_l^m$ in Jensen 1991~\cite{jensen91-1}  & & & \\
$B^l_m$ in Mulak 2000~\cite{mulak00-1}  & & & \\
$B^l_m$ in FOCUS  & & & \\
$B^l_m$ in AMOS   & & & \\
\hline
$V_l^m$ in Elliott 1953~\cite{elliott53-387}
		&             & =           $ \lambda_{lm} L_l^m $                                   & Stevens \\
\hline
		&  Wlm        & = half($m$) $ \lambda_{lm} L_l^m /\langle r^l \rangle$               & Stevens \\
\hline
		&  Vlm        & = half($m$) $ \lambda_{lm} L_l^m \langle J || \theta_l || J \rangle$ & Stevens \\
\hline
		&  Llm        & =           $              L_l^m $                                   & Wybourne \\
\hline
$B_m^l$ in Wybourne 1965~\cite{wybourne65}\footnote{Note that table 6-1 and equ 6-7 are not correct in this reference.}
		&  Dlm        & =          $ \left \{ \begin{array}{lr} L_l^0 & m=0 \\
                                                                (-1)^m(L_l^m-iL_l^{-m}) & m >0 \\ 
                                                                L_l^{-m}+iL_l^{m}      &m<0    \\                    
		                               
		                              \end{array}  \right . $                                & Wybourne  \\ 
$B_m^l$ in Kassman 1970~\cite{kassman70-4118}  & & & \\
$B_{lm}$ in Mulak 2000~\cite{mulak00-1}  & & & \\
$Ak(),CAk()$ in XTLS  & & & \\
\hline
$B_m^l$ in Newman 2000~\cite{newman00-1}
                &             & =           $(-1)^m L_l^m$                                            & Wybourne \\
\hline
\end{tabular} 
\caption{Different Crystal Field Parameter Notation Schemes in Literature.
Here half($m$) = $\frac{1}{2}$ when $m\neq 0$ and half($m$)=1 when $m=0$.
Note that {\prg lm} should be replaced by numbers - e.g. an example of Alm is {\prg A20}, and that
the operator equivalent factor $\langle J || \theta_l || J \rangle$~\cite{hutchings64-227} should be $\langle L || \theta_l || L
\rangle$ for an $L$-manifold which is the default for $d$-electron ions~\cite{abragam70-1}. The constants $\lambda_{lm}$
are given in table~\ref{tab:wytostev}.}\label{cfpartypes}
\end{center}
\end{table}

\begin{table}[h] \renewcommand{\arraystretch}{1.3}
  \begin{center}
   %\begin{tabular*}{10cm}{@{\extracolset{\fill}}|c|c|c|}
   %  \multicolumn{3}{|c|}{$k$~~~$q$~~~$\lambda_{lm}$} \\ \hline
   %  $k$  &  $q$  &  $\lambda_{lm}$ \\ \hline \hline
   %  $2$  &  $0$  &  $\frac{1}{2}$  \\
   %  $2$  &  $1$  &  $-\sqrt{6}$  \\
   %  $2$  &  $2$  &  $\frac{1}{2}\sqrt{6}$  \\
   %  $4$  &  $0$  &  $\frac{1}{8}$  \\
   %  $4$  &  $1$  &  $-\frac{1}{2}\sqrt{5}$  \\
   %  $4$  &  $2$  &  $\frac{1}{4}\sqrt{10}$  \\
   %  $4$  &  $3$  &  $-\frac{1}{2}\sqrt{35}$  \\
   %  $4$  &  $4$  &  $\frac{1}{8}\sqrt{70}$  \\
   %  $6$  &  $0$  &  $\frac{1}{16}$  \\
   %  $6$  &  $1$  &  $-\frac{1}{8}\sqrt{42}$  \\
   %  $6$  &  $2$  &  $\frac{1}{16}\sqrt{105}$  \\
   %  $6$  &  $3$  &  $-\frac{1}{8}\sqrt{105}$  \\
   %  $6$  &  $4$  &  $\frac{3}{16}\sqrt{14}$  \\
   %  $6$  &  $5$  &  $-\frac{3}{8}\sqrt{77}$  \\
   %  $6$  &  $6$  &  $\frac{1}{16}\sqrt{231}$  \\ \hline
    \begin{tabular}{c|ccccccc}
     %\hline
     %\backslashbox{$k$}{$|q|$} & $0$ & $1$ & $2$ & $3$ & $4$ & $5$ & $6$ \\
          & \multicolumn{7}{|c}{$|m|$} \\  \cline{2-8}
      $l$ & $0$ & $1$ & $2$ & $3$ & $4$ & $5$ & $6$ \\
      \hline \hline
      $2$ & $\frac{1}{2}$  & $\sqrt{6}$             & $\frac{1}{2}\sqrt{6}$  & & & & \\
      $4$ & $\frac{1}{8}$  & $\frac{1}{2}\sqrt{5}$  & $\frac{1}{4}\sqrt{10}$   & $\frac{1}{2}\sqrt{35}$ & $\frac{1}{8}\sqrt{70}$   & & \\
      $6$ & $\frac{1}{16}$ & $\frac{1}{8}\sqrt{42}$ & $\frac{1}{16}\sqrt{105}$ & $\frac{1}{8}\sqrt{105}$ & $\frac{3}{16}\sqrt{14}$ &
            $\frac{3}{8}\sqrt{77}$ & $\frac{1}{16}\sqrt{231}$ \\
      \hline
    \end{tabular}
    \caption{\emph{Ratios $\lambda_{lm}$ of the Stevens to the real valued
 Wybourne normalised parameters}.
$\lambda_{lm} \equiv B_l^m/(\theta_l^mL_l^m)= A_{lm}\langle r^l \rangle / L_l^m$. 
$ \lambda_{lm}$ are related to the prefactors in the 
tesseral harmonics (see appendix~\ref{tesseral}) $p_{lm}$ by 
$\lambda_{l0} =\sqrt{\frac{4\pi}{2l+1}}|p_{l0}|$ and
$\lambda_{lm} =\sqrt{\frac{8\pi}{2l+1}}|p_{lm}|$ for $m \neq 0$~\cite[note that Newman on p.30, equ (2.7) defines
his real valued Wybourne 
parameters $B_m^l({\rm Newman})\equiv (-1)^m L_l^m$]{newman00-1}.} \label{tab:wytostev}
  \end{center}
\end{table}
\clearpage