

%\documentclass[a4paper,12pt]{article}
%\usepackage{epsfig}

%\begin{document}
%\pagenumbering{arabic}
%\thispagestyle{empty}
%\setcounter{page}{1}

%{\Large
%\centerline
\section{Inelastic neutron-scattering from RE ions in a crystal field
%}
%\centerline{
including damping effects due to the exchange interaction with conduction
elecrons}
%}

\bigskip
This is an extension  of the theory published by Klaus W. Becker, Peter Fulde and
Joachim Keller in Z. Physik B
28,9-18, 1977 
"Line width of crystal-field excitations in metallic rare-earth systems"
and an  introduction to the computer program for the calculation of  the neutron 
scattering cross section. The computer program is written by J. Keller,
University of Regensburg.

\medskip

\noindent
Here we present a brief outline of the theoretical concepts to calculate the
dynamical susceptibility of the Re ions and the scattering cross section.  

The neutron-scattering cross section is related to the dynamic susceptibility
of the  RE ions 
$$
\chi_{\alpha\beta}(t)={i\over \hbar} \Theta(t)\langle [J^\dagger_\alpha(t), J_\beta(0)]\rangle 
$$
whose Fourier-Laplace transform
$$
\chi_{\alpha,\beta}(z)=\int_{-\infty}^{+\infty} dt e^{izt}\chi_{\alpha\beta}(t), \quad z=\omega
+i\delta 
$$
determines the inelastic neutron scattering crossection   
(Stephen W. Lovesey; "Theory of neutron 
scattering from condensed matter"
Vol 2, equ. 11,144).
$$
{d^2\sigma \over d\Omega d E'}=  {k' \over
k}({r_0\over 2}g_J F(Q))^2{1\over \pi }
\sum_{\alpha\beta}(\delta_{\alpha\beta} 
- \tilde Q_\alpha \tilde Q_\beta)
{ \chi{"}_{\alpha,\beta}(\omega)\over 1-e^{-\beta \hbar \omega}} 
$$
Here $k$ and
$ k'$ denote the  wave number of the neutron before and after the
scattering. $\vec Q = \vec k - \vec k'$ is the scattering wave vector,
$\tilde Q = \vec Q/\vert\vec Q\vert$. 
$r_0= -0.54 \cdot 10^{-12}$ cm is the basic scattering length, $g_J$ is the Land\'e factor, 
$F(Q)$ the atomic form factor of the
rare earth ion.
  
\bigskip
\noindent
Formal evaluation of the dynamic and static susceptiblity.

The dynamic spin-susceptibilities are correlation functions of the form
$$
\chi_{i,k}(t)=i \Theta(t) \langle [A_i^\dagger(t),A_k(0)]\rangle
$$ 
where $A(t)$ is a Heisenberg operator
$$
A(t)= \exp(iHt)A\exp(-iHt)
$$
Introducing a Liouville operator (acting on operators of dynamical
variables) by ${\cal L}A = [H,A]$  the Heisenberg operator can also be
written formally as  
$$
A(t)= \exp(i{\cal L}t) A
$$
With help of this definition the dynamical susceptibility $\chi_{i,k}$ of 
two variables $A_i, A_k$ can be written as
$$
\chi_{i,k}(t)=i \Theta(t) \langle [A_i^\dagger(t),A_k(0)]\rangle
$$ 
and their Laplace transform
$$
\chi_{i,k}(z)=i\int_0^\infty dt e^{izt} \langle [A_i^\dagger(t),A_k(0)]\rangle
$$ 
With help of the Liouvillian these quantities can be written as
$$
\chi_{i,k}(t)=i \Theta(t) \langle [A_i^\dagger,A_k\exp^{-i{\cal L}t}\rangle
$$
and their Laplace transform
$$
\chi_{i,k}(z)= -\langle [A_i^\dagger,{1\over {z-\cal
L}}A_k(0)]\rangle
$$ 

The static isothermal susceptibilities can also formally be calculated with help of
the Liouvillian.   
$$
\chi_{i,k}(0) = \int_0^\beta d \lambda \langle e^{\lambda H}
A_i^\dagger e^{-\lambda H}  A_k\rangle
 = \int_0^\beta d \lambda \langle (e^{\lambda {\cal L}}
A_i^\dagger)   A_k\rangle
$$


The static susceptibilities are used to define a scalar product between  the
dynamical variables:
$$
(A_i \vert A_k) =  {1\over \beta }\int_0^\beta d\lambda \langle
(e^{\lambda{\cal L}}A_i^\dagger)A_k\rangle ={1\over \beta} \chi_{ik}(0)
$$
It fulfills the axioms of a scalar product and  furthermore it has the important property
$$
({\cal L}A_i\vert A_k)=(A_i\vert {\cal L}A_k)={1\over \beta}\langle
[A_i^\dagger,A_k]\rangle
$$
With help of this relation the dynamical susceptibility can be
expressed as
$$
\chi_{i,k}(z)= -\beta (A_i\vert {{\cal L}\over {z-\cal L}} A_k)
$$
and finally as 
$$
\chi_{ik}(z)=\chi_{ik}(0)-z\beta (A_i\vert {1\over {z-\cal L}}A_k)  
$$
The second term is the so-called relaxation function 
$$
\Phi_{ik}(z)=(A_i \vert {1 \over {z-\cal L}}A_k)
$$

\bigskip







\noindent
The model:



We calculate the spin susceptibility of a RE ion in the presence of exchange
interaction with conduction electrons. The system is described by the
Hamiltonian
$$
H=H_{cf}+H_{el}+H_{el,cf} 
$$
The first part is the cf-Hamiltonian of the spin-system: 
$$ 
H_{cf}= \sum_n E_n K_{nn}, \quad K_{nm}= \vert n\rangle \langle m\vert
$$  
written in terms of the crystal field eigenstates $\vert n\rangle$. 
The second part is the Hamiltonian of the conduction electrons
$$
H_{el}=\sum_{k\alpha}\epsilon_kc^\dagger_{k\alpha}c_{k\alpha}
$$
The third part is the 
 interaction between local moments and  the conduction electrons
$$
H_{el,cf}= - J_{ex}\vec J \cdot \vec \sigma, \quad \vec \sigma = \sum_{k\alpha
Q\beta}\vec \sigma_{\alpha\beta}c^\dagger_{k\alpha}c_{k+Q\beta}, \quad \vec
J=\sum_{n,m}\vec J_{n,m}K_{nm}.
$$
We assume, that the energies $E_n$ and the eigenstates $\vert n\rangle$
expressed by angular momentum eigenstates are known.

\bigskip
\noindent
Definition of dynamical variables

In our case we use as dynamical variable the standard-basis operators
$$
A_\mu= K_{\mu}
$$
 describing a  transition $\mu= [nm]$ between CEF levels $m$ and $n$. 
In the absence of the interaction with conduction electrons
$$
{\cal L}A_\mu = (E_n-E_m)A_\mu
$$
In order to
get the spin suceptibility we have to multiply the final expressions by the
spin-matrixelements:
$$ 
\chi_{\alpha \beta}   
$$

The idea of the projection formalism to calculate the dynamical
susceptibility of a variable $A$ is to project this variable  onto a closed
set of dynamical variables $A_i$ and to solve approximately the coupled
equations between these variables. For this purpose a projector is defined
by
$$
{\cal P} A= \sum_{\nu \mu}A_\nu P^{-1}_{\nu \mu}(A_\mu\vert A) \quad
P_{\nu\mu}=(A_\nu\vert A_\mu)
$$ 
where $ P^{-1}_{\nu \mu}=[P^{-1}]_{\nu \mu}$ is the ${\nu\mu}$-component of the inverse matrix of
$P$. 

For the  resolvent operator of the relaxation function
$$
{\cal F}(z)= {1\over {z-\cal L}}, \quad ({z-\cal L}){\cal F}(z)=1 
$$
one obtains the exact equation
$$
({\cal P}(z-{\cal P}{\cal L}{\cal P} - {\cal P} {\cal M}(z) {\cal P}){\cal P} {\cal
F}(z) {\cal P}= {\cal P}
$$
with the memory function
$$
{\cal M}(z)={\cal PLQ}{1\over z-{\cal QLQ} }{\cal QLP}
$$ 
where ${\cal Q}=1-{\cal P}$. In components
$$
\Phi_{\nu\mu}(z)= (A_\nu\vert {1\over z-{\cal L}} A_\mu)
$$
$$
\sum_\lambda \Bigl(z\delta_{\nu\lambda}-\sum_\kappa\bigl[L_{\nu\kappa}+
M_{\nu\kappa}(z)\bigr]P^{-1}_{\kappa\lambda}\Bigr)\Phi_{\lambda\mu}(z)
=P_{\nu \mu}
$$
with
$$
L_{\nu\mu}=(A_\nu\vert {\cal L}A_\mu)
$$
and the memory function 
$$
M_{\nu\mu}(z)=(A_\nu\vert{\cal M}(z)A_\mu)
$$

Now we apply the formalism to the coupled spin-electron system and restrict
ourselves to the lowest order contributions of the spin electron
interaction. As dynamical variables we choose a decomposition of the  
original spin-variable:
$$
J^\alpha=\sum_{n_1,n_2}J^\alpha_{n_2,n_1}K_{n_2,n_1}=\sum_\nu J^\alpha_\nu A_\nu, 
\quad 
A_\nu= K_{n_1n_2}   
$$
where $\nu$ denotes a transition $n_2 \gets n_1$ performed with the
standard-basis operator $\vert n_2\rangle\langle n_1\vert$. 

In lowest (zeroth) order in the el-cf interaction  
$$
{\cal L}A_\nu = (E_{n_2}-E_{n_1)}A_\nu
$$
and  the scalar product is diagonal in lowest order in the transition
operators,
$$
P_{\nu\mu}=(A_\nu\vert A_\mu)\simeq\delta_{\nu
\mu}P_\nu, \quad P_\nu=(A_\nu\vert
A_\nu)={p(n_1)-p(n_2)\over \beta (E_{n2}-E_{n_1})}
$$
where $p(n)=\exp(-\beta E_n)/Z$ is the thermal occupation number. For the
frequency term we then get 
$$
L_{\nu\mu}=\delta_{\nu\mu}(A_\nu\vert A_\nu) (E_{n_2}-E_{n_1} )
+O(J_{ex}^2)
$$
Neglecting the second-order energy corrections in the following we obtain
the equation for the relaxation function 
$$
\Phi_{\nu\mu}(z)=\bigl[\Omega^{-1}\bigr]_{\nu\mu}(z)
P_\mu, \quad
\Omega_{\nu\mu}(z)=(z-E_\nu)\delta_{\nu\mu} -
M_{\nu\mu}(z)[P^{-1}]_\mu, \quad
E_\nu = E_{n_2}- E_{n_1}
$$
and it remains to calculate the memoryfunction containing the relaxation
processes. 

In lowest order in the electron-spin interaction ${\cal QL}A_\nu$ can be replaced by ${\cal L}_{el,cf}A_\nu$.
Then we get for the memory function
$$
M_{\nu \mu}(z)= ({\cal L}_{el,cf}A_\nu \vert{1\over z- {\cal
L}_0}{\cal L}_{el,cf}A_\mu)=
M_{n_2n_1,m_2m_1}(z)
$$
with
$$
M_{n_2n_1,m_2m_1}(z)=({\cal L}_{el,cf} K_{n_2n_1}\vert{1\over z-{\cal
L}_0}{\cal L}_{el,cf} K_{m_2m_1})
$$

Now
$$
{\cal L}_{el,cf} K_{n_2n_1} = J_{ex}\sum_t \vec \sigma(\vec  J_{n_1t}
K_{n_2t} - \vec
J_{tn_2}K_{tn_1})
$$
with
$$
\vec \sigma = \sum_{k\alpha, k+Q\beta}\vec \sigma_{\alpha\beta}
c^\dagger_{k\alpha}c_{k+Q\beta}
$$
With help of the symmetry properties 
$$
( \sigma^i K_{nm}\vert {1\over z- {\cal L}_0}\sigma^j K_{n'm'})=
\delta_{ij}\delta_{nn'}\delta_{mm'}G_{nm}(z)
$$
with
$$
G_{nm}(z)=( \sigma^i K_{nm}\vert {1\over  {z-\cal L}_0}\sigma^i K_{nm})
$$
we obtain
\eqnarray
M_{n_2n_1,m_2m_1}(z)=J_{ex}^2\sum_i\Bigl[&
\delta_{n_2m_2}\sum_tJ^i_{m_1t}J^i_{tn_1}G_{n_2t}
+ \delta_{n_1m_1}\sum_tJ^i_{
n_2t}J^i_{tm_2}G_{tn_1}\nonumber \\
&-J^i_{m_1n_1}J^i_{n_2m_2}G_{n_2m_1}
-J^i_{n_2m_2}J^i_{m_1n_1}G_{m_2n_1}\Bigr]\nonumber
\endeqnarray


In order to calculate the relaxation functions $G_{n,m}(z)$
we use the general relation between relaxation function and dynamic
susceptibility
$$
\chi(z)= \chi(0)-\beta z \Phi(z)
$$
and calculate instead the corresponding susceptibility (using tr
$\sigma^i\sigma^i)=2$):
\eqnarray
G_{nm}(z)&= {2\over \beta \omega}\sum_{k,k+Q} \langle \Bigl[K_{mn}
 c^\dagger_{k+Q}c_{k},(z - E_n +
E_m  -\epsilon_k+\epsilon_{k+Q})^{-1} K_{nm}
c^\dagger_{k}c_{k+Q}\Bigr]\rangle \nonumber\\
&=  {2\over \beta \omega
}\sum_{k,Q}(f_{k+Q}(1-f_{k})p_m-f_{k}(1-f_{k+Q})p_n)(
z-E_n+E_m-\epsilon_{k}+\epsilon_{k+Q})^{-1}\nonumber
\endeqnarray
We are interested in the imaginary part describing the relaxation processes:
$$
Im G_{nm}(\omega+i\delta)= - {2\pi \over \beta \omega
}\sum_{k,Q}\Bigl(f_{k+Q}(1-f_{k})p_m-f_{k}(1-f_{k+Q})p_n \Bigr)
 \delta(\omega -E_n+E_m-\epsilon_{k}+\epsilon_{k+Q})
$$
Writing $ \rho=\omega - \omega_{nm}$ and $\omega_{nm}=E_n-E_m$  we obtain
$$
Im G_{nm}(\omega+i\delta)= - {2\pi N^2(0)\over \beta \omega} \int d\epsilon  
 (f(\epsilon)(1-f(\epsilon+\rho) p_m
- f(\epsilon+\rho)(1-f(\epsilon))p_n)
$$
For the integrals we get
\eqnarray
\int d\epsilon f(\epsilon)(1-f(\epsilon+\rho)=&\int d\epsilon
\exp(\beta(\epsilon+\rho))/
(1+\exp(\beta\epsilon))(1+\exp(\beta(\epsilon+\rho))\nonumber\\
=& 
(\omega-\omega_{nm})\exp(\beta(\omega-\omega_{nm}))/
(-1+\exp(\beta(\omega-\omega_{nm})\nonumber
\endeqnarray
\eqnarray
\int d\epsilon f(\epsilon+\rho )(1-f(\epsilon)=&\int d\epsilon
\exp(\beta(\epsilon)/
(1+\exp(\beta\epsilon))(1+\exp(\beta(\epsilon+\rho))\nonumber\\
=& 
(\omega-\omega_{nm})/
(-1+\exp(\beta(\omega-\omega_{nm}))\nonumber
\endeqnarray
This makes
$$
Im G_{nm}=-{2\pi N^2(0)\over \beta \omega}(\omega -\omega_{nm}) {1-\exp(-\beta
\omega)\over
1-\exp[(\omega_{nm}-\omega)\beta]}p_m
$$
which has to be used to calculate the imaginary part of the memory function.
Writing 
$$
F_{nm}(\omega )= {1\over \beta \omega}(\omega -\omega_{nm}) {1-\exp(-\beta
\omega)\over
1-\exp[(\omega_{nm}-\omega)\beta]}p_m
$$
which also be written in symmetrized form as
$$
F_{nm}(\omega )= {\sqrt{p_np_m}\over \beta}{(\omega -\omega_{nm})\over
\omega} {\exp(\beta \omega)/2) - \exp(-\beta \omega)/2)\over
\exp(\beta (\omega-\omega_{nm})/2) - \exp(-\beta (\omega -\omega_{nm})/2)}
$$
we obtain with $g=J_{ex}N(0)$
\eqnarray
M_{n_2n_1,m_2m_1}(\omega) =- i 2\pi g^2
\sum_i\Bigl[&
\delta_{n_2m_2}\sum_tJ^i_{m_1t}J^i_{tn_1}F_{n_2t}
+ \delta_{n_1m_1}\sum_tJ^i_{
n_2t}J^i_{tm_2}F_{tn_1}\nonumber\\
&-J^i_{m_1n_1}J^i_{n_2m_2}F_{n_2m_1}
-J^i_{n_2m_2}J^i_{m_1n_1}F_{m_2n_1}\Bigr]
\nonumber
\endeqnarray
from which we get the memory function matrix in the space of dynamical variables
$$
M_{\nu \mu}(\omega)= M_{n_2n_1,m_2m_1}(\omega)
$$

\bigskip
\noindent


Summary:
For the neutron scattering cross section we need the function 
$Im \chi^{\alpha\beta}(\omega+i\delta)/(1-\exp(-\beta\omega)$, where
$\chi^{\alpha\beta}(z)$ is the frequency dependent part of the dynamic
susceptibility $\chi^{\alpha\beta}(z)$ for spin components  $J^\alpha$,$J^\beta$, which is
related to the corresponding relaxation function $\Phi^{\alpha,\beta}$ by
$$
\chi^{\alpha\beta}(z) = \chi^{\alpha\beta}(0) - \beta z \Phi^{\alpha\beta}(z)  
$$
For the full dynamical susceptibility we need the static suseptibility
$ \chi^{\alpha\beta}(0) $
which  in lowest order in the exchange interaction is given by 
$$
\chi^{\alpha\beta}(0)  = \sum_\nu  (J^\alpha_\nu)^\dagger \beta P_\nu J^\beta_\nu
$$
The above relaxation function is calculated with help of the Mori-Zwanzig
projection formalism by 
$$
\Phi^{\alpha\beta}(z)=\sum_{\mu\nu}
(J^\alpha_\nu)^*\Phi_{\nu\mu}(z)J^\beta_\mu
$$
where $\nu$ denotes a transition from $n_1$ to $n_2$ between crystal field
levels of the magnetic ion. The partial relaxation functions are obtained by 
solving the matrix
equation
$$
\Phi_{\nu\mu}(z)= [\Omega^{-1}]_{\nu\mu}P_\mu      
$$
with
$$
\Omega_{\nu\mu}(z)= (z-\omega_\nu)\delta_{\nu\mu}  -M_{\nu\mu}(z)/P_\mu
$$
where $\omega_\nu =E_{n_2}- E_{n_1}$ is the energy difference of cf-levels.

Only terms in  lowest  order in the el-ion interaction are kept. We neglect
frequency shifts due to the electron-ion interaction. 
Then the  memory function is purely
imaginary (with a negative  sign).

Note that compared to our paper BFK, Z.Physik B28, 9-18, 1977 we have used here 
a different sign-convention.

For numerical reasons it is more convenient to calculate the relaxation
function in the following way:
$$
\Phi_{\nu\mu}(z)= P_\nu[\bar\Omega^{-1}]_{\nu\mu}P_\mu      
$$
with
$$
\bar \Omega_{\nu\mu}(z)= P_\nu(z-\omega_\nu)\delta_{\nu\mu}  - M_{\nu\mu}(z)
$$

From the relaxation function we get for the dynamic scattering cross section 
$$
{d^2\sigma \over d\Omega d E'} =  {k' \over
k}S(\vec Q,\omega)
$$
with
$$
S(\vec Q, \omega)=({r_0\over 2}g_J F(\kappa))^2{1\over \pi }
\sum_{\alpha\beta}(\delta_{\alpha\beta} 
- \hat Q_\alpha \hat Q_\beta)
Im \Phi^{\alpha,\beta}(\omega){-\beta \omega \over 1-e^{-\beta \hbar \omega}} 
$$
Here the scattering function depends only on the scattering vector
$\vec Q= \vec k - \vec k'$ and the energy loss $(\hbar)\omega =E(k)-E(k')$
Note that in our formulas $\omega$ contains a factor $\hbar$ and is the
energy loss. If we want to have meV as energy unit and Kelvin as temperature
unit, we have to write $\beta= 11.6/T$.
   
For the analysis of polarised neutron scattering the different
spin-components $S^{\alpha\beta}(\vec Q,\omega)$ of $S$ are needed. 
These are defined by
$$
S(\vec Q,\omega)= \sum_{\alpha\beta}(\delta_{\alpha\beta} 
- \hat Q_\alpha \hat Q_\beta)S^{\alpha\beta}(\vec Q,\omega)
$$
with
$$
S^{\alpha\beta}(\vec Q,\omega)=
Im \chi^{\alpha\beta}/(1-e^{-\beta \hbar \omega})=
Im \Phi^{\alpha,\beta}(\omega){-\beta \omega \over 1-e^{-\beta \hbar \omega}} 
$$
The complex dynamic susceptbility is calculated from
$$
\chi^{\alpha\beta}(\omega)= \chi^{\alpha\beta}(0)-\beta \omega
\Phi^{\alpha\beta}(\omega)= \sum_{\mu\nu}\beta (J^\alpha_\mu)^*(P_{\mu\nu}-\omega
\Phi_{\mu\nu}(\omega ))J^\beta_\nu
$$
where the static susceptibilities $\beta P_{\mu\nu}$ are diagonal in our
approximation.  

\bigskip
\newpage

\vfill
\eject
\parindent=0pt
{\large Description of the program:}

The program calculates the dynamical susceptibility
and the neutron scattering cross-section of single RE ions in the presence
of crystal fields and Landau damping due to the exchange interaction with 
conduction electrons.

It needs the following input-files (not all are needed for all tasks)

1. A file containing the information about the RE ion: Type of ion, number of
CF-levels, energy eigenvalues and eigenstates. The date are extracted from
the input-file by reading the information contained in lines starting with
$\#!$ or blanks, see the attached example. 

2. File with the formfactor data for RE ion

3. File with a list of ($\vec Q,\omega$)-values, for which the calculation
shall be performed

4. A parameter-file containing  the names of the files with the
formfactor, the  table with the ($\vec Q,\omega$)-values, the energy range, 
scattering direction etc., see the attached example.

5. The value of the coupling constant $g=j_{ex}N(0)$, the temperature,  
the mode of calculation, the form of the out-put, the name of the file with
the CEF-data, the  name of the parameter file are provided by the
commandline, which is used to start the program. 

   
The program consists of a number of modules and subroutines which are
briefly described in the following: 
 
1. Modules CommonData, MatrixElements, FormfactorPreparation

These modules contain definitions of global variables and arrays 
used in the program and in different  subroutines. 
FormfactorPreparation also contains the
subroutine FormfactorTransformation which transforms an input-file with
formfactor data into a file with formfactor values for equidistant Q-values. 
and the function Formfac to calculate the formfactor at
arbitrary Q-values.
  
2. Subroutine ReadData

Subroutine to read-in data needed to calculate the dynamical susceptibility
and the neutron scattering cross-section.

It reads the commandline, containing the coupling $g=j_{ex} N(0)$,  
the temperature  $T$ (in Kelvin), mode of calculation (see below), form of out-put, 
name of the
file with RE data, name of the parameter-file (containing also the name of
the file with the formfactor data). The information about the RE ion is
transferred into a workfile cefworkfile.dat for inspection and use in the
following runs. The data contained in the parameter-file are stored in the
file bfkdata.dat. The latter two have to be given only in the first run. If
they are left-out  in the following runs, the are assumed    to be
unchanged.

3. Subroutine Matrixelements

a) Calculates angular momentum matrices jjx, jjy, jjz for the crystal-field eigenstates
(2-dim arrays, dimension Ns x Ns). The three directional components are also
stored in the 3-dimensional array jjj(3,Ns,Ns). 

b) Calculates Boltzmann-factors $p(n)$. A cut-off in the exponent $\beta
E(n)$ is introduced such that Boltzmann factors with large negative
exponents
are set equal to zero. 

c) Defines  a set of transitions  $\nu$  between states
n1 and n2, stored in two  1-dim
arrays v1($\nu$), v2($\nu$). If both Boltzmann factors of the two states
involved are zero, this transition is eliminated from the set of allowed
transitions.  

d) Calculates static suscepibilities $P(\mu)$ for the standard basis operators
$K_{n,m}$ for the allowed transitions.  

e) All these reults are stored in a  file bfkmatrix.dat for examination, if
something goes wrong.
 

4. MatrixInversionSubroutine

adapted from Numerical Recipes, to be used for the inversion of the complex
matrix $\Omega_{\nu\mu}$. Called by 5. 

5. Subroutine Relmatrix

Calculates the matrix relaxation function $\Phi_{\mu,\nu}(\omega)$ for the set 
of dynamical variables
obtained from  the standard basis operators for a given energy (freqency)
$\omega$.

6. Subroutine Suscepcomponents 

Calculates the different components of the
dynamical susceptibility
$$
\chi^{\alpha,\beta}(\omega) = \beta\sum_{\mu \nu}(J^\alpha)_\mu)^*
\bigl[P_\mu\delta_{\mu\nu}-\omega \Phi_{\mu\nu}(\omega)\bigr]J^\beta_\nu
$$ 
and 
$$
Im \chi^{\alpha\beta}(\omega)/(1-\exp(-\beta\omega))
$$
7. Function Scatfunction(Q,$\omega$)

Calculates
$$
S(\vec Q,\omega)= \sum_{\alpha\beta}\bigl(\delta_{\alpha\beta} - \tilde
Q^\alpha\tilde Q^\beta\bigr) Im \chi^{\alpha\beta}/(1-\exp(-\beta\omega))
$$


8. Subroutine OutputResults 

Here the results for the dynamical susceptibility, the scattering function
and the differential neutron scattering cross section for different
scattering geometries are calculated, and the results written into files 
bfkm.res for different scattering-modes m=0-6, which are written into the  
subdirectory /results. Depending on the value of ms=1,2 the new results
over-write the previews results or append.  

Depending on the number m=0-6 (3. entry of the commandline) the following
results are calculated. 

mode=0: all nine components $\chi^{\alpha\beta}(\omega)$  of the complex 
dynamic susceptibility are calculated for $Npoint$ equidistant energies  $\omega$ 
between $emin$ and $emax$. 

mode=1: the diagonal components of $Im \chi^{\alpha\alpha}(\omega)/\tanh{\beta\omega/2}$
are calculated and the frequency integral is compared with the sum-rule 
$$
\sum_\alpha {1\over\pi}\int d\omega {Im
\chi^{\alpha\alpha}\over \tanh(\beta\omega/2)} 
= J(J+1) 
$$
The sum-rule sometimes is not very well fulfilled, since within this
approximation the Landau damping does not fall-off fast enough at large
energies

mode=2: The scattering function 
$$
S(\vec Q,\omega)=\sum_{\alpha,\beta} (\delta_{\alpha, \beta}- \tilde Q_\alpha \tilde
Q_\beta){ \chi''_{\alpha \beta}(\omega)\over 1-\exp(-\beta\omega)}
$$ 
is calculated for a given set of values for  
energy loss $\omega$ and scattering vector $\vec Q$ contained in a file
specified in the parameterfile. 

mode=3: The 9 different components of the scattering-function 
$$
(r_0g_JF(\vec Q)/2)^2{1\over \pi} S^{\alpha\beta}(\vec Q,\omega), \quad 
S^{\alpha\beta}(\vec Q,\omega)= {Im \chi^{\alpha,\beta}(\omega)\over
1-exp(-\beta\omega)}
$$
are  calculated. 

mode= 4-6: the neutron scattering cross section
$$
{d^2\sigma \over d\Omega d E'} =  {k' \over
k}S(\vec Q,\omega)
$$
with
$$
S(\vec Q, \omega)=({r_0\over 2}g F(Q))^2{1\over \pi }
\sum_{\alpha\beta}(\delta_{\alpha\beta} 
- \hat Q_\alpha \hat Q_\beta)
{Im \chi^{\alpha,\beta}(\omega)\over 1-exp(-\beta \omega)} 
$$

is  calculated for different scattering geometries:
In mode 4 the  
direction of the wave vector $\vec k$ and the energy $E=k^2/2m$ of the
incident 
beam is fixed. The direction of the scattering wave vector $\vec Q$ is fixed, 
but the 
length of $\vec Q$ is variable. The wave vector of the scattered
particles is  $\vec k'=\vec k-\vec Q$, their energy is $E'=k'^2/2m$ and
the energy loss is $\omega =E-E'$. In mode 5 the direction  of the wave vectors 
$\vec k$ and $\vec k'$ of the incoming and scattered beam
are fixed, while the energy $E'$ of the scattered beam is variable. In mode
6
the energy $E$ of the incident particles is variable 
and the energy $E'=k^2/2m$ of the scattered particles  fixed.  

\newpage
{\large How to run the program:}

\bigskip
The translated program is started with a command-line 
like 

bfk 0.1 10 0 1 prlevels.cef paramfile.par


with  the following
structure:

name of the program: bfk;  coupling constant g; temperature T (in K); type of
calculation: mode =1...6; type of output: mst=1 overwrite, mst=2 append new
results;  name of file with RE ion data; name of parameter file.

\medskip

The last two entries can be skipped in later runs, if they are not changed.

The mode number mode = 1 \dots 6
refers to the subject of calculation. The output- number 
mst=1,2 refers to the type of output-storage.

\medskip
The file with RE data should have the  form produced by sol1on (see the attached example). 
The
lines starting with numbers or blanks contain information, the lines
starting with \# are commentaries, the lines starting with  \#! also carry
information.

\medskip 
The parameterfile contains additional parameters needed to run the program:
energy range and number of energy values. Energies of incident or scattered
particles, direction of incident  or scattered particles. 

mode=4: E energy of incident particles, k11,k12,k13 direction if incident
particles (vector with arbitrary length), k21,k22,k23 direction of scattered
particles. 

mode=5: E energy of incident particles, k11,k12,k13 direction if incident
particles (vector with arbitrary length), k21,k22,k23 direction of
scattering vector $\vec Q$.

mode =6: E energy of scattered  particles, k11,k12,k13 direction if incident
particles (vector with arbitrary length), k21,k22,k23 direction of
scattered particles. 

\medskip
The parameterfile also contains the namme of a file with  a list of
scattering vectors  $\vec Q$ and energy loss ($\omega$) needed for mode
2,3

 Finally it contains the name of a file with the formfactor of the ion.  

\bigskip

J. Keller, May 2013

%\end{document}
